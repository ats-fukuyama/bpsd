\documentclass[11pt]{article}
\usepackage{../../doc/af}
\usepackage{../../doc/sym}
\usepackage{../../doc/doc}

\itemsep=0mm

\begin{document}
\begin{flushright}
2021/08/13
\end{flushright}

\begin{center}
\textbf{\Large User Manual of the TASK/EQ Code}
\end{center}

\tableofcontents

\section{How to run TASK/EQ}

\subsection{Start TASK/EQ}

Move to the task/eq directory and run the eq code by entering the file
name.  In order to distinguish it from command names, it should be
full path name or ./file\_name for the file in the present directory.
\begin{itemize}
\item[\qquad]\tline{cd eq}
\item[\qquad]\tline{./eq}
\end{itemize}

\subsubsection{Graphic setting}

At the beginning of the task code run, graphic settings are requested.

The first inquiry is the resolution of the graphic output. Enter a one
character and hit return.
\begin{itemize}
\item[]
0: no graphic output; graphic data can be saved in a file
\item[]
1: 512x380
\item[]
2: 640x475
\item[]
3: 768x570
\item[]
4: 896x665
\item[]
5: 1024x760
\item[]
6: 1280x950
\item[]
7-9: obsolete; to be deleted
\end{itemize}

The second inquiry is a one-character command to define the output of
graphic data.
\begin{itemize}
\item[]
C: continue without graphic output
\item[]
F: save graphic data to a file; the file name to be used will be asked
\item[]
O: option setting; paper size, screen title, file output are
inquired
\item[]
H: help for one character command input at each end of page
\item[]
Q: quit; terminate the code
\end{itemize}

Graphic setting can be skipped by defining a environmental variable
GSGDP.
\begin{itemize}
\item[]
export GSGDP=''5c''
\end{itemize}

\subsubsection{Menu input}

After the graphic setting, the TASK component EQ show a menu and
accept a line of command input, one character command or parameter
change.

\begin{itemize}
\item
Typical one character command (the first character of a line, both
upper- and lower-case work)
\begin{itemize}
\item[]
P: change parameter as a namelist input with group name eq
\item[]
V: show input parameters
\item[]
R: start a run after initialization
\item[]
C: continue the run
\item[]
G: graphic output mode
\item[]
S: save present status data
\item[]
L: load previously save status data
\item[]
Q: quit the component
\end{itemize}
\item
Parameter change
\begin{itemize}
\item
You can change the input parameters listed in eqinit.f90 with a form
similar to a component of the namelist input; for example
\begin{itemize}
\item[]
\ttype{rr=3.D0}
\item[]
\ttype{pn(1)=0.5D0, pn(2)=0.4D0, pn(3)=0.1D0}
\item[]
\ttype{pn=0.5D0,0.4D0,0.1D0}
\item[]
\ttype{modelg=3}
\item[]
\ttype{knameq='eqdata.ITER'}
\end{itemize}

\item
A input line including a character ``='' is considered as a parameter change.
\end{itemize}
\end{itemize}

\subsection{Input parameter}

The input parameters and their default values are defined in
eqinit.f90.  The sequence of parameter setting is as follows.

\begin{enumerate}
\item
The default parameters are defined in eqinit.f90.
\item
At the beginning of executing eq, if there exists a namelist file
eqparm or eqparm.nl, the component reads the namelist file. The
namelist file has a form
\begin{itemize}
\item[] \ \&eq
\item[] \ \ rr=3.D0
\item[] \ \&end
\end{itemize}
\item
During the operation of the eq component, input parameters can be
changed through the menu interface.  The command line input like
\ttype{rr=3.D0} or the namelist input through``P'' command changes the
input parameters.
\end{enumerate}

\subsection{Save graphic data}

Entering ``G'' command for the menu input, the graphic command input
is prompted.  The graphic command, a sequence of characters, depends
on the TASK component.  ``X'' command is to exit the graphic menu and
go back to the main menu. 

The graphic output is page base.  After drawing a page, a key input is
waited on the graphic window, unless ``0'' graphic mode is chosen at
the beginning of component operation. 

Following key inputs are acceptable:
\begin{itemize}
\item[] C or Return key : continue operation
\item[] F: open graphic file and start to save following pages
\item[] S: save this page, open graphic file if not yet opened
\item[] Y: save this page
\item[] N: do not save this page
\item[] X: switch on/off of saving pages\item[] B: switch on/off the bell sound at the end of drawing page
\item[] D: dump a bitmap of this page to a file and draw it in a new window
\item[] K: keep this page without erase and overdraw the next page
\item[] O: change options (page size, page title, file save)
\item[] H: help, show this information
\item[] Q: quit the component after confirmation
\end{itemize}

The graphic date is composed of ASCII-text data, and
machine-independent.  It can be viewed, converted to EPS file or PS
file, and printed on a postscript printer.  The recommended extension
of the graphic date file is ``.gs''.

\subsection{View and convert the graphic data}

The graphic data file of GSAF graphic library can be viewed on the X11
screen, converted to EPS file, PS file and SVG file, printed in a
postscript printer.

\begin{itemize}
\item[] gsview: view on a X-window screen
\item[] gstoeps: convert selected pages to separate EPS files
\item[] gstops: convert selected pages to a combined postscript file
\item[] gstosvg: convert selected pages to separate SVG files
\item[] gsprint: print on a postscript printer
\end{itemize}

The available options of these commands are
\begin{itemize}
\item[]
usage: gsxxxx [-atbrcmgz] [-s ps] [-e pe] [-p np] [filename]
\item[\quad]-a    : show all page
\item[\quad]-s ps : show from page ps [1]
\item[\quad]-e pe : show until page pe [999]
\item[\quad]-p np : combine np pages on a sheet [1]
\item[\quad]-t    : keep original page title
\item[\quad]-b    : no page title
\item[\quad]-r    : rotate figure, valid for gstops/eps 
\item[\quad]-c    : color figure, valid for gstops/eps (default)
\item[\quad]-z    : gray figure, valid for gstops/eps 
\item[\quad]-m    : monochrome figure, valid for gstops/eps 
\item[\quad]-g    : gouraud shading, valid for gstops/eps 
\item[\quad]filename : if not specified, prompted
\end{itemize}

The EPS file using gouraud shading cannot be edited by Adobe Illustrator.

\section{Structure of TASK/EQ}

\subsection{Support directories}

The following directories in a component directory includes:
\begin{itemize}
\item[] in/ test input files eq.inITER...
\item[] parm/ test parameter files, eqparm...
\item[] mod/ module files (generated during compiling fortran files)
\end{itemize}

\subsection{Source file name}

File name of main components

\begin{itemize}
\item[] eqcomm.f: Definition of variable module, allocation of arrays
\item[] eqmain.f: Initialization, read paramater file, start menu,
 and termination.
\item[] eqmenu.f: Show menu, and accept command input
\item[] eqinit.f: Set default values of input parameters
\item[] eqparm.f: Read input parameters (at present, in eqinit.f)
\item[] eqview.f: Show input parameters (at present, in eqinit.f)
\item[] eqcalc.f: Solve fixed-boundary equilibrium
\item[] eqcalq.f: calculate rectangular mesh data
\item[] eqfile.f: Save and load equilibrium data
\item[] eqgout.f: Graphic output
\end{itemize}

\section{Run examples}

\begin{itemize}
\item[] Example 1:
\item[\quad] start eq from xterm window
\item[\quad] \ttype{./eq}
\item[\quad] graphic setting
\item[\quad] \ttype{5}
\item[\quad] \ttype{c}
\item[\quad] run equilibrium solver
\item[\quad] \ttype{r}
\item[\quad] show results by graphics
\item[\quad] \ttype{g}
\item[\quad] show fixed boundary solution calculated by eqcalc
\item[\quad] \ttype{c}
\item[\quad] input ten carriage returns to show a sequence of graphs
\item[\quad] show equilibrium mesh data calculated by eqcalq
\item[\quad] \ttype{s}
\item[\quad] input ten carriage returns to show a sequence of graphs
\item[\quad] exit graphics and return to main menu
\item[\quad] \ttype{x}
\item[\quad] change paramter: elongation 1.5
\item[\quad] \ttype{rkap=1.5}
\item[\quad] run equilibrium solver again
\item[\quad] \ttype{r}
\item[\quad] show results by graphics
\item[\quad] \ttype{g}
\item[\quad] show equilibrium mesh data calculated by eqcalq
\item[\quad] \ttype{s}
\item[\quad] input ten carriage returns to show a sequence of graphs
\item[\quad] exit graphics and return to main menu
\item[\quad] \ttype{x}
\item[\quad] quit eq run
\item[\quad] \ttype{q}
\item[\quad]
  
\item[] Example 2:
\item[\quad] copy parm/eqparm.ITER eqparm
\item[\quad] \ttype{cp parm/eqparm .}
\item[\quad] run eq
\item[\quad] \ttype{./eq}
\item[\quad] graphic setting with graphic data output
\item[\quad] \ttype{5}
\item[\quad] \ttype{f}
\item[\quad] \ttype{eq.gs}
\item[\quad] \ttype{c}
\item[\quad] run equilibrium solver
\item[\quad] \ttype{r}
\item[\quad] show results by graphics
\item[\quad] \ttype{g}
\item[\quad] \ttype{c}
\item[\quad] \ttype{s}
\item[\quad] \ttype{x}
\item[\quad] quit eq run
\item[\quad] \ttype{q}
\item[\quad] view graphic data
\item[\quad] \ttype{gsview eq.gs}
\item[\quad] \ttype{5}
\item[\quad] \ttype{c}
\item[\quad] to show all pages
\item[\quad] \ttype{0}
\item[\quad] quit gsview
\item[\quad] \ttype{CTRL-D}
\item[\quad]

\item[\quad]
\item[] Example 3:
\item[\quad] remove eqparm if exists
\item[\quad] \ttype{rm eqparm}
\item[\quad] start run copy in/eq.inITER .
\item[\quad] \ttype{./eq <in/eq.inITER >eq.outITER}
\item[\quad] convert all graphic data to a ps file
\item[\quad] \ttype{gstops -ab gsdata.ITER >eq.ITER.ps}
\item[\quad] (macos only) view ps file and save it to pdf file
\item[\quad] \ttype{open eq.ITER.ps}
\end{itemize}

\section{Input parameters}

{\footnotesize
\begin{verbatim}
C     $Id$
C
C     ****** DEFAULT PARAMETERS ******
C
C     ======( DEVICE PARAMETERS )======
C
C        RR    : Plasma major radius                             (m)
C        RA    : Plasma minor radius                             (m)
C        RB    : Wall minor radius                               (m)
C        RKAP  : Plasma shape elongation
C        RDLT  : Plasma shape triangularity *
C        BB    : Magnetic field at center                        (T)
C        Q0    : Safety factor at center
C        QA    : Safety factor on plasma surface
C        RIP   : Plasma current                                 (MA)
C        FRBIN : (RB_inside-RA)/(RB_outside-RA)
C
      RR    = 3.D0
      RA    = 1.D0
      RB    = 1.2D0
      RKAP  = 1.D0
      RDLT  = 0.D0
C
      BB    = 3.D0
      Q0    = 1.D0
      QA    = 3.D0
      RIP   = 3.D0
C
      FRBIN = 1.D0
      RBRA  = RB/RA
C
C     ======( PLASMA PARAMETERS )======
C
C        NSMAX : Number of particle species
C        PA    : Mass number
C        PZ    : Charge number
C        PN    : Density at center                     (1.0E20/m**3)
C        PNS   : Density on plasma surface             (1.0E20/m**3)
C        PTPR  : Parallel temperature at center                (keV)
C        PTPP  : Perpendicular temperature at center           (keV)
C        PTS   : Temperature on surface                        (keV)
C        PU    : Toroidal rotation velocity at center          (m/s)
C        PUS   : Toroidal rotation velocity on surface         (m/s)
C        RHOITB: rho at ITB (0 for no ITB)
C        PNITB : Density increment at ITB              (1.0E20/Mm*3)
C        PTITB : Temperature increment at ITB                  (keV)
C        PUITB : Toroidal rotation velocity increment at ITB   (m/s)
C
      NSMAX = MIN(2,NSM)
C
         PA(1)   = AME/AMP
         PZ(1)   =-1.0D0
         PN(1)   = 1.0D0
         PNS(1)  = 0.0D0
         PTPR(1) = 5.0D0
         PTPP(1) = 5.0D0
         PTS(1)  = 0.05D0
         PU(1)   = 0.D0
         PUS(1)  = 0.D0
         RHOITB(1)= 0.D0
         PNITB(1)= 0.D0
         PTITB(1)= 0.D0
         PUITB(1)= 0.D0
C
      IF(NSM.GE.2) THEN
         PA(2)   = 1.0D0
         PZ(2)   = 1.0D0
         PN(2)   = 1.0D0
         PNS(2)  = 0.0D0
         PTPR(2) = 5.0D0
         PTPP(2) = 5.0D0
         PTS(2)  = 0.05D0
         PU(2)   = 0.D0
         PUS(2)  = 0.D0
         RHOITB(2)= 0.D0
         PNITB(2)= 0.D0
         PTITB(2)= 0.D0
         PUITB(2)= 0.D0
      ENDIF
C
      DO NS=3,NSM
         PA(NS)   = 1.0D0
         PZ(NS)   = 1.0D0
         PN(NS)   = 0.0D0
         PNS(NS)  = 0.0D0
         PTPR(NS) = 5.0D0
         PTPP(NS) = 5.0D0
         PTS(NS)  = 0.0D0
         PU(NS)   = 0.D0
         PUS(NS)  = 0.D0
         RHOITB(NS)= 0.D0
         PNITB(NS)= 0.D0
         PTITB(NS)= 0.D0
         PUITB(NS)= 0.D0
      ENDDO
C
C     ======( PROFILE PARAMETERS )======
C
C
C        PROFN1: Density profile parameter (power of rho)
C        PROFN2: Density profile parameter (power of (1 - rho^PROFN1))
C        PROFT1: Temperature profile parameter (power of rho)
C        PROFT2: Temperature profile parameter (power of (1 - rho^PROFN1))
C        PROFU1: Rotation profile parameter (power of rho)
C        PROFU2: Rotation profile parameter (power of (1 - rho^PROFN1))
C
      DO NS=1,NSM
         PROFN1(NS)= 2.D0
         PROFN2(NS)= 0.5D0
         PROFT1(NS)= 2.D0
         PROFT2(NS)= 1.D0
         PROFU1(NS)= 2.D0
         PROFU2(NS)= 1.D0
      END DO
C
C     ======( MODEL PARAMETERS )======
C
C        MODELG: Control plasma geometry model
C                   0: Slab geometry
C                   1: Cylindrical geometry
C                   2: Toroidal geometry
C                   3: TASK/EQ output geometry
C                   4: VMEC output geometry
C                   5: EQDSK output geometry
C                   6: Boozer output geometry
C        MODELN: Control plasma profile
C                   0: Calculated from PN,PNS,PTPR,PTPP,PTS,PU,PUS; 0 in SOL
C                   1: Calculated from PN,PNS,PTPR,PTPP,PTS,PU,PUS; PNS in SOL
C                   7: Read from file by means of WMDPRF routine (DIII-D)
C                   8: Read from file by means of WMXPRF routine (JT-60)
C                   9: Read from file KNAMTR (TASK/TR)
C        MODELQ: Control safety factor profile (for MODELG=0,1,2)
C                   0: Parabolic q profile (Q0,QA,RHOMIN,RHOITB)
C                   1: Given current profile (RIP,PROFJ0,PROFJ1,PROFJ2)
C
      MODELG= 2
      MODELN= 0
      MODELQ= 0
C
C        RHOMIN: rho at minimum q (0 for positive shear)
C        QMIN  : q minimum for reversed shear
C        RHOEDG: rho at EDGE for smoothing (1 for no smooth)
C
      RHOMIN = 0.D0
      QMIN   = 1.5D0
      RHOEDG = 1.D0
C
C     ======( GRAPHIC PARAMETERS )======
C
C        RHOGMN: minimum rho in radial profile
C        RHOGMX: maximum rho in radial profile
C
      RHOGMN = 0.D0
      RHOGMX = 1.D0
C
C     ======( MODEL PARAMETERS )======
C
C        KNAMEQ: Filename of equilibrium data
C        KNAMWR: Filename of ray tracing data
C        KNAMWM: Filename of full wave data
C        KNAMFP: Filename of Fokker-Planck data
C        KNAMFO: Filename of File output
C        KNAMPF: Filename of profile data
C        KNAMEQ2:Filename of addisional equilibrium data
C
      KNAMEQ = 'eqdata'
      KNAMWR = 'wrdata'
      KNAMWM = 'wmdata'
      KNAMFP = 'fpdata'
      KNAMFO = 'fodata'
      KNAMPF = 'pfdata'
      KNAMEQ2= 'eqdata2'
C
      NRMAXPL= 100
      NSMAXPL= NSMAX
C
      IDEBUG = 0
C
C     *** PROFILE PARAMETERS ***
C
C        PP0   : Plasma pressure (main component)              (MPa)
C        PP1   : Plasma pressure (sub component)               (MPa)
C        PP2   : Plasma pressure (increment within ITB)        (MPa)
C        PROFP0: Pressure profile parameter
C        PROFP1: Pressure profile parameter
C        PROFP2: Pressure profile parameter
C
C        PPSI=PP0*(1.D0-PSIN**PROFR0)**PROFP0
C    &       +PP1*(1.D0-PSIN**PROFR1)**PROFP1
C    &       +PP2*(1.D0-(PSIN/PSIITB)**PROFR2)**PROFP2
C
C        The third term exists for RHO < RHOITB
C
      PP0    = 0.001D0
      PP1    = 0.0D0
      PP2    = 0.0D0
      PROFP0 = 1.5D0
      PROFP1 = 1.5D0
      PROFP2 = 2.0D0
C
C        PJ0   : Current density at R=RR (main component) : Fixed to 1
C        PJ1   : Current density at R=RR (sub component)       (arb)
C        PJ2   : Current density at R=RR (sub component)       (arb)
C        PROFJ0: Current density profile parameter
C        PROFJ1: Current density profile parameter
C        PROFJ2: Current density profile parameter
C
C      HJPSI=-PJ0*(1.D0-PSIN**PROFR0)**PROFJ0
C     &                *PSIN**(PROFR0-1.D0)
C     &      -PJ1*(1.D0-PSIN**PROFR1)**PROFJ1
C     &                *PSIN**(PROFR1-1.D0)
C     &      -PJ2*(1.D0-PSIN**PROFR2)**PROFJ2
C     &                *PSIN**(PROFR2-1.D0)
C
C        The third term exists for RHO < RHOITB
C
      PJ0    = 1.00D0
      PJ1    = 0.0D0
      PJ2    = 0.0D0
      PROFJ0 = 1.5D0
      PROFJ1 = 1.5D0
      PROFJ2 = 1.5D0
C
C        FF0   : Current density at R=RR (main component) : Fixed to 1
C        FF1   : Current density at R=RR (sub component)       (arb)
C        FF2   : Current density at R=RR (sub component)       (arb)
C        PROFF0: Current density profile parameter
C        PROFF1: Current density profile parameter
C        PROFF2: Current density profile parameter
C
C      FPSI=BB*RR
C     &      +FF0*(1.D0-PSIN**PROFR0)**PROFF0
C     &      +FF1*(1.D0-PSIN**PROFR1)**PROFF1
C     &      +FF2*(1.D0-PSIN**PROFR2)**PROFF2
C
C        The third term exists for RHO < RHOITB
C
      FF0    = 1.0D0
      FF1    = 0.0D0
      FF2    = 0.0D0
      PROFF0 = 1.5D0
      PROFF1 = 1.5D0
      PROFF2 = 1.5D0
C
C        PT0   : Plasma temperature (main component)           (keV)
C        PT1   : Plasma temperature (sub component)            (keV)
C        PT2   : Plasma temperature (increment within ITB)     (keV)
C        PTSEQ : Plasma temperature (at surface)               (keV)
C        PROFTP0: Temperature profile parameter
C        PROFTP1: Temperature profile parameter
C        PROFTP2: Temperature profile parameter
C
C        TPSI=PTSEQ+(PT0-PTSEQ)*(1.D0-PSIN**PROFR0)**PROFTP0
C    &       +PT1*(1.D0-PSIN**PROFR1)**PROFTP1
C    &       +PT2*(1.D0-PSIN/PSIITB)**PROFR2)**PROFTP2
C    &       +PTSEQ
C
C        The third term exits for RHO < RHOITB
C
      PT0    = 1.0D0
      PT1    = 0.0D0
      PT2    = 0.0D0
      PTSEQ  = 0.05D0
C
      PROFTP0 = 1.5D0
      PROFTP1 = 1.5D0
      PROFTP2 = 2.0D0
C----
C        PV0   : Toroidal rotation (main component)              (m/s)
C        PV1   : Toroidal rotation (sub component)               (m/s)
C        PV2   : Toroidal rotation (increment within ITB)        (m/s)
C        PROFV0: Velocity profile parameter
C        PROFV1: Velocity profile parameter
C        PROFV2: Velocity profile parameter
C
C        PVSI=PV0*(1.D0-PSIN**PROFR0)**PROFV0
C    &       +PV1*(1.D0-PSIN**PROFR1)**PROFV1
C    &       +PV2*(1.D0-(PSIN/PSIITB)**PROFR2)**PROFV2
C
C        The third term exits for RHO < RHOITB
C
      PV0    = 0.0D0
      PV1    = 0.0D0
      PV2    = 0.0D0
      PROFV0 = 1.5D0
      PROFV1 = 1.5D0
      PROFV2 = 2.0D0
C
C        PN0EQ : Plasma number density(constant)
C
      PN0EQ  = 1.D20
C
C        PROFR0: Profile parameter
C        PROFR1: Profile parameter
C        PROFR2: Profile parameter
C        RHOITB: Normalized radius SQRT(PSI/PSIA) at ITB
C
      PROFR0 = 1.D0
      PROFR1 = 2.D0
      PROFR2 = 2.D0
C
C        OTC   : Constant OMEGA**2/TPSI
C        HM    : Constant                                       (Am)
C
      OTC = 0.15D0
      HM  = 1.D6
C
C     *** MESH PARAMETERS ***
C
C        NSGMAX: Number of radial mesh points for Grad-Shafranov eq.
C        NTGMAX: Number of poloidal mesh points for Grad-Shafranov eq.
C        NUGMAX: Number of radial mesh points for flux-average quantities
C        NRGMAX: Number of horizontal mesh points in R-Z plane
C        NZGMAX: Number of vertical mesh points in R-Z plane
C        NPSMAX: Number of flux surfaces
C        NRMAX : Number of radial mesh points for flux coordinates
C        NTHMAX: Number of poloidal mesh points for flux coordinates
C        NSUMAX: Number of boundary points
C        NRVMAX: Number of radial mesh of surface average
C        NTVMAX: Number of poloidal mesh for surface average
C
      NSGMAX = 32
      NTGMAX = 32
      NUGMAX = 32
C
      NRGMAX = 33
      NZGMAX = 33
      NPSMAX = 21
C
      NRMAX  = 50
      NTHMAX = 64
      NSUMAX = 65
C
      NRVMAX = 50
      NTVMAX = 200
C
C     *** CONTROL PARAMETERS ***
C
C        EPSEQ  : Convergence criterion for equilibrium
C        NLPMAX : Maximum iteration number of EQ
C        EPSNW  : Convergence criterion for newton method
C        DELNW  : Increment for derivative in newton method
C        NLPNW  : Maximum iteration number in newton method
C
      EPSEQ  = 1.D-6
      NLPMAX = 20
      EPSNW  = 1.D-2
      DELNW  = 1.D-2
      NLPNW  = 20
C
C        MDLEQF : Profile parameter
C            0: given analytic profile  P,J_tor,T,Vph + Ip
C            1: given analytic profile  P,F           + Ip
C            2: given analytic profile  P,J_para      + Ip
C            3: given analytic profile  P,J_para
C            4: given analytic profile  P,q
C            5: given spline profile    P,J_tor,T,Vph + Ip
C            6: given spline profile    P,F           + Ip
C            7: given spline profile    P,J_para      + Ip
C            8: given spline profile    P,J_para
C            9: given spline profile    P,q
C
      MDLEQF = 0
C
C        MDLEQA : Rho in P(rho), F(rho), q(rho),...
C            0: SQRT(PSIP/PSIPA)
C            1: SQRT(PSIT/PSITA)
C
      MDLEQA = 0
C
C        MDLEQC : Poloidal coordinate parameter
C            0: Poloidal length coordinate
C            1: Boozer coordinate
C
      MDLEQC = 0
C
C        MDLEQX : Free boundary calculation
C            0: Given PSIB and RIPFC
C            1: PSIB adjusted after loop for given RR,RA,RKAP,RDLT
C            2: PSIB adjusted eqch loop for given RR,RA,RKAP,RDLT
C
      MDLEQX = 0
C
C        MDLEQV : Order of extrapolation of psi in the vacuum region
C                 if positive, wall is linearly extended from plasma surface
C                 if negative, wall is extraporated by polynomials
C
      MDLEQV = 3
C
C        NPRINT: Level print out
C            0: no print
C            1: print first and last loop
C            2: print all loop
C
      NPRINT= 0
C
C        RGMIN: Minimum R of computation region [m]
C        RGMAX: Maxmum  R of computation region [m]
C        ZGMIN: Minimum Z of computation region [m]
C        ZGMAX: Maxmum  Z of computation region [m]
C        ZLIMP: Position of upper X points
C        ZLIMM: Position of lower X points
C
      RGMIN = 1.5D0
      RGMAX = 4.5D0
      ZGMIN =-2.0D0
      ZGMAX = 2.0D0
      ZLIMM =-2.5D0
      ZLIMP = 2.5D0
C
C        PSIB(0:5): Multipole moments of poloidal flux PSIRZ on boundary
C
      PSIB(0) =  2.0D0
      PSIB(1) =  0.5D0
      PSIB(2) =  0.D0
      PSIB(3) =  0.D0
      PSIB(4) =  0.D0
      PSIB(5) =  0.D0
C
C        NPFCMAX : Number of poloidal field coils (PFXs)
C        RIPFC(NPFC) : PFC coil current    [MA]
C        RPFC(NPFC)  : PFC coil position R [m]
C        ZPFC(NPFC)  : PFC coil position Z [m]
C        WPFC(NPFC)  : PFC coil width      [m]
C
      NPFCMAX = 0
      DO NPFC=1,NPFCM
         RIPFC(NPFC) = 0.D0
         RPFC(NPFC)  = 3.D0
         ZPFC(NPFC)  =-1.75D0
         WPFC(NPFC)  = 0.75D0
      ENDDO

      MODEFW=0 ! dangerous setting
      MODEFR=0 ! dangerous setting

\end{verbatim}
}

\end{document}
