\documentclass[12pt]{article}
\usepackage{geometry}
\geometry{a4paper}
\usepackage[dvips]{graphicx,color}
\usepackage{amsmath}
\usepackage{amssymb}
\usepackage[varg]{txfonts}
\usepackage{exscale}
\usepackage{af}
\usepackage{slide}
\usepackage{card}
\usepackage{sym}
\usepackage{ulem}

\newcommand{\epspath}{/Users/fukuyama/tex/eps}
\newcommand{\ttype}[1]{\emC{\LARGE \texttt{#1}}}
\newcommand{\tline}[1]{\qquad{\Large \texttt{#1}}\\[-3mm]}

%\let\uetheta=\ueth
%\let\uephi=\ueph
\let\Ra=\Rightarrow
\let\Lra=\Longrightarrow

\newenvironment{flist}{\par\noindent\begin{tabular}{%
p{45mm}p{35mm}p{30mm}p{120mm}}}{\end{tabular}\par\bigskip}
\newcommand{\fitem}[4]{\quad #1 & #2 & #3 & #4 \\}
\newcommand{\ffitem}[2]{\multicolumn{3}{p{109mm}}{#1} & #2 \\}
\newcommand{\fffitem}[1]{\multicolumn{4}{p{228mm}}{#1} \\}

\newcommand{\vvd}{\vv_\mr{d}}
\newcommand{\vd}{v_\mr{d}}
\newcommand{\vvE}{\vv_E}
\newcommand{\Epara}{E_\para}
\newcommand{\Tperp}{T_\perp}
\newcommand{\Tb}{T_\mr{b}}
\newcommand{\kparam}{k_{\para m}}
\newcommand{\energy}{\varepsilon}
\newcommand{\tx}{TASK/TX}

\newcommand{\xtitle}[2]{%
{\fboxsep=10mm \fboxrule=3mm
\cfbox{dbback}{#1}{%
\begin{center}\Huge
 \emC{#2}
\end{center}
}}
}

\begin{document}
\huge\sf

\begin{flushleft}
\fig{0.70}{\epspath/logo/KUlogo.eps}
\end{flushleft}
\vspace*{-40mm}

\begin{flushright}\LARGE
  2021/09/03
\end{flushright}
\vspace{20mm}

\xtitle{200mm}{%
How to install the integrated code: TASK
}
\medskip

\begin{center}\huge
\emB{A.~Fukuyama}
\\[8mm]
\textcolor{fgreen}{\huge
Professor Emeritus, Kyoto University 
}
\end{center}
\medskip

\begin{center}
  \begin{minipage}{130mm}
    \begin{enumerate}
    \item
      Preparation for macOS
    \item
      Preparation for Ubuntu
    \item
      Install PETSc
    \item
      Introduction to git
    \item
      Install task and related libraries
    \item
      Introduction to the task code
    \end{enumerate}
  \end{minipage}
\end{center}

\title{Preparation for macOS (1)}
\begin{itemize}
\item
  \emA{Install Xcode}
  \begin{itemize}
  \item
    Xcode: development environment on macOS
  \item
    Use App Store
  \item
    Category: Development
  \item
    Choose and install Xcode
  \end{itemize}
\item
  \emA{Install Command\_Line\_Tools}
  \begin{itemize}
  \item
    Command\_Line\_Tools: various Unix commands for development
  \item
    Input command on terminal
  \item
    \ttype{xcode-select --install}
  \end{itemize}
\item
  \emA{Install XQartz}
  \begin{itemize}
  \item
    Download and install XQartz from https://www.xquartz.org
  \end{itemize}
\item
  \emA{Install java}
  \begin{itemize}
  \item
    Download and install java from \\
    https://www.java.com/en/download/mac\_download.jsp
  \end{itemize}
\end{itemize}


\title{Preparation for macOS (2)}
\begin{itemize}
\item
  \emA{Install Macports}
  \begin{itemize}
  \item
    Download the latest macports binary from https://www.macports.org
  \item
    Select tab: Installing MacPorts
  \item
    Quickstart: download and install MacPorts installer for appropriate macOS
    version
  \item
    MacPorts is mostly installed at /opt/local
  \item
    Update to the latest Macports
    \begin{itemize}
    \item
      \ttype{sudo port selfupdate}
    \item
      \ttype{sudo port upgrade outdated}
    \end{itemize}
  \end{itemize}
\item
  \emA{Install compiler and related modules}
  \begin{itemize}
  \item
    gfortran: \ttype{sudo port install gcc11}
  \item
    mpich: \ttype{sudo port install mpich}
  \item
    others: \ttype{sudo port install gmake cmake imake}
  \end{itemize}
\end{itemize}

\title{Preparation for Ubuntu}
\begin{enumerate}
\item
  \emA{Install required modules}
\begin{verbatim}
sudo apt-get install gfortran-11
sudo apt-get install gcc-11
sudo apt-get install g++-11

sudo apt-get install git
sudo apt-get install xorg-dev
sudo apt-get install valgrind
sudo apt-get install cmake
sudo apt-get install python

sudo apt-get install mpich
\end{verbatim}
\end{enumerate}

\title{Install PETSc (1)}

\begin{itemize}
\item
  \emA{PETSc}: Parallel matrix solver library
  \begin{itemize}
  \item
    blas,lapack: matrix solver tolls
  \item
    scalapack, metis, parmetis, blacs, superlu: parallel solver tools
  \item
    MUMPS: Direct matrix solver for real and complex
  \item
    PETSc: Iterative matrix solver for real or complex
  \end{itemize}
\item
  \emA{Make PETSc directory and change its owner}
  \begin{itemize}
  \item
    \ttype{sudo mkdir /opt/PETSc}
  \item
    \ttype{sudo chown /opt/petsc \$USERNAME}
  \item
    \ttype{cd /opt/PETSc}
  \end{itemize}
\item
  \emA{Download latest PETSc library package by git}
  \begin{itemize}
  \item
    \emB{First download of PETSc source}
    \begin{itemize}
    \item
      \ttype{git clone -b release https://gitlab.com/petsc/petsc.git
        petsc}
    \end{itemize}
  \item
    \emB{In order to update PETSc source}
    \begin{itemize}
    \item
      \ttype{git pull}
    \end{itemize}
  \end{itemize}
\end{itemize}

\title{Install PETSc (2)}
\begin{itemize}
\item
  \emA{Provide environment variables for PETSC in $\sim$/.profile or .zprofile}
  \begin{itemize}
  \item
    \ttype{export PETSC\_DIR=/opt/PETSc/petsc}
  \item
    \ttype{export PETSC\_ARCH=default}
  \end{itemize}
  
\item
  \emA{Configure script in python }
  \begin{itemize}
  \item
    \emB{Copy \ttype{default.py} to \ttype{/opt/PETSc/petsc}}
  \item
    \emB{Provide exec attribute to \ttype{default.py}}
    \begin{itemize}
    \item
      \ttype{chmod 755 default.py}
    \end{itemize}
  \item
    \emB{Execute configuration script} \qquad \emCs{(It may take half an hour.)}
    \begin{itemize}
    \item
      \ttype{./default.py}
    \end{itemize}
  \item
    \emB{Additional libraries are created in default/externalpackages}
  \end{itemize}
\item
  \emA{Make and check PETSc library}
  \begin{itemize}
  \item
    \ttype{make} \qquad \emCs{(It may take half an hour.)}
  \item
    \ttype{make check}
  \end{itemize}
\end{itemize}

\title{How to use git (1)}
\begin{itemize}
\item
  \emA{git}: version and remote repository control facility
\item
\emA{Repositories}
\begin{itemize}
\item
\emB{local}: in your machine
\item
\emB{remotes}: in remote servers
\item
\emB{remotes/origin}: in default server: bpsi.nucleng.kyoto-u.ac.jp
\end{itemize}
\item
\emA{Branches}
\begin{itemize}
\item
\emC{There are several branches for code development}
\begin{itemize}
\item
\emB{master}: default, stable version, often rather old
\item
\emB{develop}: latest version, where I am working
\item
\emB{others}: branches for working specific modules
\end{itemize}
\item
\emBs{cd task}
\item
\emBs{git branch}\qquad : list branch names, local only
\item
\emBs{git branch -a}\qquad : list branch names, local and remote
\end{itemize}
\end{itemize}

\title{How to use git (2)}
\begin{itemize}
\item
\emA{To use develop branch}
\begin{itemize}
\item
\emC{Create local branch develop and associate it with remote develop}
\item
\emBs{git checkout -t -b develop origin/develop}
\item
\emBs{git branch}
\end{itemize}
\item
\emA{Change working branch}
\begin{itemize}
\item
\emBs{git checkout master}
\item
\emBs{git checkout develop}
\end{itemize}
\item
\emA{Update working branch}: download from remote repository
\begin{itemize}
\item
\emBs{git pull}
\begin{itemize}
\item
Your modification is kept, if committed.
\item
If uncommitted modification remains, no overwrite.
\item
use \emBs{git stash} to keep away your modification.
\item
If there are conflicts with your committed modification, the conflicts
are indicated in the file. Correct them and \emBs{git pull} again.
\end{itemize}
\end{itemize}
\end{itemize}

\title{How to use git (3)}
\begin{itemize}
\item
\emA{To check your modification}
\begin{itemize}
\item
\emBs{git status}
\end{itemize}
\item
\emA{To commit your modification with message}: only local depository
is updated.  message is required.
\begin{itemize}
\item
\emBs{git commit -a -m'\textit{message}'}
\end{itemize}
\item
\emA{To list all modification}
\begin{itemize}
\item
\emBs{git log}
\end{itemize}
\item
\emA{To show difference from committed repository}
\begin{itemize}
\item
\emBs{git diff [\textit{filename}]}
\end{itemize}
\item
\emA{For more detail, visit}
\begin{itemize}
\item
\emCs{https://git-scm.com/documentation}
\end{itemize}
\end{itemize}

\title{Install TASK (1)}
\begin{itemize}
\item
\emA{Check availability of git}: just command input ``git''
\item
\emA{Set your identity}: To record who changed the code?
\begin{itemize}
\item
\emBs{git config -\,-global user.name ``[your-full-name]''}
\item
\emBs{git config -\,-global user.email [your-mail-address]}
\item
For example,
\begin{itemize}
\item
git config -\,-global user.name ``Atsushi Fukuyama''
\item
git config -\,-global user.email fukuyama@nucleng.kyoto-u.ac.jp
\end{itemize}
\item
Data is saved in \$HOME/.gitconfig
\end{itemize}
\item
\emA{Create a working directory}: any directory name is OK
\begin{itemize}
\item
\emBs{mkdir git}
\item
\emBs{cd git}
\end{itemize}
\end{itemize}


\title{Install TASK (2)}
\begin{itemize}
\item
\emA{Download TASK and necessary libraries} for download only
\begin{itemize}
\item
\emBs{git clone https://bpsi.nucleng.kyoto-u.ac.jp/pub/git/gsaf.git}
\item
\emBs{git clone https://bpsi.nucleng.kyoto-u.ac.jp/pub/git/bpsd.git}
\item
\emBs{git clone https://bpsi.nucleng.kyoto-u.ac.jp/pub/git/task.git}
\end{itemize}
\item
\emA{Download TASK and necessary libraries} for download and upload
\begin{itemize}
\item
\emBs{git clone ssh://username@bpsi.nucleng.kyoto-u.ac.jp/pub/git/gsaf.git}
\item
\emBs{git clone ssh://username@bpsi.nucleng.kyoto-u.ac.jp/pub/git/bpsd.git}
\item
\emBs{git clone ssh://username@bpsi.nucleng.kyoto-u.ac.jp/pub/git/task.git}
\item
  \emBs{username@} can be omitted if the usernames at remote and local
  are same.
\end{itemize}
\item
\emC{Three directories are created}
\begin{itemize}
\item
\emB{gsaf}: graphic library
\item
\emB{bpsd}: data interface library
\item
\emB{task}: main TASK directory
\end{itemize}
\end{itemize}

\title{Install TASK (3)}
\begin{itemize}
\item
  \emA{Install graphic library GSAF}
  \begin{itemize}
  \item
    \emBs{cd git/gsaf/src}
  \item
    Copy Makefile.arch appropriate for your environment
    \begin{itemize}
    \item
      for macOS: \emBs{cp ../arch/macos-gfortran/Makefile.arch .}
    \item
      for Ubuntu: \emBs{cp ../arch/ubuntu-gfortran64-static/Makefile.arch .}
    \end{itemize}
  \item
    \emB{Edit Makefile.arch}: adjust BINPATH and LIBPATH to available ones
    \begin{itemize}
    \item
      BINPATH: graphic commands are located, should be included in
      \$PATH in $\sim$/.profile or ./zprofile
    \item
      LIBPATH: graphic libraries are located, should be included in
      library path for compiling applications using the graphic libs.
    \end{itemize}
  \item
    \emBs{make}
  \item
    \emBs{make install}
    \begin{itemize}
    \item
      if BINPATH is protected, use ``\emBs{sudo make install}''
    \end{itemize}
  \end{itemize}
\end{itemize}

\title{Install TASK (4)}
\begin{itemize}
\item
  \emA{Check the availability of GSAF library}
  \begin{itemize}
  \item
    \emBs{cd test}
  \item
    \emBs{make}
  \item
    \emAs{Applications using GSAF library must be started from X11
      window such as xterm, not from Terminal on macOS.}
  \item
    \emBs{./bsctest} 
  \item
    \emBs{5}\qquad: Choose the size of window
  \item
    \emBs{c}\qquad: Continue the run
  \item
    \emBs{m}
  \item
    New graphic window opens and marks and lines are drawn.
  \item
    To go back to the original window, enter CR.
  \item
    If focus does not change, click the original window and check XQartz
    preferences.
  \item
    \emBs{e}\qquad:  Close the graphic window
  \item
    \emBs{cd ../../..}
  \end{itemize}
\end{itemize}

\title{Install TASK (5)}
\begin{itemize}
\item
\emA{Use the latest develop branch and setup make.header file}
\begin{itemize}
\item
\emBs{cd task}
\item
\emBs{git checkout -t -b develop origin/develop}
\item
\emBs{cp make.header.org make.header}
\item
\emB{Edit make.header} to remove comments for target OS and compiler
\end{itemize}
\item
\emA{Compile data exchange library BPSD}
\begin{itemize}
\item
\emBs{cd ../bpsd}
\item
\emBs{make}
\item
\emBs{cd ../task}
\end{itemize}
\item
\emA{Compile TASK}: eq for example
\begin{itemize}
\item
\emBs{cd eq}
\item
\emBs{make}
\item
\emBs{./eq}
\end{itemize}
\end{itemize}

\title{Install TASK (6)}
\begin{itemize}
\item
\emA{Type of parallel matrix solver configuration}
\begin{itemize}
\item
\emBs{.nompi}: only single- processing solver without MPI environment
\item
\emBs{.mpi}: only single-processing solver with MPI environment
\item
\emBs{.petsc}: various parallel iterative solvers in  with MPI
\item
\emBs{.petsc+mumps}: various parallel iterative and direct solvers with MPI
\end{itemize}
\item
\emA{Setup matrix solver library}
\begin{itemize}
\item
\emBs{cd mtxp}
\item
\emBs{cp make.mtxp.XXX make.mtxp}
\item
\emBs{make}
\end{itemize}
\item
\emA{Compile modules}:
\begin{itemize}
\item
\emBs{Edit the beginning of Makefile}: Select matrix solver
\begin{itemize}
\item
\emC{Real} matrix equation (fp,ti): any mtxp library
\item
\emC{Complex} matrix equation (wm,wf2d,wf3d): band matrix or MUMPS
\end{itemize}
\end{itemize}
\end{itemize}

\title{How to use GSAF}
\begin{itemize}
\item
\emA{At the beginning of the program}
\begin{itemize}
\item
\emB{Set graphic resolution} (0: metafile output only, no graphics)
\item
\emB{commands}
\begin{itemize}
\item
\emC{c}: continue
\item
\emC{f}: set metafile name and start saving
\end{itemize}
\end{itemize}
\item
\emA{At the end of one page drawing}
\begin{itemize}
\item
\emB{commands}
\begin{itemize}
\item
\emC{c} or \emC{CR}: change focus to original window and continue
\item
\emC{f}: set metafile name and start saving
\item
\emC{s}: start saving and save this page
\item
\emC{y}: save this page and continue
\item
\emC{n}: continue without saving
\item
\emC{d}: dump this page as a bitmap file ``gsdump\textit{n}''
\item
\emC{b}: switch on/off bell sound
\item
\emC{q}: quit program after confirmation
\end{itemize}
\end{itemize}
\end{itemize}

\title{Graphic Utilities}
\begin{itemize}
\item
\emA{Utility program}
\begin{itemize}
\item
\emC{gsview}: View metafile
\item
\emC{gsprint}: Print metafile on a postscript printer
\item
\emC{gstoeps}: Convert metafile to eps files of each page
\item
\emC{gstops}: Convert metafile to a postscript file of all pages
\item
\emC{gstotgif}: Convert metafile to a tgif file for graphic editor tgif
\item
\emC{gstotsvg}: Convert metafile to a svg file for web browser
\end{itemize}
\item
\emA{Options}
\begin{itemize}
\item
\emC{-a}: output all pages, otherwise interactive mode
\item
\emC{-s ps}: output from page ps
\item
\emC{-e pe}: output until page pe
\item
\emC{-p \textit{np}}: output contiguous \textit{np} pages on a sheet
\item
\emC{-b}: output without title
\item
\emC{-r}: rotate page
\item
\emC{-z}: gray output
\end{itemize}
\end{itemize}

\title{Typical File Name of TASK}
\begin{itemize}
\itemsep=3mm
\item
\ttype{XXcomm.f90}: Definition of global variables, allocation of arrays
\item
\ttype{XXmain.f90}: Main program for standalone use, read XXparm file
\item
\ttype{XXmenu.f90}: Command input
\item
\ttype{XXinit.f90}: Default values
\item
\ttype{XXparm.f90}: Read input parameters
\item
\ttype{XXview.f90}: Show input parameters
\item
\ttype{XXprep.f90}: Initialization of run, initial profile
\item
\ttype{XXexec.f90}: Execution of run
\item
\ttype{XXgout.f90}: Graphic output
\item
\ttype{XXfout.f90}: Text file output
\item
\ttype{XXsave.f90}: Binary file output
\item
\ttype{XXload.f90}: Binary file input
\end{itemize}

\title{Typical input command}
\begin{itemize}
\item
When input line includes \emA{=}, interpreted as a namelist
input (e.g., \ttype{rr=6.5})
\item
When the first character is not an alphabet, interpreted as line input
\item
\ttype{r}: Initialize profiles and execute
\item
\ttype{c}: Continue run
\item
\ttype{p}: Namelist input of input parameters
\item
\ttype{v}: Display of input parameters
\item
\ttype{s}: Save results into a file
\item
\ttype{l}: Load results from a file
\item
\ttype{q}: End of the program
\item
\emB{Order of input parameter setting}
\begin{itemize}
\item
Setting at the subroutine \ttype{XX\_init} in \ttype{XXinit.f90}
\item
Read a namelist file \ttype{XXparm} at the beginning of the program
\item
Setting by the input line
\end{itemize}
\end{itemize}

\end{document}
