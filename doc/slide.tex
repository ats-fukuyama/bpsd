\documentclass[12pt]{article}
\usepackage{geometry}
\geometry{a4paper}
\usepackage[dvips]{graphicx,color}
\usepackage{amsmath}
\usepackage{amssymb}
\usepackage[varg]{txfonts}
\usepackage{exscale}
\usepackage{af}
\usepackage{slide}
\usepackage{card}
\usepackage{sym}
\usepackage{ulem}

\newcommand{\epspath}{/Users/fukuyama/tex/eps}
\newcommand{\ttype}[1]{\emC{\LARGE \texttt{#1}}}
\newcommand{\tline}[1]{\qquad{\Large \texttt{#1}}\\[-3mm]}

%\let\uetheta=\ueth
%\let\uephi=\ueph
\let\Ra=\Rightarrow
\let\Lra=\Longrightarrow

\newenvironment{flist}{\par\noindent\begin{tabular}{%
p{45mm}p{35mm}p{30mm}p{120mm}}}{\end{tabular}\par\bigskip}
\newcommand{\fitem}[4]{\quad #1 & #2 & #3 & #4 \\}
\newcommand{\ffitem}[2]{\multicolumn{3}{p{109mm}}{#1} & #2 \\}
\newcommand{\fffitem}[1]{\multicolumn{4}{p{228mm}}{#1} \\}

\newcommand{\vvd}{\vv_\mr{d}}
\newcommand{\vd}{v_\mr{d}}
\newcommand{\vvE}{\vv_E}
\newcommand{\Epara}{E_\para}
\newcommand{\Tperp}{T_\perp}
\newcommand{\Tb}{T_\mr{b}}
\newcommand{\kparam}{k_{\para m}}
\newcommand{\energy}{\varepsilon}
\newcommand{\tx}{TASK/TX}

\newcommand{\xtitle}[2]{%
{\fboxsep=10mm \fboxrule=3mm
\cfbox{dbback}{#1}{%
\begin{center}\Huge
 \emC{#2}
\end{center}
}}
}

\begin{document}
\huge\sf

\begin{flushleft}
\fig{0.70}{\epspath/logo/KUlogo.eps}
\end{flushleft}
\vspace*{-40mm}

\begin{flushright}\LARGE
2021/08/10
\end{flushright}
\vspace{10mm}

\xtitle{200mm}{%
How to install the integrated code: TASK
}
\vspace{8mm}

\begin{center}\huge
\emB{A.~Fukuyama}
\\[8mm]
\textcolor{fgreen}{\huge
Professor Emeritus, Kyoto University 
}
\end{center}


\title{Preparation on macos (1)}
\begin{itemize}
\item
  \emA{Install of Xcode}
  \begin{itemize}
  \item
    Xcode: development environment on macos
  \item
    Use App Store
  \item
    Category: Development
  \item
    Choose and install Xcode
  \end{itemize}
\item
  \emA{Install of Command\_Line\_Tools}
  \begin{itemize}
  \item
    Command\_Line\_Tools: various Unix commands for development
  \item
    Input command on terminal
  \item
    \ttype{xcode-select --install}
  \end{itemize}
\item
  \emA{Install of XQartz}
  \begin{itemize}
  \item
    Download and install XQartz from https://www.xquartz.org
  \end{itemize}
\item
  \emA{Install of java}
  \begin{itemize}
  \item
    Download and install java from \\
    https://www.java.com/en/download/mac\_download.jsp
  \end{itemize}
\end{itemize}


\title{Preparation on macos (2)}
\begin{itemize}
\item
  \emA{Install of Macports}
  \begin{itemize}
  \item
    Download latest macports binary from https://www.macports.org
  \item
    tab: Installing MacPorts
  \item
    Quickstart: download and install MacPorts installer for appropriate macos
    version
  \item
    MacPorts is mostly installed at /opt/local
  \item
    Update to latest Macports
    \ttype{sudo port selfupdate}
  \end{itemize}
\item
  \emA{Install of compilers}
  \begin{itemize}
  \item
    gfortran: \ttype{sudo port install gcc11}
  \item
    mpich: \ttype{sudo port install mpich}
  \item
    others: \ttype{sudo port install gmake cmake imake}
  \end{itemize}
\end{itemize}

\title{Preparation on macos (3)}

\begin{itemize}
\item
  \emA{Install of MPI and parallel matrix solver library PETSc}
  \begin{itemize}
  \item
    MPICH, blas/lapack, scalapack, metis, parmetis, MUMPS are
    downloaded by git and installed during the configure process of PETSc
  \end{itemize}
\item
  \emA{Make PETSc directory and change its owner}
  \begin{itemize}
  \item
    \ttype{sudo mkdir /opt/petsc}
  \item
    \ttype{sudo chown /opt/petsc \$USERNAME}
  \item
    \ttype{cd /opt/petsc}
  \end{itemize}
\item
  \emA{Download of latest PETSc library package by git}
  \begin{itemize}
  \item
    \emB{First download of PETSc source}
    \begin{itemize}
    \item
      \ttype{git clone -b release https://gitlab.com/petsc/petsc.git
        petsc}
    \end{itemize}
  \item
    \emB{In order to update PETSc source}
    \begin{itemize}
    \item
      \ttype{git pull}
    \end{itemize}
  \end{itemize}
\end{itemize}

\title{Preparation on macos (4)}
\begin{itemize}
\item
  \emA{Provide environment variables for PETSC in $\sim$/.zprofile}
  \begin{itemize}
  \item
    \ttype{export PETSC\_DIR=/opt/petsc}
  \item
    \ttype{export PETSC\_ARCH=macports}
  \end{itemize}
  
\item
  \emA{Configure script in python }
  \begin{itemize}
  \item
    \emB{Copy \ttype{macports.py} to \ttype{/opt/petsc}}
  \item
    \emB{Provide exec attribute to \ttype{macports.py}}
    \begin{itemize}
    \item
      \ttype{chmod 755 macports.py}
    \end{itemize}
  \item
    \emB{Execute configuration script} \qquad \emCs{(It may take one hour.)}
    \begin{itemize}
    \item
      \ttype{./macports.py}
    \end{itemize}
  \item
    \emB{Additional libraries are created in macports/externalpackages}
  \end{itemize}
\item
  \emA{Make and check PETSc library}
  \begin{itemize}
  \item
    \ttype{make} \qquad \emCs{(It may take half an hour.)}
  \item
    \ttype{make check}
  \end{itemize}
\end{itemize}

\title{Install TASK (1)}
\begin{itemize}
\item
\emA{Check git available}: just command input ``git''
\item
\emA{Set your identity}: Who changed the code?
\begin{itemize}
\item
\emBs{git config -\,-global user.name ``[your-full-name]''}
\item
\emBs{git config -\,-global user.email [your-mail-address]}
\item
For example,
\begin{itemize}
\item
git config -\,-global user.name ``Atsushi Fukuyama''
\item
git config -\,-global user.email fukuyama@nucleng.kyoto-u.ac.jp
\end{itemize}
\item
Data is saved in \$HOME/.gitconfig
\end{itemize}
\item
\emA{Create a working directory}: any directory name is OK
\begin{itemize}
\item
\emBs{mkdir git}
\item
\emBs{cd git}
\end{itemize}
\end{itemize}


\title{Install TASK (2)}
\begin{itemize}
\item
\emA{Download TASK and necessary libraries} for download only
\begin{itemize}
\item
\emBs{git clone https://git@bpsi.nucleng.kyoto-u.ac.jp/pub/git/gsaf.git}
\item
\emBs{git clone https://git@bpsi.nucleng.kyoto-u.ac.jp/pub/git/bpsd.git}
\item
\emBs{git clone https://git@bpsi.nucleng.kyoto-u.ac.jp/pub/git/task.git}
\end{itemize}
\item
\emA{Download TASK and necessary libraries} for download and upload
\begin{itemize}
\item
\emBs{git clone ssh://username@bpsi.nucleng.kyoto-u.ac.jp/pub/git/gsaf.git}
\item
\emBs{git clone ssh://username@bpsi.nucleng.kyoto-u.ac.jp/pub/git/bpsd.git}
\item
\emBs{git clone ssh://username@bpsi.nucleng.kyoto-u.ac.jp/pub/git/task.git}
\end{itemize}
\item
\emC{Three directories are created}
\begin{itemize}
\item
\emB{gsaf}: graphic library
\item
\emB{bpsd}: data interface library
\item
\emB{task}: main TASK directory
\end{itemize}
\end{itemize}

\title{How to use git (1)}
\begin{itemize}
\item
\emA{Repositories}
\begin{itemize}
\item
\emB{local}: in your machine
\item
\emB{remotes}: in remote servers
\item
\emB{remotes/origin}: in default server: bpsi.nucleng.kyoto-u.ac.jp
\end{itemize}
\item
\emA{Branches}
\begin{itemize}
\item
\emC{There are several branches for code development}
\begin{itemize}
\item
\emB{master}: default, stable version, often rather old
\item
\emB{develop}: latest version, where I am working
\item
\emB{others}: branches for working specific modules
\end{itemize}
\item
\emBs{cd task}
\item
\emBs{git branch}\qquad : list branch names, local only
\item
\emBs{git branch -a}\qquad : list branch names, local and remote
\end{itemize}
\end{itemize}

\title{How to use git (2)}
\begin{itemize}
\item
\emA{To use develop branch}
\begin{itemize}
\item
\emC{Create local branch develop and associate it with remote develop}
\item
\emBs{git checkout -t -b develop origin/develop}
\item
\emBs{git branch}
\end{itemize}
\item
\emA{Change working branch}
\begin{itemize}
\item
\emBs{git checkout master}
\item
\emBs{git checkout develop}
\end{itemize}
\item
\emA{Update working branch}: download from remote repository
\begin{itemize}
\item
\emBs{git pull}
\begin{itemize}
\item
Your modification is kept, if committed.
\item
If uncommitted modification remains, no overwrite.
\item
use ``git stash'' to keep away your modification.
\item
If there is a conflict with your committed modification, conflict
are indicated in the file. Corrects them and ``git pull'' again.
\end{itemize}
\end{itemize}
\end{itemize}

\title{How to use git (3)}
\begin{itemize}
\item
\emA{To check your modification}
\begin{itemize}
\item
\emBs{git status}
\end{itemize}
\item
\emA{To commit your modification with message}: update local depository
\begin{itemize}
\item
\emBs{git commit -a -m'\textit{message}'}
\end{itemize}
\item
\emA{To list all modification}
\begin{itemize}
\item
\emBs{git log}
\end{itemize}
\item
\emA{To show difference from committed repository}
\begin{itemize}
\item
\emBs{git diff [\textit{filename}]}
\end{itemize}
\item
\emA{For more detail, visit}
\begin{itemize}
\item
\emCs{https://git-scm.com/documentation}
\end{itemize}
\end{itemize}

\title{Install TASK (3)}
\begin{itemize}
\item
\emA{Install graphic library GSAF} \qquad (start from directory git)
\begin{itemize}
\item
\emBs{cd gsaf/src}
\item
\emBs{cp ../arch/macosxi-gfortran64/Makefile.arch .}
\item
\emB{Edit Makefile.arch}: adjust BINPATH and LIBPATH to available paths
\item
\emBs{make}
\item
\emBs{make install}\qquad: if necessary use ``sudo make install'' 
\item
\emBs{cd test}
\item
\emBs{make}
\item
\emBs{./bsctest}
\item
\emBs{5}
\item
\emBs{c}
\item
\emBs{m}: CR to change focus to original window
\item
\emBs{e}
\item
\emBs{cd ../../..}
\end{itemize}
\end{itemize}

\title{Install TASK (4)}
\begin{itemize}
\item
\emA{Setup make.header file}
\begin{itemize}
\item
\emBs{cd task}
\item
\emBs{cp make.header.org make.header}
\item
\emB{Edit make.header} to remove comments for target OS and compiler
\end{itemize}
\item
\emA{Compile data exchange library BPSD}
\begin{itemize}
\item
\emBs{cd ../bpsd}
\item
\emBs{make}
\item
\emBs{cd ../task}
\end{itemize}
\item
\emA{Compile TASK}: eq for example
\begin{itemize}
\item
\emBs{cd eq}
\item
\emBs{make}
\item
\emBs{./eq}
\end{itemize}
\end{itemize}

\title{Install TASK (5)}
\begin{itemize}
\item
\emA{Setup matrix solver library}
\begin{itemize}
\item
\emBs{cd mtxp}
\item
\emBs{cp make.mtxp.org make.mtxp}
\item
\emB{Edit make.mtxp} to remove comments for your configuration
\item
\emBs{make}
\end{itemize}
\item
\emA{Type of configuration}
\begin{itemize}
\item
\emBs{no MPI}: only direct band matrix solver and an iterative solver
\item
\emBs{with MPI}: only direct band matrix solver and an iterative solver
\item
\emBs{with MUMPS}: parallel direct solver
\item
\emBs{with PETSc}: various parallel iterative and direct solvers
\end{itemize}
\item
\emA{Compile modules}:
\begin{itemize}
\item
\emBs{Edit the beginning of Makefile}: Select matrix solver
\begin{itemize}
\item
\emC{Real} matrix equation (fp,ti): any mtxp library
\item
\emC{Complex} matrix equation (wm,wf2d,wf3d): band matrix or MUMPS
\end{itemize}
\end{itemize}
\end{itemize}

\title{How to use GSAF}
\begin{itemize}
\item
\emA{At the beginning of the program}
\begin{itemize}
\item
\emB{Set graphic resolution} (0: metafile output only, no graphics)
\item
\emB{commands}
\begin{itemize}
\item
\emC{c}: continue
\item
\emC{f}: set metafile name and start saving
\end{itemize}
\end{itemize}
\item
\emA{At the end of one page drawing}
\begin{itemize}
\item
\emB{commands}
\begin{itemize}
\item
\emC{c} or \emC{CR}: change focus to original window and continue
\item
\emC{f}: set metafile name and start saving
\item
\emC{s}: start saving and save this page
\item
\emC{y}: save this page and continue
\item
\emC{n}: continue without saving
\item
\emC{d}: dump this page as a bitmap file ``gsdump\textit{n}''
\item
\emC{b}: switch on/off bell sound
\item
\emC{q}: quit program after confirmation
\end{itemize}
\end{itemize}
\end{itemize}

\title{Graphic Utilities}
\begin{itemize}
\item
\emA{Utility program}
\begin{itemize}
\item
\emC{gsview}: View metafile
\item
\emC{gsprint}: Print metafile on a postscript printer
\item
\emC{gstoeps}: Convert metafile to eps files of each page
\item
\emC{gstops}: Convert metafile to a postscript file of all pages
\item
\emC{gstotgif}: Convert metafile to a tgif file for graphic editor tgif
\item
\emC{gstotsvg}: Convert metafile to a svg file for web browser
\end{itemize}
\item
\emA{Options}
\begin{itemize}
\item
\emC{-a}: output all pages, otherwise interactive mode
\item
\emC{-s ps}: output from page ps
\item
\emC{-e pe}: output until page pe
\item
\emC{-p \textit{np}}: output contiguous \textit{np} pages on a sheet
\item
\emC{-b}: output without title
\item
\emC{-r}: rotate page
\item
\emC{-z}: gray output
\end{itemize}
\end{itemize}

\title{Typical File Name of TASK}
\begin{itemize}
\itemsep=3mm
\item
\ttype{XXcomm.f90}: Definition of global variables, allocation of arrays
\item
\ttype{XXmain.f90}: Main program for standalone use, read XXparm file
\item
\ttype{XXmenu.f90}: Command input
\item
\ttype{XXinit.f90}: Default values (may still include XXparm.f90)
\item
\ttype{XXparm.f90}: Handling of input parameters
\item
\ttype{XXprep.f90}: Initialization of run, initial profile
\item
\ttype{XXexec.f90}: Execution of run
\item
\ttype{XXgout.f90}: Graphic output
\item
\ttype{XXfout.f90}: Text file output
\item
\ttype{XXsave.f90}: Binary file output
\item
\ttype{XXload.f90}: Binary file input
\end{itemize}

\title{Typical input command}
\begin{itemize}
\item
When input line includes \emA{=}, interpreted as a namelist
input (e.g., \ttype{rr=6.5})
\item
When the first character is not an alphabet, interpreted as line input
\item
\ttype{r}: Initialize profiles and execute
\item
\ttype{c}: Continue run
\item
\ttype{p}: Namelist input of input parameters
\item
\ttype{v}: Display of input parameters
\item
\ttype{s}: Save results into a file
\item
\ttype{l}: Load results from a file
\item
\ttype{q}: End of the program
\item
\emB{Order of input parameter setting}
\begin{itemize}
\item
Setting at the subroutine \ttype{XXinit} in \ttype{XXinit.f90}
\item
Read a namelist file \ttype{XXparm} at the beginning of the program
\item
Setting by the input line
\end{itemize}
\end{itemize}

\title{Install on Ubuntu (1)}
\begin{enumerate}
\item
  \emA{Install of required modules}
\begin{verbatim}
sudo apt-get install gfortran-11
sudo apt-get install gcc-11
sudo apt-get install g++-11

sudo apt-get install emacs
sudo apt-get install git
sudo apt-get install xorg-dev
sudo apt-get install valgrind
sudo apt-get install cmake
sudo apt-get install python
\end{verbatim}

\item
\emA{Install of MPICH}

\begin{enumerate}
\item
\emB{Download the latest mpich-n1.n2.n3.tar.gz from
ttype{http://www.mpich.org}}
\item
\emB{Expand at /opt/mpich}
\item
\emB{Configure by executing ``./run''}

\begin{quote}
run:
\hrule
\begin{verbatim}
CC=gcc-11 CFLAGS=''-m64'' CXX=g++-11 CXXFLAGS=''-m64'' FC=gfortran-11
FFLAGS=''-m64'' ./configure --prefix=/usr/local/mpich-gfortran-gcc
--enable-cxx --enable-fast --enable-romio --disable-shared
\end{verbatim}
\hrule
\end{quote}
\end{enumerate}

\item
  \emB{Compile and install}
  \begin{enumerate}
  \item
    \ttype{make}
  \item
    \ttype{make install}
  \end{enumerate}
\end{enumerate}


\title{Install on Ubuntu (2)}
\begin{enumerate}
\item
  \emA{Install of PETSc}
\begin{enumerate}
\item
Download of petsc-3.11.3.tar.gz \qquad (\textit{See 2.1})
\item
Expand at /opt/PETSc/ \qquad (\textit{See 2.1})
\item
Setup environment variables \qquad (\textit{See 2.2})
\begin{quote}
\begin{verbatim}
export PETSC_DIR=/opt/PETSc/petsc-3.11.3
export PETSC_ARCH=gfortran
\end{verbatim}
\end{quote}
\item
Configure by gfortran.py  \qquad (\textit{See 2.2})
\item
Execute fortran.py

\begin{quote}
gortran.py:
\hrule
\begin{verbatim}
#!/usr/bin/env python

# Build PETSc, with gfortran

configure_options = [
  '--with-mpi=1',
  '--with-mpi-dir=/usr/local/mpich331-gfortran-gcc8’,
  '--with-shared-libraries=0',
  '--with-cxx-dialect=C++11',
  '--download-mpich=0',
  '--download-hypre=0',
  '--download-fblaslapack=1',
  '--download-spooles=1',
  '--download-superlu=1',
  '--download-metis=1',
  '--download-parmetis=1',
  '--download-superlu_dist=1',
  '--download-blacs=1',
  '--download-scalapack=1',
  '--download-mumps=1'
  ]

if __name__ == '__main__':
  import sys,os
  sys.path.insert(0,os.path.abspath('config'))
  import configure
  configure.petsc_configure(configure_options)
\end{verbatim}
\hrule
\end{quote}
\end{enumerate}
\end{enumerate}


\end{document}
