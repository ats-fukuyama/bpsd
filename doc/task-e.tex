\documentclass[11pt]{article}
\usepackage{af}
\usepackage{sym}
\usepackage{doc}

\begin{document}
\begin{flushright}
2021/02/10
\end{flushright}

\begin{center}
\textbf{\Large User Manual of the TASK Code}
\end{center}

\tableofcontents

\section{What is the TASK code?}

The TASK (\textbf{T}ransport \textbf{A}nalyzing \textbf{S}ystem for
takama\textbf{K}) code is a suite of modeling tools for analyzing equilibrium,
transport, wave propagation, and velocity distribution function in
tokamak plasmas.

\subsection{Features of the TASK code}
\begin{itemize}
\item
\textbf{Simulation on time evolution of tokamak plasmas}
\begin{itemize}
\item
Integrated simulation code suite with modular structure
\item
Analysis of various heating and current drive mechanism
\item
High portability
\item
Extension to three-dimensional helical plasmas
\item
Parallel processing using MPI libraries
\item
Interface to experimental database
\end{itemize}
\item
\textbf{Core code suite in Burning Plasma Simulation Initiative}
\begin{itemize}
\item
Minimum integrated code: all components are exchangeable
\item
Test implementation of BPSD: standard data interface
\item
Unified user interface: 
\end{itemize}
\end{itemize}

\subsection{Module structure of the TASK code}
\begin{center}
\begin{tabular}{r|ll}
\hline
\texttt{TASK/EQ} & {2D equilibrium} & Fixed boundary \\
\texttt{EQU} & {2D equilibrium} & Free boundary \\[2mm]
\texttt{TR} & {1D transport} & Diffusive transport \\
\texttt{TX} & {1D transport} & Dynamic transport \\[2mm]
\texttt{WR} & {3D Geometrical Optics} & Ray/beam tracing \\
\texttt{WM} & {3D Full wave} & 2D FFT, 1D FDM \\
\texttt{WF2D} & {3D Full wave} & 1D FFT, 2D FEM \\
\texttt{WF3D} & {3D Full wave} & 3D FEM \\
\texttt{W1} & {1D Full wave} & MLM, FEM, differential, FLR integral \\
\texttt{WI} & {1D Full wave} & FEM, Landau/Cyclotron damping integral \\
\texttt{DP} & {Dispersion} & Dielectric tensor \\[2mm]
\texttt{FIT3D} & {NBI} & Birth, orbit, deposition \\[2mm]
\texttt{FP} & {Velocity distribution fn} & 3D Fokker-Planck equation \\[2mm]
\texttt{PLX} & {Profile data} & Initial profile, file interface \\
\texttt{LIB} & {Common library} & Matrix solver, FFT, spline,  \\
\texttt{MTXP} & {Matrix solver} & MPI, Direct, Iterative, parallelize \\
\texttt{TOT} & {Total code} & Integrated operation \\
\hline
\end{tabular}
\end{center}

\subsection{Specifications}

\begin{itemize}
\item
\textbf{Language}: FORTRAN95 
\begin{itemize}
\item
Allocatable array is used as a component of a derived type
\item
Graphic library requires C compiler and X11 library.
\end{itemize}
\item
\textbf{Compiler}: 
\begin{itemize}
\item
Intel fortran, gfortran, PGI fortran
\end{itemize}
\item
\textbf{Graphic library}: 
\begin{itemize}
\item
GSAF (available from the same git depository)
\item
X11 library is required for interactive graphic output.
\end{itemize}
\item
\textbf{Data interface library}: BPSD (available from the same git depository)
\item
\textbf{Parallelization}: MPI (if available)
\end{itemize}

\section{Install of the TASK code}

\subsection{To get the TASK code}

\begin{itemize}
\item
\textbf{Install of git}
\begin{itemize}
\item
The version control system ``git'' is required.
\item
If git is not installed in your computer system yet, you have to install
it.
\item
Git is an open source software and can be downloaded from
http://git-scm.com or installed through many distributions for Linux
and Mac OSX. A slightly older version for Mac OSX is included in
Xcode.
\item
Introduction to git is available from http://git-scm.com/book/en/v2 in
various languages and in various forms, HTML, PDF, and ePub.
\end{itemize}
\item
\textbf{Initial setup of git}
\begin{itemize}
\item
In order to record the modification of the code, git requires user's
name and email address.
\item
If you have not set up the git environment yet, you have to run the
following commands:
\item[\qquad]
\tline{git config --global user.name ``[your-full-name]''}
\item[\qquad]
\tline{git config --global user.email ``[your-mail-address]''}
\end{itemize}

\item
\textbf{Download of the source files}
\begin{itemize}
\item
Make a git directory  and download three repositories in it.

\item[\qquad]
\tline{mkdir git}
\item[\qquad]
\tline{cd git}
\item[\qquad]
\tline{git clone https://git@bpsi.nucleng.kyoto-u.ac.jp/pub/git/gsaf.git}
\item[\qquad]
\tline{git clone https://git@bpsi.nucleng.kyoto-u.ac.jp/pub/git/bpsd.git}
\item[\qquad]
\tline{git clone https://git@bpsi.nucleng.kyoto-u.ac.jp/pub/git/task.git}

\item
The password needed for ``git clone ...'' is ``git''.
\item
You can download the codes with this procedure, but upload requires a
user account on bpsi.nucleng.kyoto-u.ac.jp. If you have an interest in
uploading your contributions, please contact to fukuyama@nucleng.kyoto-u.ac.jp.
\item
HTTPS connection may require to run the following command,

\item[\qquad]
\tline{git config --global http.sslverify false}

to accept non-official SSL certificates.
\end{itemize}
\end{itemize}

\subsection{Compile of the TASK code}

\begin{itemize}
\item
Now you have three directories, GSAF, BPSD, and TASK.
\item
Then you compile them one by one.
\end{itemize}

\subsubsection{Compile of the GSAF library}

\begin{enumerate}
\item
Move to the source directory 
\item[\qquad]
\tline{cd gsaf/src}
\item
Copy an appropriate configuration file to the directory
\item[\qquad]
\tline{cp ../arch/XXXX-XXXX/Makefile.arch .}
\begin{itemize}
\item
linux-gfortran64: compile with gfortran and gcc on intel86-64 Linux
\item
linux-ifc64: compile with intel fortran ifort and icc on intel86-64 Linux
\item
linux-pgf64: compile with PGI fortran pgf64 and cc on intel86-64 Linux
\item
macosxi-gfortran64: compile with gfortran and gcc on Mac-OSX intel
\item
macosxi-ifc64: compile with intel fortran and gcc on Mac-OSX intel
\item
macosxi-pgf64: compile with PGI fortran pgf64 and cc on Mac-OSX intel
\end{itemize}
\item
Set appropriate directories for BINPATH and LIBPATH in Makefile.arch
\begin{itemize}
\item
If /usr/local is writable, use \ttype{/usr/local/bin} and 
\ttype{/usr/local/lib}. 
\item
If /usr/local is not writable, use \ttype{\$HOME/bin} and \ttype{\$HOME/lib}
\end{itemize}
\item
Compile the library
\item[\qquad]
\tline{make}
\item
Install the library and the commands
\item[\qquad]
\tline{make install}
\item
Include \$LIBPATH in the library searching path
\item[\qquad]
\ttype{export LD\underline{ }LIBRARY\underline{ }PATH=/usr/local/lib}
\item
Test the library installation
\item[\qquad]
\tline{cd test}
\item[\qquad]
\tline{make}
\item
Run the basic routine test
\item[\qquad]
\tline{./bsctest}
\item
Go back to the git directory
\item[\qquad]
\tline{cd ../..}
\end{enumerate}

\subsubsection{Select branch}

\begin{itemize}
\item
  There are several branches.
  \begin{itemize}
  \item
    master: default standard branch (updated not frequently, at present
    less than once a year)
  \item
    develop: development branch (updated frequently, sometimes broken
    in some directories)
    \tline{git checkout -t -b develop origin/develop}
  \end{itemize}
\end{itemize}

\subsubsection{Set up the configuration files and common library}

\begin{itemize}
\item
Copy make.header from make.header.org in the TASK directory 
\item[\qquad]
\tline{cd task}
\item[\qquad]
\tline{cp make.header.org make.header}
\item
Adjust the content of make.header
\begin{itemize}
\item
Remove the comment mark \# in appropriate lines.
\end{itemize}
\item
Compile the common library
\item[\qquad]
\tline{cd lib}
\item[\qquad]
\tline{make}
\item[\qquad]
\tline{cd ..}

\item
Some modules, such as FP, WMXX, WFXX, TI, utilize parallel matrix
solver to reduce computation time.  For compatibility, now most of
modules are coupled with parallel matrix solver library, mtxp.
Therefore mtxp library has to be compiled in given environment. 
\item
The first step is to copy mtxp/make.mtxp from mtxp/make.mtxp.org.
\item[\qquad]
\tline{cd mtxp}
\item[\qquad]
\tline{cp make.mtxp.org make.mtxp}
\item
Adjust the content of make.mtxp
\begin{itemize}
\item
Remove the comment mark \# in appropriate lines:
\begin{itemize}
\item
without MPI: no MPI library, band-matrix direct solver or simple
iterative solver are available
\item
with MPI: MPI library exits, but no parallel solver is available.
\item
with MUMPS: MPI library and parallel direct matrix solver MUMPS are available.
\item
with PETSc: MPI and parallel iterative   matrix solver library PETSc
are available.  PETSc may include MUMPS as well.
\end{itemize}
\item
compile the mtxp libraries
\item[\qquad]
\tline{make}
\item
Go back to the git directory
\item[\qquad]
\tline{cd ../..}
\end{itemize}
\end{itemize}

\subsubsection{Compile the BPSD library}

\begin{itemize}
\item
Compile in the bpsd directory
\item[\qquad]
\tline{cd BPSD}
\item[\qquad]
\tline{make}
\item[\qquad]
\tline{cd ..}
\end{itemize}

\subsubsection{Compile a required TASK component}

\begin{itemize}
\item
Move to the required TASK component, e.g. eq, and compile it.
\item[\qquad]
\tline{cd eq}
\item[\qquad]
\tline{make libs}
\item[\qquad]
\tline{make}
\item
If you would like to compile an integrated code, use tot.
\item[\qquad]
\tline{cd tot}
\item[\qquad]
\tline{make libs}
\item[\qquad]
\tline{make}
\item
If you modify program files in the directory, use
\item[\qquad]
\tline{make}
\\
to recompile the code.
\item
If you modify or update program files in other directories, such as
\ttype{lib} or \ttype{mtxp}, you may need 
\item[\qquad]
\tline{make libs}
\end{itemize}

\section{Run the TASK code}

\subsection{Run a TASK component}

TASK components start when the component name is entered as a
command, for example, 
\begin{itemize}
\item[\qquad]\tline{cd eq}
\item[\qquad]\tline{./eq}
\end{itemize}

\subsubsection{Graphic setting}

Most of TASK components employ the graphic library GSAF.  Therefore at
the beginning of execution (rigorously speaking, when the subroutine
GSOPEN is called), graphic settings are inquired. 

The first inquiry is the resolution of the graphic output. Enter a one
character and hit return.
\begin{itemize}
\item[]
0: no graphic output; graphic data can be saved in a file
\item[]
1: 512x380
\item[]
2: 640x475
\item[]
3: 768x570
\item[]
4: 896x665
\item[]
5: 1024x760
\item[]
6-9: obsolete; to be deleted
\end{itemize}

The second inquiry is to define output of graphic data.
\begin{itemize}
\item[]
C: continue without graphic output
\item[]
F: save graphic data to a file; the file name to be used will be asked
\item[]
O: option setting; paper size, screen title, file output are
inquired
\item[]
H: help for one character command input at each end of page
\item[]
Q: quit; terminate the code
\end{itemize}

Graphic setting can be skipped by defining a environmental variable
GSGDP.
\begin{itemize}
\item[]
export GSGDP=''3c''
\end{itemize}

\subsubsection{Menu input}

After graphic setting, the TASK component XX show a menu and accept
a command input, one character command or parameter change.

\begin{itemize}
\item
Typical one character command (the first character of a line, both
upper- and lower-case work)
\begin{itemize}
\item[]
P: change parameter as a namelist input with group name XX
\item[]
V: show input parameters
\item[]
R: start a run after initialization
\item[]
C: continue the run
\item[]
G: graphic output mode
\item[]
S: save present status data
\item[]
L: load previously save status data
\item[]
Q: quit the component
\end{itemize}
\item
Parameter change
\begin{itemize}
\item
You can change the input parameters listed in XXinit.f90 with a form
similar to a component of the namelist input; for example
\begin{itemize}
\item[]
\ttype{RR=3.D0}
\item[]
\ttype{PN(1)=0.5D0, PN(2)=0.4D0, PN(3)=0.1D0}
\item[]
\ttype{PN=0.5D0,0.4D0,0.1D0}
\item[]
\ttype{MODELG=3}
\item[]
\ttype{KNAMEQ='eqdata.ITER'}
\end{itemize}

\item
A input line including a character ``='' is considered as a parameter change.
\end{itemize}
\end{itemize}

\subsection{Run the tot component}

The ``tot'' component is a primitive control component which activates
main TASK components (EQ, TR, DP, WR, WM, FP) sequentially.  

The menu of the tot component accepts two-character commands of the
components and a quit commend ``Q''.  For example ``EQ'' command run
the task/eq component and after quitting the task/eq component, the tot
menu reappears.  

The data transfer between components will be made through the BPSD
library.  At present , however, some of the data transfer is made
through file I/O.  

\subsection{Parameter input}

The input parameters and their default values are usually defined in
XXinit.f90.  The sequence of parameter setting is as follows.

\begin{enumerate}
\item
The default parameters are defined in XXinit.f90.
\item
At the beginning of executing XX component, if there exists a namelist file
XXparm or XXparm.nl, the component reads the namelist file. The
namelist file has a form
\begin{itemize}
\item[] \ \&XX
\item[] \ \ RR=3.D0
\item[] \ \&end
\end{itemize}
\item
During the operation of the XX component, input parameters can be
changed through the menu interface.  The command line input like
\ttype{RR=3.D0} or the namelist input through``P'' command changes the
input parameters.
\end{enumerate}

We should note that some of input parameters are defined in a common
components, such as PLX or DP. 

\subsection{Save graphic data}

Entering ``G'' command for the menu input, the graphic command input
is prompted.  The graphic command is a sequence of characters defined
for each XX components.  ``X'' command is to exit the graphic menu and
go back to the main menu. 

The graphic output is page base.  After drawing a page, a key input is
waited on the graphic window, unless ``0'' graphic mode is chosen at
the beginning of component operation. 

Following key inputs are acceptable:
\begin{itemize}
\item[] C or Return key : continue operation
\item[] F: open graphic file and start to save following pages
\item[] S: save this page, open graphic file if not yet opened
\item[] Y: save this page
\item[] N: do not save this page
\item[] X: switch on/off of saving pages\item[] B: switch on/off the bell sound at the end of drawing page
\item[] D: dump a bitmap of this page to a file and draw it in a new window
\item[] K: keep this page without erase and overdraw the next page
\item[] O: change options (page size, page title, file save)
\item[] H: help, show this information
\item[] Q: quit the component after confirmation
\end{itemize}

The graphic date is composed of ASCII-text data, and
machine-independent.  It can be viewed, converted to EPS file, and
printed on a postscript printer.  The recommended extension of the
graphic date file is ``.gs''.

\subsection{View and convert the graphic data}

The graphic data file of GSAF graphic library can be viewed on the X11
screen, converted to EPS file, PS file and SVG file, printed in a
postscript printer.

\begin{itemize}
\item[] gsview: view on a X-window screen
\item[] gstoeps: convert selected pages to separate EPS files
\item[] gstops: convert selected pages to a combined postscript file
\item[] gstosvg: convert selected pages to separate SVG files
\item[] gsprint: print on a postscript printer
\end{itemize}

The available options of these commands are
\begin{itemize}
\item[]
usage: gsxxxx [-atbrcmgz] [-s ps] [-e pe] [-p np] [filename]
\item[\quad]-a    : show all page
\item[\quad]-s ps : show from page ps [1]
\item[\quad]-e pe : show until page pe [999]
\item[\quad]-p np : combine np pages on a sheet [1]
\item[\quad]-t    : keep original page title
\item[\quad]-b    : no page title
\item[\quad]-r    : rotate figure, valid for gstops/eps 
\item[\quad]-c    : color figure, valid for gstops/eps (default)
\item[\quad]-z    : gray figure, valid for gstops/eps 
\item[\quad]-m    : monochrome figure, valid for gstops/eps 
\item[\quad]-g    : gouraud shading, valid for gstops/eps 
\item[\quad]filename : if not specified, prompted
\end{itemize}

The EPS file using gouraud shading cannot be edited by Adobe Illustrator.

\section{Structure of a typical TASK component}

\subsection{Support directories}

The following directories in a component directory includes:
\begin{itemize}
\item[] in/ test input files XX.in (./XX <XX.in >XX.out)
\item[] parm/ test parameter files,, XXparm or XXparm.nl
\item[] mod/ module files (generated during compiling fortran files)
\end{itemize}

\begin{itemize}
\item[] Example 1:
\item[\quad] copy parm/eqparm.ITER eqparm
\item[\quad] ./eq
\item[\quad]
\item[] Example 2:
\item[\quad] copy in/eq.inITER .
\item[\quad] ./eq <eq.inITER >eq.outITER
\end{itemize}

\subsection{Typical file name}

The followings are typical file names of the components:

\begin{itemize}
\item[] XXcomm.f90: Definition of variable module, allocation of
  arrays
\item[] XXmain.f90: Initialization, read paramater file, call menu, termination.
\item[] XXmenu.f90: Process menu input
\item[] XXinit.f90: Default values of input parameters
\item[] XXparm.f90: Read and view input parameters (may be included in XXinit.f90)
\item[] XXprep.f90: Preparation of run (grid, initial profile)
\item[] XXexec.f90: Execution of run
\item[] XXsave.f90: Save run data (may be included in XXfile.f90)
\item[] XXsave.f90: Load run data (may be included in XXfile.f90)
\item[] XXgout.f90: Graphic output
\item[] XXfout.f90: File output of results
\end{itemize}

\end{document}
