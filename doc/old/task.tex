% $Id$
\documentclass[11pt]{jarticle}
\usepackage{af}
\usepackage{sym}
\usepackage{doc}

\begin{document}
\begin{flushright}
2008/01/28
\end{flushright}

\begin{center}
{\Large \gt TASK コードの利用説明書}
\end{center}                                   

\section{TASK コードとは}

TASK (\textbf{T}ransport \textbf{A}nalyzing \textbf{S}ystem for
takama\textbf{K}) コードは,主にトカマクプラズマにおける平衡,輸送,波
動伝播,速度分布を解析するコード群である.

\subsection{TASK コードの特色}
\begin{itemize}
\item
\textbf{トカマクの時間発展シミュレーション}
\begin{itemize}
\item
モジュール構造の統合シミュレーション
\item
様々な加熱・電流駆動機構の解析
\item
高い移植性
\item
ヘリカル系への拡張
\item
MPI ライブラリを用いた並列分散処理
\item
実験データベースの利用
\end{itemize}
\item
\textbf{核燃焼プラズマ統合コード構想のコアコード}
\begin{itemize}
\item
最小限の統合コード:各モジュールは交換可能
\item
インターフェースの標準化:実装の検証
\item
利用者の拡大:マニュアル等の整備
\end{itemize}
\end{itemize}

\subsection{TASK コードのモジュール構成}
\begin{center}
\begin{tabular}{r|ll}
\hline
\texttt{TASK/EQ} & {2次元平衡解析} & 固定境界,トロイダル回転効果 \\[2mm]
\texttt{TR} & {1次元輸送解析} & 拡散型輸送方程式,輸送モデル \\[2mm]
\texttt{WR} & {幾何光学的波動解析} & EC, LH: 光線追跡法,ビーム追跡
法 \\[2mm]
\texttt{WM} & {波動光学的波動解析} & IC, AW: アンテナ励起,固有モー
ド \\[2mm]
\texttt{FP} & {速度分布解析} & 相対論的,軌道平均,3次元 \\[2mm]
\texttt{DP} & {波動分散解析} & 局所誘電率テンソル,任意速度分布 \\
\texttt{PL} & {データ交換} & 座標変換,標準データ,分布データベース \\[2mm]
\texttt{LIB} & {共通ライブラリ} & 行列解法,特殊関数 \\
\texttt{MTX} & {行列解法} & 直接法/反復法,並列化 \\
\texttt{MPI} & {並列化} & 並列化ライブラリインターフェース \\
\texttt{TOT} & {一体化} & 一体化コード \\
\hline
\end{tabular}
\end{center}

\subsection{仕様}

使用言語は FORTRAN95 であり,allocatable 変数を derived type の
component とすることを可能にする拡張を含むコンパイラで,コンパイルする
ことができる.

使用するグラフィックライブラリは GSAF である.このライブラリは,

\tline{http://p-grp.nucleng.kyoto-u.ac.jp/gsaf/}

から入手することができる.このライブラリを用いて,図形を X-window に表
示し,図形データをファイルに保存し,このデータを postscript ファイルに
変換することができる.数値計算ライブラリとして LAPACK を使用する場合が
あるが,その機能が不必要であれば省略することができる.並列化ライブラリ
として MPI を使用しているが,MPI ライブラリがなくてもコンパイルできる.

\section{TASK コードのインストール}

\subsection{TASK コードの入手}

TASK コードは次のいずれかの方法で入手することができる.
\begin{enumerate}
\item
TASK と GSAF の最新ソースファイルは,CVS サーバーから以下の手順で
download することができる.この場合に必要な password は
fukuyama@nucleng.kyoto-u.ac.jp まで問い合わせること.なお,この方法では
修正したファイルを upload することはできない.

\tline{export CVS\underline{ }RSH=ssh} \\
\tline{export
  CVSROOT=:pserver:anonymous@p-grp.nucleng.kyoto-u.ac.jp:/home/fukuyama/cvs} \\
\tline{cvs login} \\
\tline{password: XXXXXXXX} \\
\tline{cvs co bpsd} \\
\tline{cvs co task} \\
\tline{cvs co gsaf}

\item
CVS サーバーに account を有する場合には,修正したファイルを upload す
ることができる.この場合の download 手順は

\tline{export CVS\underline{ }RSH=ssh} \\
\tline{export CVSROOT=p-grp.nucleng.kyoto-u.ac.jp:/home/fukuyama/cvs} \\
\tline{cvs co bpsd} \\
\tline{cvs co task} \\
\tline{cvs co gsaf}

\item
圧縮されたファイルツリー taskXXX.tar.gz ファイルは

\tline{http://bpsi.nucleng.kyoto-u.ac.jp/task/}

から入手できる.最新のファイルツリーである保証はないが,比較的安定なバー
ジョンである.得られた task.tar.gz は gnu tar の場合

\tline{tar xvzf taskXXX.tar.gz}

で,あるいは通常の tar の場合

\tline{gzcat taskXXX.tar.gz | tar xvf}

等によって解凍することができる.
\end{enumerate}

\subsection{TASK コードの更新}

CVS を利用している場合には,次のいずれかの方法で TASK コードを最新バー
ジョンに更新することができる.ただし,最新バージョンにはバグが含まれて
いる可能性もあるので,注意すること.
\begin{enumerate}
\item
pserver を利用して download した場合:task ディレクトリにおいて

\tline{export CVS\underline{ }RSH=ssh} \\
\tline{cvs login} \\
\tline{password: XXXXXXXX} \\
\tline{cvs update -Pd}

を実行する.

\item
CVS サーバーに account を有する場合:task ディレクトリにおいて

\tline{export CVS\underline{ }RSH=ssh} \\
\tline{cvs update -Pd}

を実行する.

\end{enumerate}

\subsection{コードのコンパイル}

\subsubsection{GSAF のコンパイル}

\begin{enumerate}
\item
\tline{cd gsaf/src}:ソースディレクトリに移動する.
\item
\tline{cp ../arch/XXXX-XXXX/Makefile.arch .}:必要な設定ファイルをコピー
する.
\begin{itemize}
\item
linux-g77: intel86系 Linux において g77, gcc を利用してコンパイル.
\item
linux-pgf: intel86系 Linux において PGI 製 pgf77, gcc を利用してコンパイル.
\item
linux-pgf: intel86系 Linux において Intel 製 ifort, icc を利用してコンパイル.
\item
macosx-g77: Mac OSX において g77, gcc を利用してコンパイル.
\item
macosx-xlf: Mac OSX において IBM/Absoft 製 xlf, xlc を利用してコンパイル.
\item
sx-nifs: NIFS 汎用計算機 において sxf90, sxcc を利用してコンパイル.
\item
hp-nifs: NIFS Application server において f90, cc を利用してコンパイル.
\end{itemize}

\item
Makefile.arch の中の directory BINPATH, LIBPATH を適切に設定する.
\begin{itemize}
\item
/usr/local が利用できる場合には,\ttype{/usr/local/bin}, \ttype{/usr/local/lib}
を利用
\item
/usr/local が利用できない場合には \ttype{\$HOME/bin},\ttype{\$HOME/lib} を利用
\end{itemize}

\item
\tline{make}:ライブラリを作成する.
\item
BINPATH, LIBPATH に書込権限をもつ user として,\\
\tline{make install}:コマンドを作成し,ライブラリとともにインストール
する.
\item
LIBPATH がライブラリ検索 path に含まれているようにする.
\begin{itemize}
\item
LIBPATH を \ttype{/etc/ld.so.conf} に登録し,\ttype{ldconfig}
を実行しておく.これらの操作には root 権限が必要.
\item
環境変数 \ttype{LD\underline{ }LIBRARY\underline{ }PATH} に LIBPATH を
含める.\\
\qquad 例:
\ttype{export LD\underline{ }LIBRARY\underline{ }PATH=/usr/local/lib}
\end{itemize}
\item
\tline{cd test}:テストプログラムを作成する場合には
\item
環境変数 \ttype{LD\underline{ }LIBRARY\underline{ }PATH} に \ttype{../lib} を
含める.\\
\qquad\ttype{export LD\underline{ }LIBRARY\underline{ }PATH=../lib}
\item
\tline{make}
\item
\tline{./bsctest}:基本テストプログラムを実行する.
\item
\tline{cd ../../..}
\end{enumerate}

\subsubsection{TASK のコンパイル}

まず,task ディレクトリの \ttype{make.header.org} ファイルを
\ttype{make.header} という名前でコピーし,その内容を環境に合わせて変
更する.基本的には,利用するコンパイラに関連した script 行のコメント文
字 \# を削除する.

次に利用するコードのディレクトリに移動し,

\ttype{make libs} 
\ttype{make} 

と入力すれば,コンパイルリンクが実行される.

原則的に,task ディレクトリで update した場合や他のサブディレクトリの
ファイルを修正した場合には,\ttype{make libs} が必要であり,現在のサブ
ディレクトリ内のファイルだけを修正した場合には \ttype{make} だけでよい.

\subsection{コードの各プログラムの起動}

各プログラムはグラフィックライブラリに GSAF を使用しており,プログラム
起動時(正確には \ttype{CALL GSOPEN} が実行されたとき)に,原則として
グラフィック出力設定を問い合わせる.
\begin{enumerate}
\item
最初の問い合わせは解像度の指定であり,1文字で指定する.'0' が指定され
た場合には,画面には出力されず,必要に応じてファイルに図形データが出力
さえる.
\item
2番目の問い合わせは,図形ファイルへの出力の指定であり,やはり1文字で
指定する.ファイルに出力せずに続行する場合は 'C' を指定する.表示内容
をファイルに保存する場合は,'Y','F' を指定し,出力ファイル名の問い合
わせがある.'Y' が指定された場合には常に保存,'F' が指定された場合には
ファイルは指定されたが保存は可能になっていない.最初の問い合わせへの回
答が無出力 '0' の場合には,ファイルが指定されると,特に指定しない限り,
各ページを保存する.ファイルが指定されなければ,保存はしない.最初の問
い合わせへの回答が '0' 以外の場合には,各ページ出力後に問い合わせがあ
り,単に改行の場合は標準値,' Y' は連続保存,'S' は単発保存,' N' は保
存せず,となる.

\item
グラフィック出力設定の入力は,環境変数 GSGDP を指定することで,省略す
ることができる.例えば
\tline{export GSGDP=3c}

\end{enumerate}

\subsection{コードの各プログラムの入力}
\begin{verbatim}
入力行の解釈
  1.行中に "=" を含む場合は namelist 変数の入力
  2.先頭の1文字が英字の場合はコマンドの入力
  3.それ以外は要求されている変数の入力

namelist 変数の入力
  ・すべての namelist 変数はファイル xxinit.f のサブルーチン XXINIT で
   初期化されている.
  ・実行時のディレクトリにファイル xxparm があれば,namelist 入力として読み込まれる.
  ・入力行が "=" を含む場合は,先頭に " &xx",末尾に " &end" を付加し,
   namelist 入力として読み込まれる.
  ・コマンド "P" を入力して,namelist 入力を行うことも可能.

コマンド入力
  ・先頭の1文字が英字の入力行はコマンドの入力
  ・コマンドの種類を表す英字は大文字と小文字を区別しない.

  P:namelist 変数の入力
  V:namelist 変数等の表示
  R:プログラムの実行開始
  C:プログラムの実行継続
  G:計算結果の図形表示
  S:計算結果の保存
  L:計算結果の読込
  Q:実行の終了
\end{verbatim}

\section{コンパイル・パラメータ}

XXcomn.inc の中で配列の大きさを指定するパラメータが設定されており,それ
らを変更することにより,計算パラメータ領域を拡大あるいは縮小できる.



\title{plcom1.inc}
\begin{tabbing}
\compileParmLine
\compileParm{NSM}{I}{  5}{Maximum number of particle species}
\end{tabbing}

\title{plcom2.inc}
\begin{tabbing}
\compileParmLine
\compileParm{NHM}{I}{100}{Maximum number of cyclotron harmonics}
\compileParm{NXM}{I}{1001}{Maximum number of 1D graph mesh points}
\compileParm{NPM}{I}{100}{Maximum number of momentum amplitude mesh}
\compileParm{NTHM}{I}{100}{Maximum number of pitch angle mesh}
\compileParm{NRM}{I}{25}{Maximum number of radial mesh}
\compileParm{NGXM}{I}{101}{Maximum number of graphic x mesh}
\compileParm{NGYM}{I}{101}{Maximum number of graphic y mesh}
\end{tabbing}

\title{eqcom1.inc}
\begin{tabbing}
\compileParmLine
\compileParm{NRGM}{I}{131}{Maximum number of R mesh (display)}
\compileParm{NZGM}{I}{131}{Maximum number of Z mesh (display)}
\compileParm{NPSM}{I}{131}{Maximum number of psi mesh (display)}
\end{tabbing}

\title{eqcom2.inc}
\begin{tabbing}
\compileParmLine
\compileParm{NSGM}{I}{64}{Maximum number of radial mesh (solver)}
\compileParm{NTGM}{I}{64}{Maximum number of poloidal mesh (solver)}
\compileParm{NUGM}{I}{128}{Maximum number of poloidal mesh of radial boundary (solver)}
\end{tabbing}

\title{eqcom3.inc}
\begin{tabbing}
\compileParmLine
\compileParm{NRM}{I}{201}{Maximum number of radial mesh (interpolation)}
\compileParm{NTHM}{I}{129}{Maximum number of poloidal mesh (interpolation)}
\compileParm{NSUM}{I}{1025}{Maximum number of poloidal mesh of radial boundary (interpolation)}
\compileParm{NNM}{I}{1000}{Maximum number of poloidal mesh to follow
  field line}
\end{tabbing}

\title{eqcom4.inc}
\begin{tabbing}
\compileParmLine
\compileParm{NTRM}{I}{200}{Maximum number of radial mesh (interpolation)}
\end{tabbing}

\title{trcom0.inc}
\begin{tabbing}
\compileParmLine
\compileParm{NRM}{I}{80}{Maximum number of radial mesh}
\compileParm{NTM}{I}{10001}{Maximum number of time mesh for global quantities}
\compileParm{NGM}{I}{1001}{Maximum number of time mesh for profiles}
\compileParm{NSM}{I}{4}{Maximum number of particle species (bulk ions and electrons)}
\compileParm{NSZM}{I}{2}{Maximum number of impurity species}
\compileParm{N0M}{I}{2}{Maximum number of neutral species}
\compileParm{NFM}{I}{2}{Maximum number of fast ion species}
\compileParm{NCGM}{I}{22}{Maximum number of profiles}
\compileParm{NCTM}{I}{100}{Maximum number of global quantities}
\compileParm{NTUM}{I}{1001}{Maximum number of time mesh in UFILE}
\end{tabbing}

\title{wrcom1.inc}
\begin{tabbing}
\compileParmLine
\compileParm{NEQ}{I}{8}{Number of equations in ray tracing}
\compileParm{NBEQ}{I}{19}{Number of equations in beam tracing}
\compileParm{NBVAR}{I}{53}{Number of variables in beam tracing}
\compileParm{NRAYM}{I}{9}{Maximum number of rays and beams}
\compileParm{NITM}{I}{10000}{Maximum number of iterations in tracing}
\compileParm{NRADM}{I}{1000}{Maximum number of radial division in
power deposition}
\end{tabbing}

\title{wmcomm.inc}
\begin{tabbing}
\compileParmLine
\compileParm{NRM}{I}{201}{Maximum number of radial mesh}
\compileParm{NDPM}{I}{2}{Maximum power of 2 for toroidal mode number}
\compileParm{MDPM}{I}{4}{Maximum power of 2 for poloidal mode number}
\compileParm{NAM}{I}{4}{Maximum number of antennas}
\compileParm{NSUM}{I}{2049}{Maximum number of poloidal mesh on radial boundary}
\compileParm{NGZM}{I}{401}{Maximum number of graphic mesh}
\compileParm{NTHGM}{I}{64}{Maximum number of poloidal mesh for graphics}
\end{tabbing}

\title{vmcomm.inc}
\begin{center}
Under reorganization
\end{center}

\title{fpcom1.inc}
\begin{tabbing}
\compileParmLine
\compileParm{NRM}{I}{25}{Maximum number of radial mesh}
\compileParm{NPM}{I}{50}{Maximum number of momentum amplitude mesh}
\compileParm{NTHM}{I}{50}{Maximum number of pitch angle mesh}
\compileParm{NTG1M}{I}{21}{Maximum number of detail time mesh}
\compileParm{NTG2M}{I}{501}{Maximum number of time mesh}
\compileParm{NCRM}{I}{5}{Maximum number of cyclotron harmonics}
\compileParm{NLM}{I}{13}{Maximum number of Legendre harmonics}
\end{tabbing}

\title{dpcom1.inc}
\begin{tabbing}
\compileParmLine
\compileParm{NHM}{I}{100}{Maximum number of cyclotron harmonics}
\compileParm{NXM}{I}{1001}{Maximum number of one-dimensional graphic mesh}
\compileParm{NPM}{I}{100}{Maximum number of momentum amplitude mesh}
\compileParm{NTHM}{I}{100}{Maximum number of pitch angle mesh}
\compileParm{NRM}{I}{25}{Maximum number of radial mesh}
\compileParm{NGXM}{I}{101}{Maximum number of x-mesh for two-dimensional graphics}
\compileParm{NGYM}{I}{1011}{Maximum number of y-mesh for two-dimensional graphics}
\end{tabbing}

\section{pl コード}
\begin{verbatim}
C
C     ======( PHYSICAL CONSTANTS )======
C
C        PI    : Pi
C        AEE   : Elementaty charge
C        AME   : Electron mass
C        AMP   : Proton mass
C        VC    : Speed of light in vacuum
C        RMU0  : Permeability of free space
C        EPS0  : Permittivity of free space
C        CI    : Imaginary unit
C
      PI    = 2.D0*ACOS(0.D0)
      AEE   = 1.6021892  D-19
      AME   = 9.109534   D-31
      AMP   = 1.6726485  D-27
      VC    = 2.99792458 D  8
      RMU0  = 4.D0*PI*1.D-7
      EPS0  = 1.D0/(VC*VC*RMU0)
      CI    = (0.D0,1.D0)
C
C     ======( DEVICE PARAMETERS )======
C
C        RR    : Plasma major radius                             (m)
C        RA    : Plasma minor radius                             (m)
C        RB    : Wall minor radius                               (m)
C        RKAP  : Plasma shape elongation
C        RDEL  : Plasma shape triangularity *
C        BB    : Magnetic field at center                        (T)
C        Q0    : Safety factor at center
C        QA    : Safety factor on plasma surface
C        RIP   : Plasma current                                 (MA)
C        PROFJ : Curren density profile parameter (power of (1 - rho^2))
C
      RR    = 3.D0
      RA    = 1.D0
      RB    = 1.2D0
      RKAP  = 1.D0
      RDLT  = 0.D0
C
      BB    = 3.D0
      Q0    = 1.D0
      QA    = 3.D0
      RIP   = 3.D0
      PROFJ = 2.D0
C
C     ======( PLASMA PARAMETERS )======
C
C        NSMAX : Number of particle species
C        PA    : Mass number
C        PZ    : Charge number
C        PN    : Density at center                     (1.0E20/m**3)
C        PNS   : Density on plasma surface             (1.0E20/m**3)
C        PZCL  : Ratio of collision frequency to wave frequency
C        PTPR  : Parallel temperature at center                (keV)
C        PTPP  : Perpendicular temperature at center           (keV)
C        PTS   : Temperature on surface                        (keV)
C        PU    : Toroidal rotation velocity at center          (m/s)
C        PUS   : Toroidal rotation velocity on surface         (m/s)
C        PNITB : Density increment at ITB              (1.0E20/Mm*3)
C        PTITB : Temperature increment at ITB                  (keV)
C        PUITB : Toroidal rotation velocity increment at ITB   (m/s)
C
      NSMAX = MIN(2,NSM)
C
         PA(1)   = AME/AMP
         PZ(1)   =-1.0D0
         PN(1)   = 1.0D0
         PNS(1)  = 0.0D0
         PZCL(1) = 0.00D0
         PTPR(1) = 5.0D0
         PTPP(1) = 5.0D0
         PTS(1)  = 0.05D0
         PU(1)   = 0.D0
         PUS(1)  = 0.D0
         PNITB(1)= 0.D0
         PTITB(1)= 0.D0
         PUITB(1)= 0.D0
C
      IF(NSM.GE.2) THEN
         PA(2)   = 1.0D0
         PZ(2)   = 1.0D0
         PN(2)   = 1.0D0
         PNS(2)  = 0.0D0
         PZCL(2) = 0.00D0
         PTPR(2) = 5.0D0
         PTPP(2) = 5.0D0
         PTS(2)  = 0.05D0
         PU(2)   = 0.D0
         PUS(2)  = 0.D0
         PNITB(2)= 0.D0
         PTITB(2)= 0.D0
         PUITB(2)= 0.D0
      ENDIF
C
      DO NS=3,NSM
         PA(NS)   = 1.0D0
         PZ(NS)   = 1.0D0
         PN(NS)   = 0.0D0
         PNS(NS)  = 0.0D0
         PZCL(NS) = 0.0D0
         PTPR(NS) = 5.0D0
         PTPP(NS) = 5.0D0
         PTS(NS)  = 0.0D0
         PU(NS)   = 0.D0
         PUS(NS)  = 0.D0
         PNITB(NS)= 0.D0
         PTITB(NS)= 0.D0
         PUITB(NS)= 0.D0
      ENDDO
C
C     ======( PROFILE PARAMETERS )======
C
C
C        PROFN1: Density profile parameter (power of rho)
C        PROFN2: Density profile parameter (power of (1 - rho^PROFN1))
C        PROFT1: Temperature profile parameter (power of rho)
C        PROFT2: Temperature profile parameter (power of (1 - rho^PROFN1))
C        PROFU1: Rotation profile parameter (power of rho)
C        PROFU2: Rotation profile parameter (power of (1 - rho^PROFN1))
C
      PROFN1= 2.D0
      PROFN2= 0.5D0
      PROFT1= 2.D0
      PROFT2= 1.D0
      PROFU1= 2.D0
      PROFU2= 1.D0
C
C     ======( MODEL PARAMETERS )======
C
C        MODELG: Control plasma geometry model
C                   0: Slab geometry
C                   1: Cylindrical geometry
C                   2: Toroidal geometry
C                   3: TASK/EQ output geometry
C                   4: VMEC output geometry
C                   5: EQDSK output geometry
C                   6: Boozer output geometry
C        MODELN: Control plasma profile
C                   0: Calculated from PN,PNS,PTPR,PTPP,PTS,PU,PUS
C                   1: PT calculated from TASK/EQ pressure profile
C                   2: PN*PT proportional to TASK/EQ pressure profile
C                   8: Read from file by means of WMDPRF routine (DIII-D)
C                   9: Read from file by means of WMXPRF routine (JT-60)
C        MODELQ: Control safety factor profile (for MODELG=0,1,2)
C                   0: Parabolic q profile (Q0,QA,RHOMIN,RHOITB)
C                   1: Given current profile (RIP,PROFJ)
C
      MODELG= 2
      MODELN= 0
      MODELQ= 0
C
C        RHOMIN: rho at minimum q (0 for positive shear)
C        QMIN  : q minimum for reversed shear
C        RHOITB: rho at ITB (0 for no ITB)
C        RHOEDG: rho at EDGE for smoothing (1 for no smooth)
C
      RHOMIN = 0.D0
      QMIN   = 1.5D0
      RHOITB = 0.D0
      RHOEDG = 1.D0
C
C     ======( GRAPHIC PARAMETERS )======
C
C        RHOGMN: minimum rho in radial profile
C        RHOGMX: maximum rho in radial profile
C
      RHOGMN = 0.D0
      RHOGMX = 1.D0
C
C     ======( MODEL PARAMETERS )======
C
C        KNAMEQ: Filename of equilibrium data
C        KNAMWR: Filename of ray tracing data
C        KNAMWM: Filename of full wave data
C        KNAMFP: Filename of Fokker-Planck data
C        KNAMFO: Filename of File output
C        KNAMPF: Filename of profile data
C
      KNAMEQ = 'eqdata'
      KNAMWR = 'wrdata'
      KNAMWM = 'wmdata'
      KNAMFP = 'fpdata'
      KNAMFO = 'fodata'
      KNAMPF = 'pfdata'
\end{verbatim}


\section{EQ コード}
\begin{verbatim}
C
C     *** CONSTANTS ****
C
C        PI    : Pi
C        RMU0  : Permeability of free space
C        AMP   : Proton mass
C        AEE   : Electron charge
C
      PI     = 2.D0*ASIN(1.D0)
      RMU0   = 4.D0*PI*1.D-7
      AMP    = 1.6726231D-27
      AEE    = 1.60217733D-19
C
C     *** CONFIGURATION PARAMETERS ***
C
C        RR    : Plasma major radius                             (m)
C        RA    : Plasma minor radius                             (m)
C        RB    : Wall minor radius                               (m)
C        RKAP  : Plasma shape elongation
C        RDLT  : Plasma shape triangularity 
C        BB    : Magnetic field at center                        (T)
C        RIP   : Plasma current                                 (MA)
C
      RR     = 3.D0
      RA     = 1.D0
      RB     = RA*1.1D0
      RKAP   = 1.6D0
      RDLT   = 0.25D0
      BB     = 3.D0
      RIP    = 3.D0
C
C     *** PROFILE PARAMETERS ***
C
C        PP0   : Plasma pressure (main component)              (MPa)
C        PP1   : Plasma pressure (sub component)               (MPa)
C        PP2   : Plasma pressure (increment within ITB)        (MPa)
C        PROFP0: Pressure profile parameter
C        PROFP1: Pressure profile parameter
C        PROFP2: Pressure profile parameter
C
C        PPSI=PP0*(1.D0-PSIN**PROFR0)**PROFP0
C    &       +PP1*(1.D0-PSIN**PROFR1)**PROFP1
C    &       +PP2*(1.D0-(PSIN/PSIITB)**PROFR2)**PROFP2
C
C        The third term exists for RHO < RHOITB
C
      PP0    = 0.001D0
      PP1    = 0.0D0
      PP2    = 0.0D0
      PROFP0 = 1.5D0
      PROFP1 = 1.5D0
      PROFP2 = 2.0D0
C
C        PJ0   : Current density at R=RR (main component) : Fixed to 1
C        PJ1   : Current density at R=RR (sub component)       (arb)
C        PJ2   : Current density at R=RR (sub component)       (arb)
C        PROFJ0: Current density profile parameter
C        PROFJ1: Current density profile parameter
C        PROFJ2: Current density profile parameter
C
C      HJPSI=-PJ0*(1.D0-PSIN**PROFR0)**PROFJ0
C     &                *PSIN**(PROFR0-1.D0)
C     &      -PJ1*(1.D0-PSIN**PROFR1)**PROFJ1
C     &                *PSIN**(PROFR1-1.D0)
C     &      -PJ2*(1.D0-PSIN**PROFR2)**PROFJ2
C     &                *PSIN**(PROFR2-1.D0)
C
C        The third term exists for RHO < RHOITB
C
      PJ0    = 1.00D0
      PJ1    = 0.0D0
      PJ2    = 0.0D0
      PROFJ0 = 1.5D0
      PROFJ1 = 1.5D0
      PROFJ2 = 1.5D0
C
C        FF0   : Current density at R=RR (main component) : Fixed to 1
C        FF1   : Current density at R=RR (sub component)       (arb)
C        FF2   : Current density at R=RR (sub component)       (arb)
C        PROFF0: Current density profile parameter
C        PROFF1: Current density profile parameter
C        PROFF2: Current density profile parameter
C
C      FPSI=BB*RR
C     &      +FF0*(1.D0-PSIN**PROFR0)**PROFF0
C     &      +FF1*(1.D0-PSIN**PROFR1)**PROFF1
C     &      +FF2*(1.D0-PSIN**PROFR2)**PROFF2
C
C        The third term exists for RHO < RHOITB
C
      FF0    = 1.0D0
      FF1    = 0.0D0
      FF2    = 0.0D0
      PROFF0 = 1.5D0
      PROFF1 = 1.5D0
      PROFF2 = 1.5D0
C
C        QQ0   : Safety factor on axis for QQ1=QQ2=0
C        QQS   : Safety factor on surface
C        QQ0   : Safety factor 
C        QQ1   : Safety factor (sub component)
C        QQ2   : Safety factor (increment within ITB)
C        PROFQ0: Safety factor profile parameter
C        PROFQ1: Safety factor profile parameter
C        PROFP2: Pressure profile parameter
C
C        QPSI=QQS
C    &       +(QQ0-QQS)*(1.D0-PSIN**PROFR0)**PROFQ0
C    &       +QQ1*(1.D0-PSIN**PROFR1)**PROFQ1
C    &       +QQ2*(1.D0-(PSIN/PSIITB)**PROFR2)**PROFQ2
C
C        The third term exists for RHO < RHOITB
C
      QQ0    = 1.D0
      QQS    = 3.D0
      QQ1    = 0.0D0
      QQ2    = 0.0D0
      PROFQ0 = 1.0D0
      PROFQ1 = 1.0D0
      PROFQ2 = 1.0D0
C
C        PT0   : Plasma temperature (main component)           (keV)
C        PT1   : Plasma temperature (sub component)            (keV)
C        PT2   : Plasma temperature (increment within ITB)     (keV)
C        PTS   : Plasma temperature (at surface)               (keV)
C        PROFT0: Temperature profile parameter
C        PROFT1: Temperature profile parameter
C        PROFT2: Temperature profile parameter
C
C        TPSI=PTS+(PT0-PTS)*(1.D0-PSIN**PROFR0)**PROFT0
C    &       +PT1*(1.D0-PSIN**PROFR1)**PROFT1
C    &       +PT2*(1.D0-PSIN/PSIITB)**PROFR2)**PROFT2
C    &       +PTS
C
C        The third term exits for RHO < RHOITB
C
      PT0    = 1.0D0
      PT1    = 0.0D0
      PT2    = 0.0D0
      PTS    = 0.05D0
      PROFT0 = 1.5D0
      PROFT1 = 1.5D0
      PROFT2 = 2.0D0
C
C        PV0   : Toroidal rotation (main component)              (m/s)
C        PV1   : Toroidal rotation (sub component)               (m/s)
C        PV2   : Toroidal rotation (increment within ITB)        (m/s)
C        PROFV0: Velocity profile parameter
C        PROFV1: Velocity profile parameter
C        PROFV2: Velocity profile parameter
C
C        PVSI=PV0*(1.D0-PSIN**PROFR0)**PROFV0
C    &       +PV1*(1.D0-PSIN**PROFR1)**PROFV1
C    &       +PV2*(1.D0-(PSIN/PSIITB)**PROFR2)**PROFV2
C
C        The third term exits for RHO < RHOITB
C
      PV0    = 0.0D0
      PV1    = 0.0D0
      PV2    = 0.0D0
      PROFV0 = 1.5D0
      PROFV1 = 1.5D0
      PROFV2 = 2.0D0
C
C        PN0 : Plasma number density(constant)
C
      PN0    = 1.D20
C
C        PROFR0: Profile parameter
C        PROFR1: Profile parameter
C        PROFR2: Profile parameter
C        RHOITB: Normalized radius SQRT(PSI/PSIA) at ITB
C
      PROFR0 = 1.D0
      PROFR1 = 2.D0
      PROFR2 = 2.D0
      RHOITB = 0.5D0
C
C        OTC   : Constant OMEGA**2/TPSI
C        HM    : Constant                                       (Am)
C
      OTC = 0.15D0
      HM  = 1.D6
C
C     *** MESH PARAMETERS ***
C
C        NSGMAX: Number of radial mesh points for Grad-Shafranov eq.
C        NTGMAX: Number of poloidal mesh points for Grad-Shafranov eq.
C        NUGMAX: Number of radial mesh points for flux-average quantities
C        NRGMAX: Number of horizontal mesh points in R-Z plane
C        NZGMAX: Number of vertical mesh points in R-Z plane
C        NPSMAX: Number of flux surfaces
C        NRMAX : Number of radial mesh points for flux coordinates
C        NTHMAX: Number of poloidal mesh points for flux coordinates
C        NSUMAX: Number of boundary points
C
      NSGMAX = 32
      NTGMAX = 32
      NUGMAX = 32
C
      NRGMAX = 32
      NZGMAX = 32
      NPSMAX = 21
C
      NRMAX  = 50
      NTHMAX = 64
      NSUMAX = 65
C
C     *** CONTROL PARAMETERS ***
C
C        EPSEQ : Convergence criterion for equilibrium
C
      EPSEQ  = 1.D-6
C
C        MDLEQF : Profile parameter
C            0: given analytic profile  P,Jtoroidal,T,Vph
C            1: given analytic profile  P,F,T,Vph
C            2: given analytic profile  P,Jparallel,T,Vph
C            3: given analytic profile  P,q,T,Vph
C            5: given spline profile  P,Jtoroial,T,Vph
C            6: given spline profile  P,F,T,Vph
C            7: given spline profile  P,Jparapllel,T,Vph
C            8: given spline profile  P,q,T,Vph
C
      MDLEQF = 0
C
C        MDLEQC : Poloidal coordinate parameter
C            0: Poloidal length coordinate
C            1: Boozer coordinate
C
      MDLEQC = 0
C
C        NPRINT: Level print out
C
      NPRINT= 0
C
C     *** FILE NAME ***
C
C        KNAMEQ: Filename of equilibrium data
C
      KNAMEQ = 'eqdata'
\end{verbatim}

\section{WR コード}
\begin{verbatim}
C
C     *** CONSTANTS ****
C
C        PI    : Pi
C        AEE   : Elementaty charge
C        AME   : Electron mass
C        AMM   : Proton mass
C        VC    : Speed of light in vacuum
C        RMU0  : Permeability of free space
C        EPS0  : Permittivity of free space
C        VOID  : 0.D0
C
      PI      = ASIN(1.D0)*2.D0
      AEE     = 1.60217733D-19
      AME     = 9.1093897D-31
      AMM     = 1.6726231D-27
      VC      = 2.99792458D8
      RMU0    = 4.D0*PI*1.D-7
      EPS0    = 1.D0/(VC*VC*RMU0)
      RKEV    = AEE*1.D3
      VOID    = 0.D0
C
C     ==== DEVICE PARAMETERS ====
C
C        RR     : PLASMA MAJOR RADIUS (M)
C        RA     : PLASMA MINOR RADIUS (M)
C        RKAP   : ELIPTICITY OF POLOIDAL CROSS SECTION
C        RDLT   : TRIANGULARITY OF POLOIDAL CROSS SECTION
C        BB     : TOROIDAL MAGNETIC FIELD ON PLASMA AXIS (T)
C        RIPS   : INITIAL VALUE OF PLASMA CURRENT (MA)
C        RIPE   : FINAL VALUE OF PLASMA CURRENT (MA)
C        RIPSS  : VALUE OF PLASMA CURRENT FOR INITIAL CONVERGENCE
C        RHOA   : EDGE OF CALCULATE REGION (NORMALIZED SMALL RADIUS)
C
      RR      = 3.0D0
      RA      = 1.2D0
      RKAP    = 1.5D0
      RDLT    = 0.0D0
      BB      = 3.D0
      RIPS    = 3.D0
      RIPE    = 3.D0
      RIPSS   = 3.D0
      RHOA    = 1.D0
C
C     ==== PLASMA PARAMETERS ====
C
C        NSMAX  : NUMBER OF MAIN PARTICLE SPECIES (NS=1:ELECTRON)
C        NSZMAX : NUMBER OF IMPURITIES SPECIES
C        NSNMAX : NUMBER OF NEUTRAL SPECIES
C
C        PA(NS) : ATOMIC NUMBER
C        PZ(NS) : CHARGE NUMBER
C        PN(NS) : INITIAL NUMBER DENSITY ON AXIS (1.E20 M**-3)
C        PNS(NS): INITIAL NUMBER DENSITY ON SURFACE (1.E20 M**-3)
C        PT(NS) : INITIAL TEMPERATURE ON AXIS (KEV)
C        PTS(IS): INITIAL TEMPERATURE ON SURFACE (KEV)
C
      NSMAX=2
      NSZMAX=0  ! the number of impurities
      NSNMAX=2  ! the number of neutrals, 0 or 2 fixed
C
      PA(1)   = AME/AMM
      PZ(1)   =-1.D0
      PN(1)   = 0.5D0
      PT(1)   = 1.5D0
      PTS(1)  = 0.05D0
      PNS(1)  = 0.05D0
C
      PA(2)   = 2.D0
      PZ(2)   = 1.D0
      PN(2)   = 0.5D0-2.D-7
      PT(2)   = 1.5D0
      PTS(2)  = 0.05D0
      PNS(2)  = 0.05D0-2.D-8
C
      PA(3)   = 3.D0
      PZ(3)   = 1.D0
      PN(3)   = 1.D-7
      PT(3)   = 1.5D0
      PTS(3)  = 0.05D0
      PNS(3)  = 1.D-8
C
      PA(4)   = 4.D0
      PZ(4)   = 2.D0
      PN(4)   = 1.D-7
      PT(4)   = 1.5D0
      PTS(4)  = 0.05D0
      PNS(4)  = 1.D-8
C
      PA(5)   = 12.D0
      PZ(5)   = 2.D0
      PN(5)   = VOID
      PT(5)   = 0.D0
      PTS(5)  = 0.D0
      PNS(5)  = VOID
C
      PA(6)   = 12.D0
      PZ(6)   = 4.D0
      PN(6)   = VOID
      PT(6)   = 0.D0
      PTS(6)  = 0.D0
      PNS(6)  = VOID
C
      PA(7)   = 2.D0
      PZ(7)   = 0.D0
      PN(7)   = 1.D-15
      PT(7)   = 0.D0
      PTS(7)  = 0.D0
      PNS(7)  = 2.D-4
C
      PA(8)   = 2.D0
      PZ(8)   = 0.D0
      PN(8)   = 1.D-15
      PT(8)   = 0.D0
      PTS(8)  = 0.D0
      PNS(8)  = 1.D-15
C
C     ==== IMPURITY PARAMETERS ====
C
C        PNC    : CARBON DENSITY FACTOR
C        PNFE   : IRON DENSITY FACTOR
C                      COMPARED WITH ITER PHYSICS DESIGN GUIDELINE
C        PNNU   : NEUTRAL NUMBER DENSITY ON AXIS (1.E20 M**-3)
C        PNNUS  :                        ON SURFACE (1.E20 M**-3)
C
      PNC     = 0.D0
      PNFE    = 0.D0
      PNNU    = 0.D0
      PNNUS   = 0.D0
C
C     ==== PROFILE PARAMETERS ====
C
C        PROFN*: PROFILE PARAMETER OF INITIAL DENSITY
C        PROFT*: PROFILE PARAMETER OF INITIAL TEMPERATURE
C        PROFU*: PROFILE PARAMETER OF INITIAL NEUTRAL DENSITY
C        PROFJ*: PROFILE PARAMETER OF INITIAL CURRENT DENSITY
C                    (X0-XS)(1-RHO**PROFX1)**PROFX2+XS
C
C        ALP   : ADDITIONAL PARAMETERS
C           ALP(1): RADIUS REDUCTION FACTOR
C           ALP(2): MASS WEIGHTING FACTOR FOR NC
C           ALP(3): CHARGE WEIGHTING FACTOR FOR NC
C
      PROFN1 = 2.D0
      PROFN2 = 0.5D0
      PROFT1 = 2.D0
      PROFT2 = 1.D0
      PROFU1 =12.D0
      PROFU2 = 1.D0
      PROFJ1 =-2.D0
      PROFJ2 = 1.D0
C
      ALP(1) = 1.0D0
      ALP(2) = 0.D0
      ALP(3) = 0.D0
C
C     ==== TRANSPORT PARAMETERS ====
C
C        AV0    : INWARD PARTICLE PINCH FACTOR
C        AD0    : PARTICLE DIFFUSION FACTOR
C        CNC    : COEFFICIENT FOR NEOCLASICAL DIFFUSION
C        CDW(8) : COEFFICIENTS FOR DW MODEL
C
      AV0    = 0.5D0
      AD0    = 0.5D0
C
      CNC    = 1.D0
      CDW(1) = 0.04D0
      CDW(2) = 0.04D0
      CDW(3) = 0.04D0
      CDW(4) = 0.04D0
      CDW(5) = 0.04D0
      CDW(6) = 0.04D0
      CDW(7) = 0.04D0
      CDW(8) = 0.04D0
C
C     ==== TRANSPORT MODEL ====
C
C        MDLKAI: TURBULENT TRANSPORT MODEL
C
C        *****  0.GE.MDLKAI.LT.10 : CONSTANT COEFFICIENT MODEL *****
C        ***** 10.GE.MDLKAI.LT.20 : DRIFT WAVE (+ITG +ETG) MODEL *****
C        ***** 20.GE.MDLKAI.LT.30 : REBU-LALLA MODEL *****
C        ***** 30.GE.MDLKAI.LT.40 : CURRENT-DIFFUSIVITY DRIVEN MODEL *****
C        ***** 40.GE.MDLKAI.LT.60 : DRIFT WAVE BALLOONING MODEL *****
C        *****       MDLKAI.GE.60 : ITG(/TEM, ETG) MODEL ETC *****
C
C           *****  MDLKAI.EQ. 0   : CONSTANT *****
C           *****  MDLKAI.EQ. 1   : CONSTANT/(1-A*rho^2) *****
C           *****  MDLKAI.EQ. 2   : CONSTANT*(dTi/drho)^B/(1-A*rho^2) *****
C           *****  MDLKAI.EQ. 3   : CONSTANT*(dTi/drho)^B*Ti^C *****
C                                                                  
C           *****  MDLKAI.EQ. 10  : etac=1 *****
C           *****  MDLKAI.EQ. 11  : etac=1 1/(1+exp) *****
C           *****  MDLKAI.EQ. 12  : etac=1 1/(1+exp) *q *****
C           *****  MDLKAI.EQ. 13  : etac=1 1/(1+exp) *(1+q^2) *****
C           *****  MDLKAI.EQ. 14  : etac=1+2.5*(Ln/RR-0.2) 1/(1+exp) *****
C           *****  MDLKAI.EQ. 15  : etac=1 1/(1+exp) func(q,eps,Ln) *****
C                                                                  
C           *****  MDLKAI.EQ. 20  : Rebu-Lalla model *****
C                                                                  
C           *****  MDLKAI.EQ. 30  : CDBM 1/(1+s) *****
C           *****  MDLKAI.EQ. 31  : CDBM F(s,alpha,kappaq) *****
C           *****  MDLKAI.EQ. 32  : CDBM F(s,alpha,kappaq)/(1+WE1^2) *****
C           *****  MDLKAI.EQ. 33  : CDBM F(s,0,kappaq) *****
C           *****  MDLKAI.EQ. 34  : CDBM F(s,0,kappaq)/(1+WE1^2) *****
C           *****  MDLKAI.EQ. 35  : CDBM (s-alpha)^2/(1+s^2.5) *****
C           *****  MDLKAI.EQ. 36  : CDBM (s-alpha)^2/(1+s^2.5)/(1+WE1^2) *****
C           *****  MDLKAI.EQ. 37  : CDBM s^2/(1+s^2.5) *****
C           *****  MDLKAI.EQ. 38  : CDBM s^2/(1+s^2.5)/(1+WE1^2) *****
C           *****  MDLKAI.EQ. 39  : CDBM F2(s,alpha,kappaq,a/R) *****
C           *****  MDLKAI.EQ. 40  : CDBM F3(s,alpha,kappaq,a/R)/(1+WS1^2) *****
C
C           *****  MDLKAI.EQ. 60  : GLF23 model *****
C           *****  MDLKAI.EQ. 61  : GLF23 (stability enhanced version) *****
C           *****  MDLKAI.EQ. 62  : IFS/PPPL model *****
C           *****  MDLKAI.EQ. 63  : Weiland model *****
C           *****  MDLKAI.EQ. 64  : Bohm/Gyro-Bohm model *****
C
C        MDLETA: RESISTIVITY MODEL
C                   0: CLASSICAL 
C                   1: NEOCLASSICAL
C        MDLAD : PARTICLE DIFFUSION MODEL
C                   0: NO PARTICL TRANSPORT 
C                   1: CONSTANT D
C        MDLAVK: HEAT PINCH MODEL
C                   0: NO HEAT PINCH
C        MDLJBS: BOOTSTRAP CURRENT MODEL
C        MDLKNS: NEOCLASSICAL TRANSPORT MODEL
C
      MDLKAI = 31
      MDLETA = 3
      MDLAD  = 3
      MDLAVK = 3
      MDLJBS = 5
      MDLKNC = 1
C
C        MDLWLD : Weiland model mode selector
C            0    : using effective transport coefficients
C            else : using transport coefficients' vectors
C
      MDLWLD=0
C
C        MDDW : mode selector for anomalous particle transport coefficient.
C            you must NOT modify this parameter.
C            0    : if MDDW=0 from start to finish when you choose
C                   a certain transport model (MDLKAI),
C                   you could control a ratio of anomalous particle
C                   transport to total particle transport to manipulate
C                   the factor of AD0.
C            else : this is because you chose MDLKAI=60, 61, or 63
C                   which assign the transport models that can calculate
C                   an anomalous particle transport coefficient
C                   on their own.
      MDDW=0
C
C     ==== Semi-Empirical Parameter for Anomalous Transport ====
C
      CHP    = 0.D0
      CK0    = 12.D0 ! for electron
      CK1    = 12.D0 ! for ions
      CWEB   = 0.D0  ! for omega ExB
      CALF   = 1.D0  ! for s-alpha
      CKALFA = 0.D0
      CKBETA = 0.D0
      CKGUMA = 0.D0
C
C     ==== CONTROL PARAMETERS ====
C
C        DT     : SIZE OF TIME STEP
C        NRMAX  : NUMBER OF RADIAL MESH POINTS
C        NTMAX  : NUMBER OF TIME STEP
C        NTSTEP : INTERVAL OF SNAP DATA PRINT
C        NGRSTP : INTERVAL OF RADIAL PROFILE SAVE
C        NGTSTP : INTERVAL OF TIME EVOLUTION SAVE
C        NGPST  : ???
C        TSST   : ???
C
      DT     = 0.01D0 
      NRMAX  = 50
      NTMAX  = 100
      NTSTEP = 10
      NGRSTP = 100
      NGTSTP = 2
      NGPST  = 4
      TSST   = 1.D9
C
C     ==== Convergence Parameter ====
C
C        EPSLTR : CONVERGENCE CRITERION OF ITERATION
C        LMAXTR : MAXIMUM COUNT OF ITERATION
C
      EPSLTR = 0.001D0
C      EPSLTR = 1.D99
      LMAXTR = 10
C
C     ==== SAWTOOTH PARAMETERS ====
C
C        TPRST  : SAWTOOTH PERIOD (S)
C        MDLST  : SAWTOOTH MODEL TYPE
C                    0:OFF
C                    1:ON
C        IZERO  : SAWTOOTH CRASH TYPE
C
      TPRST  = 0.1D0
      MDLST  = 0
      IZERO  = 3
C
C     ==== FUSION REACTION PARAMETERS ====
C
C        MDLNF  : FUSION REACTION MODEL TYPE
C                    0:OFF
C                    1:ON
C
      MDLNF  = 0
C
C     ==== NBI HEATING PARAMETERS ====
C        PNBTOT : NBI TOTAL INPUT POWER (MW)
C        PNBR0  : RADIAL POSITION OF NBI POWER DEPOSITION (M)
C        PNBRW  : RADIAL WIDTH OF NBI POWER DEPOSITION (M)
C        PNBENG : NBI BEAM ENERGY (keV)
C        PNBRTG : TANGENTIAL RADIUS OF NBI BEAM (M)
C        PNBCD  : CURRENT DRIVE FACTOR
C        MDLNB  : NBI MODEL TYPE
C                    0:OFF
C                    1:GAUSSIAN
C                    2:PENCIL BEAM
C
      PNBTOT = 0.D0
      PNBR0  = 0.D0
      PNBRW  = 0.5D0
      PNBENG = 80.D0
      PNBRTG = 3.D0
      PNBCD  = 1.D0
      MDLNB  = 1
C
C     ==== ECRF PARAMETERS ====
C
C        PECTOT : ECRF INPUT POWER (MW)
C        PECR0  : RADIAL POSITION OF POWER DEPOSITION (M)
C        PECRW  : RADIAL WIDTH OF POWER DEPOSITION (M)
C        PECTOE : POWER PARTITION TO ELECTRON
C        PECNPR : PARALLEL REFRACTIVE INDEX
C        PECCD  : CURRENT DRIVE FACTOR
C        MDLEC  : ECRF MODEL
C
      PECTOT = 0.D0
      PECR0  = 0.D0
      PECRW  = 0.2D0
      PECTOE = 1.D0
      PECNPR = 0.D0
      PECCD  = 0.D0
      MDLEC  = 0
C
C     ==== LHRF PARAMETERS ====
C
C        PLHTOT : LHRF INPUT POWER (MW)
C        PLHR0  : RADIAL POSITION OF POWER DEPOSITION (M)
C        PLHRW  : RADIAL WIDTH OF POWER DEPOSITION (M)
C        PLHTOE : POWER PARTITION TO ELECTRON
C        PLHNPR : PARALLEL REFRACTIVE INDEX
C        PLHCD  : CURRENT DRIVE FACTOR
C        MDLLH  : LHRF MODEL
C
      PLHTOT = 0.D0
      PLHR0  = 0.D0
      PLHRW  = 0.2D0
      PLHTOE = 1.D0
      PLHNPR = 2.D0
      MDLLH  = 0
C
C     ==== ICRF PARAMETERS ====
C
C        PICTOT : ICRF INPUT POWER (MW)
C        PICR0  : RADIAL POSITION OF POWER DEPOSITION (M)
C        PICRW  : RADIAL WIDTH OF POWER DEPOSITION (M)
C        PICTOE : POWER PARTITION TO ELECTRON
C        PICNPR : PARALLEL REFRACTIVE INDEX
C        PICCD  : CURRENT DRIVE FACTOR
C        MDLIC  : ICRF MODEL
C
      PICTOT = 0.D0
      PICR0  = 0.D0
      PICRW  = 0.5D0
      PICTOE = 0.5D0
      PICNPR = 2.D0
      PICCD  = 0.D0
      MDLIC  = 0
C
C     ==== CURRENT DRIVE PARAMETERS ====
C
C        PBSCD : BOOTSTRAP CURRENT DRIVE FACTOR
C        MDLCD : CURRENT DRIVE OPERATION MODEL
C                  0: TOTAL PLASMA CURRENT FIXED
C                  1: TOTAL PLASMA CURRENT VARIABLE
C
      PBSCD  = 1.D0
      MDLCD  = 0
C
C     ==== PELLET INJECTION PARAMETERS ====
C
C        MDLPEL : PELLET INJECTION MODEL TYPE
C                    0:OFF  1:GAUSSIAN  2:NAKAMURA  3:HO
C        PELTOT : TOTAL NUMBER OF PARTICLES IN PELLET
C        PELR0  : RADIAL POSITION OF PELLET DEPOSITION (M)
C        PELRW  : RADIAL WIDTH OF PELLET DEPOSITION (M)
C        PELRAD : RADIUS OF PELLET (M)
C        PELVEL : PELLET INJECTION VELOCITY (M/S)
C        PELTIM : TIME FOR PELLET TO BE INJECTED
C        PELPAT : PARTICLE RATIO IN PELLET'
C
      MDLPEL = 1
      PELTOT = 0.D0
      PELR0  = 0.D0
      PELRW  = 0.5D0
      PELRAD = 0.D0
      PELVEL = 0.D0
      PELTIM = -10.D0
C
      DO NS=1,NSMAX
         PELPAT(NS) = 1.0D0
      ENDDO
C
C     ==== DEVICE NAME AND SHOT NUMBER IN UFILE DATA ====
C        KUFDEV : DEVICE NAME
C        KUFDCG : DISCHARGE NUMBER
C
      KUFDEV='jet'
      KUFDCG='19649'
C
C     ==== LOG FILE NAME ====
C        KFNLOG : LOG FILE NAME
C
      KFNLOG='trf.log'
C
C     ==== INTERACTION WITH EQ ====
C
C        MODELG: 0 : TR ONLY
C                3 : USING GEOMETRIC FACTORS FROM EQ FOR INITIAL PROFILE
C        MODELQ: 0 : TR ONLY
C                3 : TR/EQ COUPLED
C        NTEQIT: STEP INTERVAL OF EQ CALCULATION
C
      MODELG=0
      MODELQ=0
      NTEQIT=10
C
C     ==== INPUT FROM UFILE ====
C
C        MDLUF :
C           0 : not used
C           1 : time evolution
C           2 : steady state
C           3 : compared with TOPICS
C
      MDLUF=0
C
C     ==== IMPURITY TREATMENT ====
C
C        MDNI  :
C           0 : NSMAX=2, ne=ni
C           1 : calculate nimp and zeff profiles from NE, ZIMP and NM1
C           2 : calculate nimp and ni profiles from NE, ZIMP and ZEFFR
C           3 : calculate zeff and ni profiles from NE, ZIMP and NIMP
C
      MDNI=0
C
C     ==== INITIAL PROFILE SWITCH ====
C
C        MODEP : initial profile selector for steady-state simulation
C
      MODEP=3
C
C     ==== INITIAL CURRENT PROFILE SWITCH ====
C
C        MDLJQ : 
C
C           0 : create AJ(NR) profile from experimental Q profile
C           1 : create QP(NR) profile from experimental CURTOT profile
C
      MDLJQ=0
C
C     *** Eqs. Selection Parameter ***
C
      MDLEQB=1  ! 0/1 for B_theta
      MDLEQN=0  ! 0/1 for density
      MDLEQT=1  ! 0/1 for heat
      MDLEQU=0  ! 0/1 for rotation
      MDLEQZ=0  ! 0/1 for impurity
      MDLEQ0=0  ! 0/1 for neutral
      MDLEQE=0  ! 0/1 for electron density
C
      MDLEOI=0  ! 0/1/2 for electron only or bulk ion only if NSMAX=1
C
C     *** NCLASS SWITCH ***
C        0    : off
C        else : on
      MDNCLS=0
C
C     *** MODERATE TIME EVOLUTION FOR ANOMALOUS TRANSPORT COEFFICIENTS ***
C        0    : off
C        else : multiplyer for TAUK (which is the required time of
C               averaging magnetic surface)
      MDTC=0
\end{verbatim}


\section{WR コード}

WR コードは光線追跡法あるいはビーム追跡法
による伝播解析,分散関係表示などができる.

\subsection{WR コードの実行}
\begin{itemize}
 \item[] {\tt P} : パラメータを変更する.
 \item[] {\tt V} : パラメータを表示する.
 \item[] {\tt R} : 光線追跡法を実行する.
 \item[] {\tt B} : ビーム追跡法を実行する.
 \item[] {\tt G} : 伝播解析結果のグラフを表示する.
 \item[] {\tt S} : 伝播解析結果のデータを保存する.
 \item[] {\tt 1} : 分散関係を表示する.
 \item[] {\tt 2} : 分散関係を表示する.
 \item[] {\tt 3} : 分散関係を表示する.
 \item[] {\tt F} : 分散式の解を求める.
 \item[] {\tt Q} : 終了する.
\end{itemize}

一連の作業の流れとしては{\tt P}でパラメータを変更した後,{\tt R}で光線
追跡を実行した後,{\tt G}でグラフを見たり,{\tt S}で伝播解析の結果を保
存したりする.

\subsection{{\tt P,V} :  入力パラメータ}

入力パラメータは,namelist を用いて,任意のパラメータを変更することが
できる.入力行は,まず空白1文字の後,'\&wr ' に引き続いて,「パラメー
タ名=新しい値」の形式で設定を繰り返し,最後に,'\&end ' を入力して終
了する.入力例は
\tline{\ \&wr\ RR=8.14,RA=2.8,BB=5.68\ \&end}
\tline{\ \&wr\ PN=0.2,0.1,0.1\ \&end}
\\
ここで \ttype{PN=} は \ttype{PN(1),PN(2),PN(3)=}に対応する.

以下に入力パラメータの説明と標準値を示す.

   \begin{itemize}
   \item[]
   \begin{namelist}{RKX1,RKX2xx}
      \vspace{-2mm}
      \item[ RR   ] プラズマ主半径 : 3.0 [m]
      \item[ RA   ] プラズマ小半径 : 1.0 [m]
      \item[ RB   ] 壁小半径 : 1.2 [m]
      \item[ RKAP ] 楕円率 : 1.0
      \item[ RDLT ] 三角形度 : 0.0
      \item[ BB   ] 中心での磁場 : 3.0 [T]
      \item[ Q0   ] r=0 での安全係数 : 1.0
      \item[ QA   ] r=RA での安全係数 : 3.0
      \item[ RIP  ] 全電流 : 3.0 [MA]
      \item[ PROFJ] 電流分布パラメータ : 2.0
      \vspace{3mm}  

      \item[ PROFN1] 密度分布形状パラメータ : 2.0D0
      \item[ PROFN2] 密度分布形状パラメータ : 0.5D0
      \item[ PROFT1] 温度分布形状パラメータ : 2.0D0
      \item[ PROFT2] 温度分布形状パラメータ : 1.0D0
      \item[ PROFU1] 平行速度分布形状パラメータ : 2.0D0
      \item[ PROFU2] 平行速度分布形状パラメータ : 1.0D0
      \vspace{3mm}  

      \item[ NSMAX] 粒子種の数 : 2 (1 : 電子)
      \item[ PA   ] 原子の質量 [陽子質量]: PA(1)=5.4462e-4, PA(2)=1.0
      \item[ PZ   ] 電荷の数 [素電荷]: PZ(1)=-1.0, PZ(2)=1.0
      \item[ PN   ] 中心密度 [$10^{20}/\mr{m}^3$] : PN(1)=1.0, PN(2)=1.0
      \item[ PNS  ] 周辺密度 [$10^{20}/\mr{m}^3$] : PNS(1)=0.0, PNS(2)=0.0
      \item[ PZCL ] 衝突周波数 [$\nu/\omega$] : PZCL(1)=0.0, PZCL(2)=0.0
      \item[ PTPR ] 中心平行方向温度 [keV] : PTPR(1)=5.0, PTPR(2)=5.0
      \item[ PTPP ] 中心垂直方向温度 [keV] : PTPP(1)=5.0, PTPP(2)=5.0
      \item[ PTS  ] 周辺温度 [keV] : PTS(1)=0.05, PTS(2)=0.05
      \item[ PU   ] 中心平行方向速度 [m/s] : PU(1)=0.0, PU(2)=0.0
      \item[ PUS  ] 周辺平行方向速度 [m/s] : PUS(1)=0.0, PUS(2)=0.0
      \item[ PNITB] ITB での密度増分 [$10^{20}/\mr{m}^3$] : PNITB(1)=0.0
      \item[ PTITB] ITB での温度増分 [keV] : PTITB(1)=0.0
      \item[ PUITB] ITB での速度増分 [m/s] : PUITB(1)=0.0
      \vspace{3mm}  

      \item[ MODELG] 配位モデル
\begin{eqnarray*}
0&:&平板モデル \\
1&:&円柱モデル \\
2&:&トカマクモデル \\
3&:&TASK/EQ 平衡配位 \\
4&:&VMEC 平衡配位\\
\end{eqnarray*}

      \item[ MODELN] 径方向分布モデル
\begin{eqnarray*}
0&:&PN,PNS,PT,PTS 等で指定 \\
1&:&PN は指定,PT は平衡圧力から計算 \\
2&:&PN*PT が平衡圧力に比例 \\
9&:&分布データ読み込み \\
\end{eqnarray*}

      \item[ MODELQ] 安全係数分布モデル (MODELG=0,1,2)
\begin{eqnarray*}
0&:&QO, QA を指定 \\
1&:&RIP, PROFJ を指定 \\
\end{eqnarray*}

      \item[ RHOMIN] 安全係数が極小となる規格化半径: 0.D0
      \item[ QMIN]   極小安全係数
      \item[ RHOITB] ITB を与える規格化半径: 0.D0
      \item[ RHOEDG] プラズマ表面での分布の平滑化を与える規格化半径: 1.D0
      \vspace{3mm}  

      \item[RHOGMN] 径方向分布グラフの規格化半径の下限: 0.00
      \item[RHOGMX] 径方向分布グラフの規格化半径の上限: 1.D0
      \vspace{3mm}  

      \item[ KNAMEQ] 平衡データファイル名: eqdata
      \item[ KNAMWR] 波動伝播データファイル名: wrdata
      \item[ KNAMFP] 速度分布データファイル名: fpdata
      \item[ KNAMFO] 数値データファイル名: fodata
      \vspace{3mm}  

      \item[ MODELP] 誘電率テンソル : MODELP(1)=5,MODELP(2)=0
\begin{eqnarray*}
0&:&無衝突の冷たいプラズマ\\
1&:&衝突のある冷たいプラズマ\\
2&:&理想電磁流体プラズマ\\
3&:&抵抗性電磁流体プラズマ\\
4&:&有限ラーモア半径効果を無視した運動論的プラズマ\\
5&:&有限ラーモア半径効果を取り入れた運動論的プラズマ\\
6&:&相対論効果を取り入れた運動論的プラズマ\\
7&:&速度分布を与えた運動論的プラズマ\\
8&:&ジャイロ運動論的プラズマ\\
9&:&速度分布を与えたジャイロ運動論的プラズマ\\
0-9&:&伝播=与えられたモデル\\
& &偏波=与えられたモデル\\
& &吸収=与えられたモデル\\
10-19&:&伝播=冷たいプラズマモデル\\
& &偏波=与えられたモデル\\
& &吸収=与えられたモデル\\
20-29&:&伝播=冷たいプラズマモデル\\
& &偏波=冷たいプラズマモデル\\
& &吸収=与えられたモデル\\
\end{eqnarray*}
      \item[NDISP1] 最小サイクロトロン高調波番号:NDISP1(1)=-2, NDISP1(2)=-2
      \item[NDISP2] 最大サイクロトロン高調波番号:NDISP2(1)=2, NDISP2(2)=2
      \item[MODELV] 速度分布モデル
\begin{eqnarray*}
0&:&非相対論的マクスウェル速度分布\\
1&:&非相対論的任意速度分布(ファイルから読み込み)\\
2&:&相対論的マクスウェル速度分布\\
3&:&相対論的任意速度分布(ファイルから読み込み)\\
\end{eqnarray*}
      \vspace{3mm}

      \item[RF    ] 周波数 [MHz]
      \item[RPI   ] 初期大半径位置 [m]
      \item[ZPI   ] 初期垂直位置 [m]
      \item[PHII  ] 初期トロイダル角 [Rad]
      \item[RNZI  ] 初期垂直方向屈折率
      \item[RNPHII] 初期トロイダル方向屈折率
      \item[RKR0  ] 径方向波数の初期推定値(Newton法の初期値)
      \item[UUI   ] 規格化パワー初期値
      \vspace{3mm}

      \item[SMAX  ] 光線長の最大値 [m] : 5.00
      \item[DELS  ] 光線の刻み幅 [m] : 1.00e-2
      \item[UUMIN ] 光線を追跡する最小パワー : 1.00e-4
      \item[NRAYMX] 光線本数
      \vspace{3mm}

      \item[EPSRAY] 常微分方程式の収束判定条件
      \item[DELRAY] 常微分方程式のステップ幅の下限
      \item[DELDER] 数値微分のステップ幅
      \item[DELKR ] ニュートン法における数値微分のステップ幅
      \item[EPSNW ] ニュートン法における収束判定条件
      \item[LMAXNW] ニュートン法における反復回数の上限
      \vspace{3mm}

      \item[INTYPE] 計算出発パラメータの入力形式
\begin{eqnarray*}
0&:& RF,RP,ZP,PHI,RKR0,RNZ,RNPHI,UU \\
1&:& RF,RP,ZP,PHI,RKR0,ANGZ,ANGPH,UU \\
2&:& RF,RP,ZP,PHI,MODE,ANGZ,ANGPH,UU
\end{eqnarray*}
      \item[IGTYPE] グラフの表示形式
\begin{center}
\begin{tabular}{ll}
0:& 全トーラス \\
1:& 部分トーラス
\end{tabular}
\end{center}
      \item[IQTYPE] 光線追跡法の常微分方程式解法
\begin{center}
\begin{tabular}{ll}
0:& Runge-Kutta, 固定幅 \\
1:& Runge-Kutta, 自動幅 \\
2:& Runge-Kutta-Fahlberg, 自動幅 \\
\end{tabular}
\end{center}
      \vspace{3mm}
      \item[NRZMAX] 吸収パワーの径方向分布を求めるための分割数
      \item[NRADMX] ビーム追跡法における吸収パワー分布分割数
      \item[RCURVA] ビーム波面の初期曲率半径($\vk$ と $\vB$ に垂直)
      \item[RCURVB] ビーム波面の初期曲率半径($\vk$ と $\vk\times\vB$ に垂直)
      \item[RBRADA] ビームの初期半径($\vk$ と $\vB$ に垂直)
      \item[RBRADB] ビームの初期半径($\vk$ と $\vk\times\vB$ に垂直)
   \end{namelist}
   \end{itemize}

\subsection{{\tt R} : 光線追跡法を実行する}

\ttype{R} を入力した場合は光線追跡法を実行する.
\tline{170e3,10.8,,,-2000,,0.8/}
\\

など.現在値のままの場合は省略可能であり,/ は以降を省略する.
光線の数だけ繰り返す.パラメータの物理的意味を以下に示す.
\begin{itemize}
\item
\ttype{INTYPE=0}
   \begin{tabbing}
MMM\=MMMMMMM\=\kill
入力データ\\
    \> RF    \>周波数 [MHz] \\
      \> RPI   \>初期主半径 R [m] \\
      \> ZPI   \>初期垂直方向位置 Z [m]\\
      \> PHII  \>初期トロイダル角 [radian]\\
      \> RKR0  \>径方向波数の初期推測値(ニュートン法の初期値)[1/m] \\
      \> RNZI  \>初期垂直方向屈折率 \\
      \> RNPHII\>初期トロイダル方向屈折率\\
      \> UUI   \>規格化パワー初期値\\
\end{tabbing}   
\item
\ttype{INTYPE=1}
   \begin{tabbing}
MMM\=MMMMMMM\=\kill
入力データ\\
    \> RF    \>周波数 [MHz] \\
      \> RPI   \>初期主半径 R [m] \\
      \> ZPI   \>初期垂直方向位置 Z [m]\\
      \> PHII  \>初期トロイダル角 [radian]\\
      \> RKR0  \>径方向波数の初期推測値(ニュートン法の初期値)[1/m] \\
      \> ANGZ  \>初期ポロイダル方向入射角 \\
      \> ANGPH \>初期トロイダル方向入射角\\
      \> UUI   \>規格化パワー初期値\\
\end{tabbing}   
\item
\ttype{INTYPE=2}(未サポート)
   \begin{tabbing}
MMM\=MMMMMMM\=\kill
入力データ\\
    \> RF    \>周波数 [MHz] \\
      \> RPI   \>初期主半径 R [m] \\
      \> ZPI   \>初期垂直方向位置 Z [m]\\
      \> PHII  \>初期トロイダル角 [radian]\\
      \> MODEW \>モード選択(0:slow wave, 1:fast wave)\\
      \> ANGZ  \>初期ポロイダル方向入射角 \\
      \> ANGPH \>初期トロイダル方向入射角\\
      \> UUI   \>規格化パワー初期値\\
\end{tabbing}   
\end{itemize}

\subsection{{\tt G} : グラフ表示}
\begin{itemize}
\item['1']:ポロイダル軌跡とパワー分布
\item['2']:径方向依存性1
\item['3']:径方向依存性2
\item['4']:ビーム軌跡とパワー分布
\item['5']:偏光面角度と $s$
\item['6']:波数方向
\item['X']:終了
\end{itemize}




\section{FP コード}
FP コードは波動による電流駆動を解析するために,相対論効果や捕捉粒子の
寄与を含めて速度分布の時間発展を記述することができる.
\subsection{FP コードを使う}
\begin{itemize}
 \item[] {\tt R} : FP方程式の時間発展計算を開始する.
 \item[] {\tt C} : FP方程式の時間発展計算を続行する.
 \item[] {\tt P} : パラメータを変更する.
 \item[] {\tt V} : パラメータを表示する.
 \item[] {\tt G} : 結果のグラフを表示する.
 \item[] {\tt F} : 結果を ascii 形式でファイルに出力する.
 \item[] {\tt I} : 過去の履歴データをクリアする.
 \item[] {\tt W} : 結果を再表示する.
 \item[] {\tt Y} : FP方程式の係数を計算する.
 \item[] {\tt S} : 速度分布関数をファイルに保存する.
 \item[] {\tt L} : 速度分布関数をファイルから読み込む..
 \item[] {\tt Q} : 終了する.
\end{itemize}
作業の流れの例としては,{\tt P}でパラメータを変更した後,{\tt R}でフォッ
カープランク方程式を解き,,
{\tt G}でグラフを見たり,{\tt S}で速度分布関数を保存したりする.

コンパイル・パラメータ:

XXcomn.inc の中で配列の大きさを指定するパラメータが設定されており,それ
らを変更することにより,計算パラメータ領域を拡大あるいは縮小できる.

\title{fpcom1.inc}
\begin{tabbing}
\compileParmLine
\compileParm{NRM}{I}{25}{Maximum number of radial mesh}
\compileParm{NPM}{I}{50}{Maximum number of momentum amplitude mesh}
\compileParm{NTHM}{I}{50}{Maximum number of pitch angle mesh}
\compileParm{NTG1M}{I}{21}{Maximum number of detail time mesh}
\compileParm{NTG2M}{I}{501}{Maximum number of time mesh}
\compileParm{NCRM}{I}{5}{Maximum number of cyclotron harmonics}
\compileParm{NLM}{I}{13}{Maximum number of Legendre harmonics}
\end{tabbing}

\subsection{{\tt P,V} :  入力パラメータ}
入力パラメータは,namelist を用いて,任意のパラメータを変更することが
できる.入力行は,まず空白1文字の後,'\&fp ' に引き続いて,「パラメー
タ名=新しい値」の形式で設定を繰り返し,最後に,'\&end ' を入力して終
了する.
入力例は
\tline{\ \&fp\ DELT=0.1,NTMAX=5, \&end}
\\

以下に入力パラメータの説明と標準値を示す.

   \begin{itemize}
   \item[]    
   \begin{namelist}{RNPHIIxx}
      \item[R1    ] NRMAX=1 の場合の半径方向の位置 [m]
      \item[DELR1 ] NRMAX=1 の場合の半径方向の仮想的間隔 [m]
      \item[RMIN  ] NRMAX$\ne$1 の場合の最小半径
      \item[RMAX  ] NRMAX$\ne$1 の場合の最大半径
      \item[E0    ] トロイダル電場 [V/m] : 
      \item[DRR0  ] 半径方向の拡散係数 $[\mr m^2/\mr s]$
      \vspace{3mm}

      \item[DEC   ] 規格化された電子サイクロトロン波による拡散係数
      \item[PEC1  ] 電子サイクロトロン波の N パラレルスペクトルの中心
      \item[PEC2  ] 電子サイクロトロン波の N パラレルスペクトルの幅
      \item[RFEC  ] 電子サイクロトロン周波数[MHz]
      \item[DELYEC] 電子サイクロトロン波ビームの垂直幅 [m]
      \vspace{3mm}

      \item[DLH   ] 規格化された低域混成波の拡散係数
      \item[PLH1  ] 低域混成波スペクトル(最小速度,またはスペクトルの中心)
      \item[PLH2  ] 低域混成波スペクトル(最大速度,またはスペクトルの幅)
      \item[RLH   ] 低域混成波の最小近接小半径[m]
      \vspace{3mm}

      \item[DFW   ] 規格化された速波の拡散係数
      \item[RFW   ] 速波の最小近接短軸半径[m]
      \item[PFW1  ] 速波スペクトル(最小速度,またはスペクトルの中心)
      \item[PFW2  ] 速波スペクトル(最大速度,またはスペクトルの幅)
      \vspace{3mm}

      \item[RFDW  ] 周波数 [MHz] :
      \item[DELNPR] トロイダル方向屈折率のスペクトル幅
      \item[NCMIN ] サイクロトロン高調波番号の下限
      \item[NCMAX ] サイクロトロン高調波番号の上限
      \vspace{3mm}

      \item[CEWR  ] 波動電界の径方向成分
      \item[CEWTH ] 波動電界のポロイダル方向成分
      \item[CEWPH ] 波動電界のトロイダル方向成分
      \item[RKWR  ] 波数の径方向成分
      \item[RKWTH ] 波数のポロイダル方向成分
      \item[RKWPH ] 波数のトロイダル方向成分
      \item[REWY  ] 光線の垂直方向位置
      \item[DREWY ] 光線の垂直方向幅
      \vspace{3mm}

      \item[PMAX  ] 中心熱運動量で正規化された最大運動量
      \item[DELT  ] 時間ステップ幅[s]
      \item[RIMPL ] 計算の implicit パラメーター
      \item[EPSM  ] 行列方程式を解く場合の収束限界
      \item[EPSE  ] 電場計算での収束限界
      \item[LMAXE ] 電場計算での最大繰り返し数
      \item[EPSDE ] 2 重指数積分法での収束限界
      \item[H0DE  ] 2 重指数積分法での初期ステップ幅
      \item[NGLINE] 等高線の最大本数
      \item[LLMAX ] Legendre 展開の最大次数
      \vspace{3mm}

      \item[NPMAX ] 運動量の大きさ方向の分割数
      \item[NTHMAX] 運動量の角度方向の分割数
      \item[NRMAX ] 半径方向の分割数
      \item[NAVMAX] 波の拡散係数を計算するための軌道平均の分割数
      \vspace{3mm}

      \item[NTMAX ] 最大時間ステップ幅
      \item[NTSTP1] 半径方向分布形状データを保存するための時間ステップ幅
      \item[NTSTP2] 全体的なデータを保存するための時間ステップ幅
      \item[NTSTPC] 係数を再計算するための時間ステップ幅
      \vspace{3mm}

      \item[MODELE] 電場の計算をする場合には 1 とする
      \item[MODELR] 相対効果を含む場合には 1 とする
      \item[MODELA] バウンス平均をする場合には 1 とする
      \item[MODELC] 非線形衝突演算子を用いる場合には 1 とする
      \vspace{3mm}
      \item[MODELW] 拡散係数の計算モデル
\begin{eqnarray*}
0&:&近似モデル式\\
1&:&近似電磁界を用いて計算\\
2&:&波動伝播解析 WR の結果を読み込んで計算\\
\end{eqnarray*}

      \item[PWAVE ] 入力パワー
      \item[LMAXNWR] 光線と磁気面の交点を求める Newton 法の反復回数上限
      \item[EPSNWR] 光線と磁気面の交点を求める Newton 法の収束判定条件
   \end{namelist}
   \end{itemize}

\subsection{{\tt G, F} : グラフ表示およびファイル出力}
各種グラフ出力およびファイル出力の説明を以下に示す.
   \begin{tabbing}
MMM\=MMMMMMM\=\kill
\> F1   \>1次元速度分布\\
\> F2   \>2次元速度分布\\
\> FX2  \>2次元速度分布のピッチ角差分\\
\> FS11 \>1次元速度分布の内側境界値\\
\> FS12 \>2次元速度分布の内側境界値\\
\> FS21 \>1次元速度分布の外側境界値\\
\> FS22 \>2次元速度分布の外側境界値\\
\> DPP  \>拡散係数 $D_{pp}$ の運動量依存性\\
\> DPT  \>拡散係数 $D_{p\theta}$ の運動量依存性\\
\> DTP  \>拡散係数 $D_{\theta p}$ の運動量依存性\\
\> DTT  \>拡散係数 $D_{\theta\theta}$ の運動量依存性\\
\> DRR  \>拡散係数 $D_{rr}$ の運動量依存性\\
\> DCPP \>衝突拡散係数 $D_{\mr{C}pp}$ の運動量依存性\\
\> DCPT \>衝突拡散係数 $D_{\mr{C}p\theta}$ の運動量依存性\\
\> DCTP \>衝突拡散係数 $D_{\mr{C}\theta p}$ の運動量依存性\\
\> DCTT \>衝突拡散係数 $D_{\mr{C}\theta\theta}$ の運動量依存性\\
\> DCRR \>衝突拡散係数 $D_{\mr{C}rr}$ の運動量依存性\\
\> DWPP \>波動による拡散係数 $D_{\mr{W}pp}$ の運動量依存性\\
\> DWPT \>波動による拡散係数 $D_{\mr{W}p\theta}$ の運動量依存性\\
\> DWTP \>波動による拡散係数 $D_{\mr{W}\theta p}$ の運動量依存性\\
\> DWTT \>波動による拡散係数 $D_{\mr{W}\theta\theta}$ の運動量依存性\\
\> DWRR \>波動による拡散係数 $D_{\mr{W}rr}$ の運動量依存性\\
\> FP   \>摩擦係数 $F_{p}$ の運動量依存性\\
\> FT   \>摩擦係数 $F_{\theta}$ の運動量依存性\\
\> FR   \>摩擦係数 $F_{r}$ の運動量依存性\\
\> FCP  \>衝突摩擦係数 $F_{\mr{C}p}$ の運動量依存性\\
\> FCT  \>衝突摩擦係数 $F_{\mr{C}\theta}$ の運動量依存性\\
\> FCR  \>衝突摩擦係数 $F_{\mr{C}r}$ の運動量依存性\\
\> FEP  \>電界加速係数 $F_{\mr{E}p}$ の運動量依存性\\
\> FET  \>電界加速係数 $F_{\mr{E}\theta}$ の運動量依存性\\
\> FER  \>電界加速係数 $F_{\mr{E}r}$ の運動量依存性\\
\> RN   \>電子密度の径方向依存性\\
\> RI   \>電流密度の径方向依存性\\
\> RW   \>エネルギー密度の径方向依存性\\
\> RPC  \>衝突吸収パワー密度の径方向依存性\\
\> RPW  \>波動吸収パワー密度の径方向依存性\\
\> RPE  \>電界吸収パワー密度の径方向依存性\\
\> RT   \>温度の径方向依存性\\
\> RQ   \>安全係数の径方向依存性\\
\> RE   \>トロイダル電界の径方向依存性\\
\> TN   \>電子密度の時間依存性\\
\> TI   \>電流の時間依存性\\
\> TW   \>エネルギーの時間依存性\\
\> TPC  \>衝突吸収パワーの時間依存性\\
\> TPW  \>波動吸収パワーの時間依存性\\
\> TPE  \>電界吸収パワーの時間依存性\\
\> TT   \>温度の時間依存性\\
\> TQ   \>安全係数の時間依存性\\
\> TE   \>トロイダル電界の時間依存性\\ 
\end{tabbing}   

\section{WM コード}

\subsection{利用説明}

\begin{verbatim}
コマンド入力

  P:namelist 変数の入力
  V:namelist 変数等の表示
  W:アンテナ励起による波動伝播の計算
  A:与えられた複素周波数に対する振幅の計算
  F:振幅パラメータの実周波数依存性の計算
  C:振幅パラメータの複素周波数平面における等高線表示
  E:与えられた複素周波数から出発して固有周波数の計算
  S:固有周波数のパラメータ依存性の計算
  G:固有関数の図形表示
  Q:実行の終了

  コードにおける振幅パラメータの定義:
    プラズマ中の電子密度に比例する分布をもつ励起電流に対して,
    電界振幅の絶対値の自乗を空間積分した値の逆数

  コードにおける固有周波数の定義:
    複素周波数平面において振幅が極小をもち,
    極大値の逆数が EPSNW よりも小さい周波数

図形指定コマンド入力

    ABC/DEF/GHI は A,B,C の中から1文字,D,E,F の中から1文字,
  G,H,I の中から1文字の計3文字を順に並べた入力を表す.
  大文字と小文字は区別しない.

  注)現在,電流密度分布は正しく計算されていない.

  電磁界の径方向分布
    R/EB/ATMN
      R: 径方向分布
      E: 波動電界と吸収パワー密度
      B: 波動磁界と駆動電流密度
      A: ポロイダル角・トロイダル角指定(パワーは角度積分値)
      T: ポロイダル角・トロイダル角指定(パワーは角度指定)
      M: ポロイダルモード・トロイダルモード指定
      N: 複数ポロイダルモード・単一トロイダルモード指定

    例)REN は波動電界と吸収パワー密度の径方向分布を,
      複数のポロイダルモード成分を重ねて表示する.

  電磁界の2次元分布
    CPM/EB/RTZ/RIA
      C: 円形断面等高線表示
      P: ポロイダル断面等高線表示
      M: ポロイダルモード数・径方向面等高線表示
      E: 波動電界
      B: 波動磁界
      R: 径方向成分
      T: ポロイダル成分
      Z: トロイダル成分
      R: 電磁界の実数成分(励起電流と同位相の成分)
      I: 電磁界の虚数成分(励起電流と位相が 90 度遅れた成分)
      A: 電磁界の絶対値

  吸収パワー密度の2次元分布
    RCPM/P/123
      R: 径方向分布
      C: 円形断面等高線表示
      P: ポロイダル断面等高線表示
      M: ポロイダルモード数・径方向面等高線表示
      P: 吸収パワー密度
      1: 粒子種1=電子
      2: 粒子種2=通常は多数イオン
      3: 粒子種3=通常は少数イオン

  駆動電流密度の2次元分布
    CP/J
      C: 円形断面等高線表示
      P: ポロイダル断面等高線表示

  プラズマの空間分布
    P/F/TPQB
      T: 温度分布
      P: 圧力分布
      Q: 安全係数
      B: 磁界強度

  測度テンソル成分の径方向分布
    P/G

  ヘリカル系における磁気面
    S

  2次元表示の種類設定
    G/1234
      1: 等高線表示
      2: 塗り分け表示
      3: 鳥瞰図格子表示
      4: 鳥瞰図等高線表示

  図形表示の終了
    X
\end{verbatim}

\subsection{入力パラメータ}

\begin{verbatim}
C
C     **** ALPHA PARTICLE PARAMETERS ****
C
C        PNA   : Alpha density at center               (1.0E20/Mm*3)
C        PNAL  : Density scale length                            (m)
C        PTA   : Effective temperature                         (keV)
C
      PNA  = 0.02D0
      PNAL = 0.5D0
      PTA  = 3.5D3
C
C     **** ZEFF PARAMETERS ****
C
C        ZEFF  : Effective Z (sum n Z^2 / sum n Z)
C
         ZEFF  = 2.D0
C
C     *** WAVE PARAMETER ***
C
C     CRF   : Wave frequency                            (MHz)
C     RD    : Antenna minor radius                      (m)
C     BETAJ : Antenna current profile parameter
C     NTH0  : Central value of poloidal mode number
C     NPH0  : Central value of toroidal mode number
C     NHC   : Number of helical coils
C     PRFIN : Input Power (0 for given antenna current) (W)
C
      CRF    = (50.0D0,0.D0)
      RD     = 1.1D0
      BETAJ  = 0.D0
      NTH0   = 0
      NPH0   = 8
      NHC    = 10
      PRFIN  = 0.D0
C
C     *** ANTENNA PARAMETERS ***
C
C        NAMAX : Number of antennae
C        AJ    : Antenna current density                       (A/m)
C        APH   : Antenna phase                              (degree)
C        THJ1  : Start poloidal angle of antenna            (degree)
C        THJ2  : End poloidal angle of antenna              (degree)
C        PHJ1  : Start toroidal angle of antenna            (degree)
C        PHJ2  : End toroidal angle of antenna              (degree)
C
      NAMAX=1
      DO NA=1,NAM
         AJ(NA)   = 1.D0
         APH(NA)  = 0.D0
         THJ1(NA) =-45.D0
         THJ2(NA) = 45.D0
         PHJ1(NA) = 0.D0
         PHJ2(NA) = 0.D0
      ENDDO
C
C     *** MESH PARAMETERS ***
C
C        NRMAX : Number of radial mesh points
C        NTHMAX: Number of poloidal mesh points
C        NPHMAX : Number of toroidal mesh points
C
      NRMAX   = 50
      NTHMAX  = 1
      NPHMAX  = 1
C
C     *** CONTROL PARAMETERS ***
C
C        NPRINT: Control print output
C                   0: No print out
C                   1: Minimum print out (without input data)
C                   2: Minimum print out (with input data)
C                   3: Standard print out
C                   4: More print out
C        NGRAPH: Control graphic output
C                   0: No graphic out
C                   1: Standard graphic out (2D: Coutour)
C                   2: Standard graphic out (2D: Paint)
C                   3: Standard graphic out (2D: Bird's eye)
C        MODELJ: Control antenna current model
C                   0: Real antenna
C                   1: Real antenna
C                   2: Poloidal current
C                   3: Toroidal current
C                  2X: Vacuum eigen mode, poloidal current
C                  3X: Vacuum eigen mode, toroidal current
C                   0: Vacuum
C                   -1: MHD plasma
C                   -2: Cold plasma
C                   -3: Hot plasma (No FLR)
C                   -4: Hot plasma (Cold FLR)
C                   -5: Hot plasma (FLR)
C                   8: TASK/DP (No FLR)
C                   9: TAKS/DP (Cold FLR)
C        MODELA: Control alpha particle contribution
C                   0: No alpha effect
C                   1: Precession of alpha particles
C                   2: Precession of electrons
C                   3: Precession of both alpha particles and electrons
C                   4: Calculate alpha particle density using slowing down
C        MODELK: Control mode number cutoff
C                   0: No cutoff
C                   1: With cutoff (this should not be used)
C        MODELM: Control matrix solver
C                   0: BANDCD
C                   1: BCGCDB
C                   2: CGSCDB
C                   3: BCGSTAB
C                   4: BANDCDM
C                   5: BCGCDBM
C                   6: CGSCDBM
C                   7: BSTABCDBM
C                   8: BANDCDBM
C                   9: BCGCDBMA
C                  10: CGSCDBMA
C                  11: BSTABCDBMA
C                  12: BANDCDB
C        MODELW: Control writting a data of absorped power
C                   0: Not writting
C                   1: Writting
C
C        RHOMIN: rho at minimum q for reversed shear
C        QMIN  : q minimum for reversed shear
C        RHOITB: rho at ITB
C
      NPRINT = 2
      NGRAPH = 1
      MODELJ = 0
      MODELA = 0
      MODELK = 0
      MODELM = 2
      MODELW = 0
C
C     *** EIGEN VALUE PARAMETERS ***
C
C        FRMIN : Minimum real part of frequency in amplitude scan
C        FRMAX : Maximum real part of frequency in amplitude scan
C        FIMIN : Minimum imag part of frequency in amplitude scan
C        FIMAX : Maximum imag part of frequency in amplitude scan
C        FI0   : Imag part of frequency in 1D amplitude scan
C
C        NGFMAX: Number of real freq mesh in 1D amplitude scan
C        NGXMAX: Number of real freq mesh in 2D amplitude scan
C        NGYMAX: Number of imag freq mesh in 2D amplitude scan
C
C        SCMIN : Minimum value in parameter scan
C        SCMAX : Maximum value in parameter scan
C        NSCMAX: Number of mesh in parameter scan
C
C        LISTEG: Listing in parameter scan
C
C        FRINI : Initial real part of frequency in Newton method
C        FIINI : Initial imag part of frequency in Newton method
C
C        DLTNW : Step size in evaluating derivatives in Newton method
C        EPSNW : Convergence criterion in Newton method
C        LMAXNW: Maximum iteration count in Newton method
C        LISTNW: Listing in Newton method
C        MODENW: Type of Newton method
C
C        NCONT : Number of contour lines
C        ILN1  : Line type of lower contours
C        IBL1  : Line boldness of lower contours
C        ICL1  : Line color of lower contours
C        ILN2  : Line type of higher contours
C        IBL2  : Line boldness of higher contours
C        ICL2  : Line color of higher contours
C
      FRMIN = 0.1D0
      FRMAX = 1.D0
      FIMIN =-0.1D0
      FIMAX = 0.1D0
      FI0   = 0.D0
C
      FRINI = DBLE(CRF)
      FIINI = DIMAG(CRF)
C
      NGFMAX= 11
      NGXMAX= 11
      NGYMAX= 11
C
      SCMIN = 0.1D0
      SCMAX = 1.D0
      NSCMAX= 11
C
      LISTEG= 1
C
      DLTNW = 1.D-6
      EPSNW = 1.D-6
      LMAXNW= 10
      LISTNW= 1
      MODENW= 0
C
C     *** ALFVEN FREQUENCY PARAMETERS ***
C
C        WAEMIN : Minimum frequency in Alfven frequency scan
C        WAEMAX : Maximum frequency in Alfven frequency scan
C
      WAEMIN = 0.001D0
      WAEMAX = 0.200D0
\end{verbatim}

\end{document}
