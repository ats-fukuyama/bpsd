\documentclass[11pt]{article}
\usepackage{../../doc/af}
\usepackage{../../doc/sym}
\usepackage{../../doc/doc}

\begin{document}
\begin{flushright}
2021/10/11
\end{flushright}

\begin{center}
\textbf{\Large User Manual of orbit following code TASK/OB}
\end{center}

\tableofcontents

\section{Outline of TASK/OB}

\subsection{Purpose of TASK/OB}

The purpose of TASK/OB is to describe charged particle orbits in a
given magnetic configuration.  At present, gyro orbits in the Boozer
coordinates are described based on the textbook by R. White.  More
general coordinates including vacuum regions and full orbit
description will be included in future.  

\subsection{Modules included}

\begin{center}
  \begin{tabular}{ll}
	obcomm\_parm & Deficnition of input parameters \\
	obcomm & Definition of common variables \\
	obinit & Initialization of input parameters \\
	obparm & Procedures of input parameters (read, check,
        broadcast) \\
	obview & Print out input parameters \\
	obmenu & Command menu \\
	obprep & Preparation for calculation (equilibrium,
        interpolation) \\
	obcalc & Calculation of coefficients \\
	obexec & Solving equation of motion \\
	obgout & Visualization of orbits \\
	obfile & File output of orbits \\
	obsub & Common subroutines using obcomm \\
	oblib & Common subroutines independent of obcomm
  \end{tabular}
\end{center}

\section{Parameters}

\subsection{Adjustable parameters, but fixed at compilation time}

\begin{center}
  \begin{tabular}{lrl}
    nobt\_m & 100 & maximum number of orbits
  \end{tabular}
\end{center}

\subsection{Input parameters and their default values}

\begin{center}
  \begin{tabular}{lrl}
    modelg & 3 & geometry model (parameter of plparm) \\
    \\
    nobt\_max &     1 & number of orbits \\
    nstp\_max & 10000 & maximum number of orbit step \\
    ns\_ob    &     2 & id of particle species \\
    lmax\_nw  &    20 & maximum number of iteration (initial
    condition)
  \end{tabular}
\end{center}

\begin{center}
  \begin{tabular}{lrl}
    mdlobp   &     0 & model id of equation of motion \\
                    && \quad 0: Eq of Motion  with Boozer coordinates \\
                    && \quad 1: Eq of Motion  with Cylindrical coord. \\
    mdlobi   &     0 & model id of input scheme of initial parameters \\
                    && \quad 0: penergy,pcangle,zeta,psipn,theta \\
                    && \quad 1: penergy,pcangle,zeta,rr,zz (TBI)\\
                    && \quad 100: line input with psipn,theta \\
                    && \quad 101: line input with rr,zz (TBI)\\
    mdlobq   &     0 & model id of ODE solver \\
                    && \quad 0: 4th-order Runge-Kutta-Gill \\
                    && \quad 1: universal ODE solver (TBI) \\
                    && \quad 2: symplectic solver (TBI) \\
    mdlobt   &     1 & model id of time normalization \\
                    && \quad 0: real time \\
                    && \quad 1: normalized by approximate bounce time \\
    mdlobc   &     0 & model id of one cycle calculation \\
                    && \quad 0: independent of cycle, until tmax\_ob \\
                    && \quad 1: one cycle for trapped and untrapped \\
    mdlobw   &     3 & model id of output interval \\
                    && \quad 0: no output \\
                    && \quad 1: every step \\
                    && \quad 2: every 10 step \\
                    && \quad 3: every 100 step \\
                    && \quad 4: every 1000 step \\
                    && \quad 5: every 10000 step \\
    mdlobg   &     0 & model id of graphics \\
                    && \quad 0: default \\
    mdlobx   &     1 & model id of wall \\
                    && \quad 0: calculate only inside the wall
                                (psip\_ob<=psipa) \\
                    && \quad 1: continue Runge-Kutta
                                (psip\_ob>psipa) \\
    \\
    tmax\_ob  & 10.D0 & maximum of orbit following time in omega\_bounce \\
    delt\_ob  & 0.1D0 & time step size in omega\_bounce \\
                    && \quad t\_bounce = 2 Pi/ omega\_bounce \\
                    && \quad omega bounce = (v\_perp/qR) SQRT(r/2R) \\
                    && \quad omega\_bounce$^2$ = (mu B /m)*(r/q$^2$ R$^3$) \\
    eps\_ob   & 1.D-6 & convergence criterion of orbit solution \\
    del\_ob   & 1.D-4 & step size of iteration (initial condition) \\
    eps\_nw   & 1.D-6 & convergence criterion of iteration (initial c.) \\
    \\
    penergy\_ob\_in(1) & 1.D0  & initial particle energy (mdlobi=0,1)
    [keV]: conserved \\
    pcangle\_ob\_in(1) & 0.5D0 & initial cosine of pitch angle
    (mdlobi=0,1) \\
    zeta\_ob\_in(1)    & 0.D0  & initial toroidal angle (mdlobi=0,1)
    [degree] \\
    psipn\_ob\_in(1)   & 0.5D0 & initial normalized poloidal flux
    (mdlobi=0) \\
    theta\_ob\_in(1)   & 0.D0  & initial poloidal angle (mdlobi=0) [deg] \\
    rr\_ob\_in(1)      & 4.D0  & initial major radius (mdlobi=1) [m] \\
    zz\_ob\_in(1)      & 0.D0  & initial vertical position (mdlobi=1)
    [m] \\
    \\
    nrmax\_ob   & 100  & number of equilibrium radial meshes \\
    nthmax\_ob  &  64  & number of equilibrium poloidal meshes \\
    nsumax\_ob  & 100  & number of equilibrium plasma boundary meshes
  \end{tabular}
\end{center}

\subsection{Initial orbit parameters and results}

\begin{itemize}
\item
  Initial orbit parameters: (nobt)
  \begin{center}
    \begin{tabular}{ll}
    penergy\_ob\_in(1) & initial particle energy (mdlobi=0,1)
    [keV] \\
    pcangle\_ob\_in(1) & initial cosine of pitch angle
    (mdlobi=0,1) \\
    zeta\_ob\_in(1)    & initial toroidal angle (mdlobi=0,1)
    [degree] \\
    psipn\_ob\_in(1)   & initial normalized poloidal flux
    (mdlobi=0) \\
    theta\_ob\_in(1)   & initial poloidal angle (mdlobi=0)
    [deg] \\
    nthmax\_ob\_in(1)   & number of equilibrium poloidal meshes \\
    nsumax\_ob\_in(1)   & number of equilibrium plasma boundary meshes
    \end{tabular}
  \end{center}

\item
  Initial variables:  (nobt)
  \begin{center}
    \begin{tabular}{ll}
       zetab\_pos     & toroidal boozer angle zeta translated from obts \\
       thetab\_pos    & poloidal boozer angle theta \\
       psip\_pos      & poloidal magnetic flux \\
       rhopara\_pos   & parallel velocity devidede by cyclotron freq.
    \end{tabular}
  \end{center}

\item
  Orbit results: (nstp,nobt)
  \begin{center}
    \begin{tabular}{ll}
       nstp\_max\_nobt & number of steps for nobt \\
       time\_ob       & time \\
       zetab\_ob      & toroidal boozer angle zeta translated from obts \\
       thetab\_ob     & poloidal boozer angle theta \\
       psip\_ob       & poloidal magnetic flux \\
       rhopara\_ob    & parallel velocity devidede by cyclotron freq. \\
       pzeta\_ob      & toroidal momentum pzeta \\
       ptheta\_ob     & poloidal momentum ptheta \\
       babs\_ob       & absolute value of magnetic field \\
       phi\_ob        & electrostatic potential \\
       vpara\_ob      & parallel velocity \\
       vperp\_ob      & perpendicular velocity \\
       psit\_ob       & toroidal magnetic flux \\
       zeta\_ob       & toroidal angle \\
       rr\_ob         & major radius \\
       zz\_ob         & vertical positon \\
       rs\_ob         & minor radius \\
       theta\_ob      & poloidal angle
    \end{tabular}
  \end{center}
\end{itemize}

\end{document}
