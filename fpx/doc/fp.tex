\documentclass[11pt]{article}
\usepackage{../../doc/af}
\usepackage{../../doc/sym}
\usepackage{../../doc/doc}

\begin{document}
\begin{flushright}
2021/09/13
\end{flushright}

\begin{center}
\textbf{\Large User Manual of the TASK/FP Code}
\end{center}

\tableofcontents

\section{Structure}

\subsection{Source File}

\begin{itemize}
\item
  Environment routines
  \begin{itemize}

  \item \ttype{fpcomm.f90}: Definition of variable module, allocation of
    arrays
    \begin{itemize}
    \item Module \ttype{fpcomm\_parm}: Definition of constants and
      input parameters (USE bpsd\_kinds, bpsd\_constants, commpi,
      plcomm, obcomm\_parm)
    \item Module \ttype{fpcomm}: Definition of common variables
    \item Subroutine \ttype{fp\_allocate}: Allocation of common arrays
    \item Subroutine \ttype{fp\_deallocate}: Deallocation of common arrays
    \item Subroutine \ttype{fp\_allocate\_ntg1}: Allocation of time
      history \ttype{ntg1} array
    \item Subroutine \ttype{fp\_deallocate\_ntg1}: Deallocation of time
      history \ttype{ntg1} array
    \item Subroutine \ttype{fp\_adjust\_ntg1}: Adjust (save,
      deallocate, allocate larger, recover) time
      history \ttype{ntg1} array
    \item Subroutine \ttype{fp\_allocate\_ntg2}: Allocation of
      profile time history \ttype{ntg2} array
    \item Subroutine \ttype{fp\_deallocate\_ntg2}: Deallocation of
      profile time history \ttype{ntg2} array
    \item Subroutine \ttype{fp\_adjust\_ntg2}: Adjust (save,
      deallocate, allocate larger, recover) profile time history
      \ttype{ntg2} array
    \end{itemize}

  \item \ttype{fpmain.f90}: Main routine
    \begin{itemize}
    \item
      Initialization of MPI and graphics
    \item
      Initialization of input parameters of module, pl, eq, and fp
    \item
      Read input namelist file \ttype{fpparm}
    \item
      Start menu input loop \ttype{fpmenu}
    \item
      Termination of graphics and MPI
    \end{itemize}

  \item \ttype{fpinit.f90}: Initialization of input parameters
    \begin{itemize}
    \item
      Module \ttype{fp\_init}: Set default values of input parameters
    \end{itemize}
    
  \item \ttype{fpmenu.f90}: Process control
    \begin{itemize}
    \item
      Module \ttype{fp\_menu}: menu command input and execute
      \begin{itemize}
      \item
        \ttype{P} or with \ttype{=}: namelist read of input parameters
      \item
        \ttype{V}: view all input parameters
      \item
        \ttype{R}: start new calculation (set initial condition and
        go)
      \item
        \ttype{C}: continue previous calculation (just go)
      \item
        \ttype{G}: graphic output of calculated results
      \item
        \ttype{F}: file output of calculated results
      \item
        \ttype{S}: save present status of calculation into a file
      \item
        \ttype{L}: load a file to continue previous calculation
      \item
        \ttype{W}: print out interesting quantities
      \item
        \ttype{Y}: debug output [temporal]
      \item
        \ttype{Z}: debug output [transport coefficients]
      \item
        \ttype{Q}: quit menu input and stop
      \end{itemize}
    \end{itemize}
    
  \item \ttype{fpparm.f90}: Read and check input parameters
    \begin{itemize}
    \item
      Subroutine \ttype{fp\_parm}: analyze parameter input
      \begin{itemize}
      \item
        read one line from standard input and analyze command or
        namelist
      \item
        read namelist input from a file
      \item
        read one phrase and analyze as a namelist
      \end{itemize}
    \end{itemize}
  \item \ttype{fpbroadcast.f90}: Broadcast input parameters
  \item \ttype{fpview.f90}: Show input parameters
  \end{itemize}

\item
    Execution routines
    \begin{itemize}
    \item[] \ttype{fpprep.f90}: Preparation of run (mesh, initial profile,
      initialization)
    \item[] \ttype{fpbounc.f90}: Preparation of bounce parameters
    \item[] \ttype{fploop.f90}: Time loop for execution
    \item[] \ttype{fpexec.f90}: Execution of one time step
\end{itemize}
\item
  Calculation of coefficients of equations
  \begin{itemize}
  \item[] \ttype{fpcoef.f90}: Calculation of various coefficients
  \item[] \ttype{fpcalc.f90}: Calculation of collisional term (linear operator)
  \item[] \ttype{fpcalcn.f90}: Calculation of collisional term (nonlinear operator)
  \item[] \ttype{fpcalcnr.f90}: Calculation of collisional term (relativistic nonlinear operator)
  \item[] \ttype{fpcale.f90}: Calculation of static electric field term
  \item[] \ttype{fpcalr.f90}: Calculation of radial diffusion term
  \item[] \ttype{fpcalw.f90}: Calculation of quasi-linear term (given wave field)
  \item[] \ttype{fpcalwm.f90}: Calculation of quasi-linear term (using wm results)
  \item[] \ttype{fpcalwr.f90}: Calculation of quasi-linear term (using wr results)
  \item[] \ttype{fpcdbm.f90}: Calculation of CDBM radial diffusion coefficients
  \item[] \ttype{fpnfrr.f90}: Calculation of fusion reaction term (isotropic distribution)
  \item[] \ttype{fpnflg.f90}:  Calculation of fusion reaction term (anisotropic distribution)
  \item[] \ttype{fpdisrupt.f90}: Calculation of disruption-related tems
  \end{itemize}
\item
  Calculation of transport coefficients (by Ota)
  \begin{itemize}
  \item[] \ttype{fpcaldeff.f90}: Effective particle diffusion coefficients
  \item[] \ttype{fpcalchieff.f90}: Effectiv thermal diffusion coefficients
  \item[] \ttype{fpcaltp.f90}: Particle confinement time, tauP
  \item[] \ttype{fpcalte.f90}: Energy confinement time, tauE
  \item[] \ttype{fpchecknc.f90}: Radial diffusion coefficients (neoclassical diffusion)
  \end{itemize}
\item
  File IO routines
  \begin{itemize}
  \item[] \ttype{fpfile.f90}: Save and load restart data
  \item[] \ttype{fpfout.f90}: File output of graphic data
  \item[] \ttype{fpoutdata.f90}: File output of intermediate data (by Ota)
  \item[] \ttype{fpread.f90}: Read FIT3D data from file
  \item[] \ttype{fpreadeg.f90}: Read Experimental data from file (by Nuga)
  \item[] \ttype{fpsave.f90}: File output of various data (by Nuga)
  \item[] \ttype{fpwmin.f90}: File input of full-wave analysis (wm) results
  \item[] \ttype{fpwrin.f90}: File input of ray/beam tracing analysis
    (wr) results
  \item[] \ttype{fpwrite.f90}: File output of trcoef data (by Ota)
  \end{itemize}
\item
  Graphic routines
  \begin{itemize}
  \item[] \ttype{fpgout.f90}: Graphic output for gsaf
  \item[] \ttype{fpcont.f90}: Graphic subroutines
  \end{itemize}
\item
  Library routines
  \begin{itemize}
  \item[] \ttype{fpmpi.f90}: MPI interface for fp
  \item[] \ttype{fpsub.f90}: Subroutine library (FPMXWL, FPNEWTON) 
  \end{itemize}

\item
  Orbit-averaging routines (by Ota)
  \begin{itemize}
  \item[] \ttype{fowcomm.f90}: Definition of fow variables and
    allocation 
  \item[] \ttype{fowprep.f90}: Preparation of fow, initialization of variables
  \item[] \ttype{foworbit.f90}: Interface to ob
  \item[] \ttype{fowclassify.f90}: Orbit classification and data output to file
  \item[] \ttype{fowdistribution.f90}: Distribution conversion and
    integrated quantities
  \item[] \ttype{fowloop.f90}: Time loop for execution
  \item[] \ttype{fowexec.f90}: Execution of one time step
  \item[] \ttype{fowcoef.f90}: Calculation of various coefficients
  \item[] \ttype{fowsource.f90}: Calculation of source terms
  \item[] \ttype{fowlib.f90}: Library for fow
  \end{itemize}
\end{itemize}

\end{document}

