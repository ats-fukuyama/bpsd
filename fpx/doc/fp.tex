\documentclass[11pt]{article}
\usepackage{geometry}
\geometry{a4paper}
\usepackage[dvips]{graphicx,color}
\usepackage{../../doc/af}
\usepackage{../../doc/sym}
\usepackage{../../doc/emt}
\usepackage{../../doc/doc}

\newcommand{\hfile}[1]{\emAs{\ttype{#1}}}
\newcommand{\hprog}[1]{\emC{Program \ttype{#1}}}
\newcommand{\hmod}[1]{\emC{Module \ttype{#1}}}
\newcommand{\hsub}[1]{\emB{Subroutine \ttype{#1}}}
\newcommand{\hsubx}[1]{\emBs{Subroutine \ttype{#1}}}
\newcommand{\hfunc}[1]{\emB{Function \ttype{#1}}}
\newcommand{\hfuncx}[1]{\emBs{Function \ttype{#1}}}

\begin{document}
\begin{flushright}
2021/09/13
\end{flushright}

\begin{center}
\textbf{\Large User Manual of the TASK/FP Code}
\end{center}

\tableofcontents

\section{Structure}

\subsection{\emA{Environment routines}}

\begin{itemize}

  \item \hfile{fpcomm.f90}: Definition of variable module, allocation of
  arrays
  \begin{itemize}
    \item \hmod{fpcomm\_parm}: Definition of constants and input
      parameters including
      \begin{itemize}
      \item
        bpsd\_kinds
      \item
        bpsd\_constants
      \item
        commpi
      \item
        plcomm
      \item
        obcomm\_parm
      \end{itemize}
    \item \hmod{fpcomm}: Definition of common variables
    \item \hsub{fp\_allocate}: Allocation of common arrays
    \item \hsub{fp\_deallocate}: Deallocation of common arrays
    \item \hsub{fp\_allocate\_ntg1}: Allocation of time history
      \ttype{ntg1} array
    \item \hsub{fp\_deallocate\_ntg1}: Deallocation of time history
      \ttype{ntg1} array
    \item \hsub{fp\_adjust\_ntg1}: Adjust (save, deallocate,
      allocate larger, recover) time history \ttype{ntg1} array
    \item \hsub{fp\_allocate\_ntg2}: Allocation of profile time
      history \ttype{ntg2} array
    \item \hsub{fp\_deallocate\_ntg2}: Deallocation of profile time
      history \ttype{ntg2} array
    \item \hsub{fp\_adjust\_ntg2}: Adjust (save, deallocate,
      allocate larger, recover) profile time history \ttype{ntg2}
      array
  \end{itemize}

  \item \hfile{fpmain.f90}: Main routine
    \begin{itemize}
    \item \hprog{fp}
      \begin{itemize}
      \item
        Initialization of MPI and graphics
      \item
        Initialization of input parameters of module, pl, eq, and fp
      \item
        Read input namelist file \ttype{fpparm}
      \item
        Start menu input loop \ttype{fpmenu}
      \item
        Termination of graphics and MPI
      \end{itemize}
    \end{itemize}
    
  \item \hfile{fpinit.f90}: \hmod{fpinit}:
    Initialization of input parameters
    \begin{itemize}
    \item
      \hsub{fp\_init}: Set default values of input parameters
    \end{itemize}
    
  \item \hfile{fpmenu.f90}: \hmod{fpmenu}: Process control
    \begin{itemize}
    \item
      \hsub{fp\_menu}: menu command input and execute
      \begin{itemize}
      \item
        \ttype{P} or with \ttype{=}: namelist read of input parameters
      \item
        \ttype{V}: view all input parameters
      \item
        \ttype{R}: start new calculation (set initial condition and
        go)
      \item
        \ttype{C}: continue previous calculation (just go)
      \item
        \ttype{G}: graphic output of calculated results
      \item
        \ttype{F}: file output of calculated results
      \item
        \ttype{S}: save present status of calculation into a file
      \item
        \ttype{L}: load a file to continue previous calculation
      \item
        \ttype{W}: print out interesting quantities
      \item
        \ttype{Y}: debug output [temporal]
      \item
        \ttype{Z}: debug output [transport coefficients]
      \item
        \ttype{Q}: quit menu input and stop
      \end{itemize}
    \end{itemize}
    
  \item \hfile{fpparm.f90}: \hmod{fpparm}:
    Read and check input parameters
    \begin{itemize}
    \item
      \hsub{fp\_parm}: analyze parameter input
      \begin{itemize}
      \item
        read one line from standard input and analyze command or
        namelist
      \item
        read namelist input from a file
      \item
        read one phrase and analyze as a namelist
      \end{itemize}
    \item \hsubx{fp\_nlin}: namelist input slave routine
    \item \hsubx{fp\_plst}: input parameter list slave routine
    \item \hsubx{fp\_check}: input parameter check
    \item \hsub{fp\_broadcast}: Broadcast input parameters
    \item \hsub{fp\_view}: Show input parameters
    \end{itemize}
\end{itemize}

\subsection{\emA{Execution routines}}
\begin{itemize}
\item
  \hfile{fpprep.f90}: \hmod{fpprep}:
  Preparation of run (mesh, initial profile, initialization)
  \begin{itemize}
  \item \hsub{fp\_mesh}: create mesh quantities
    \begin{itemize}
    \item
      set \ttype{pmax}
    \item
      read EQ data \emAs{(RKAP=1.0 should be removed)}
    \item
      set radial mesh
    \item
      read WR data
    \item
      read WM data
    \item
      set approximate poloidal magnetci field
      \emAs{(exact poloidal magnetic field should be used)}
    \item
      set momentum space mesh
    \item
      set element volume in real space
    \item
      set bounce-average parameters
    \end{itemize}
  \item
    \hsub{FPCINI}: clear friction and diffusion coefficients
  \item
    \hsub{fp\_comm\_setup}: check partitions and setup shadow and commucicators
  \item
    \hsub{fp\_set\_nsa\_nsb}: set table on nsa and nsb
  \item
    \hsubx{FNSP\_INIT}: initialize distribution functions FNSP and delta f
  \item
    \hsub{FNSP\_INIT\_EDGE}: initialize edge distribution functions FS1, FS2
  \item
    \hsub{fp\_set\_normalize\_param}: set normalized density,
    temprature and other quantities
  \item
    \hsub{Coulomb\_log}: calculate Coulomb logarithm
  \item
    \hsub{fp\_continue}: setup for continuation
  \item
    \hsub{fp\_set\_intial\_value\_from\_f}: record initial values
  \item
    \hsub{fp\_prep}:
    \begin{itemize}
    \item
      initialize time counter
    \item
      allocate variables
    \item
      setup matrix parameters for MPI
    \item
      setup particle species table (fp\_set\_nsa\_nsa)
    \item
      create mesh (fp\_mesh)
    \item
      initialize diffusion coefficients (FPCINI)
    \item
      set parameters (fp\_set\_normalize\_param)
    \item
      initialize distribution functions (FNSP\_INIT, FNSP\_INIT\_EDGE)
    \item
      set background distribution functions (update\_fnsb)
    \item
      read FIT3D rsults for NBI (READ\_FIT3D, SV\_WEIGHT\_R)
    \item
      set parallel electric field
    \item
      setup for Legendre expansion and fusion reaction rate (NF\_*)
    \item
      record initial values (fp\_set\_initial\_value\_from\_f)
    \end{itemize}
  \end{itemize}

\item
  \hfile{fpbounce.f90}: Preparation of bounce parameters
  \begin{itemize}
  \item
    \hsub{set\_bounce\_parm}: calculate bounce parameters
    \begin{itemize}
    \item
      set\_etamg, set\_rlambda, set\_rlambda\_tpb\_from\_dens,
      set\_etamg\_nrmaxp1, set\_rlambda\_nrmaxp1,
      set\_rlambda\_tpb\_from\_den\_nrmaxp1, set\_etamg\_rg,
      set\_rlambda\_rg
    \end{itemize}
  \item
    \hsubx{set\_etamg}: set bounce point, \ttype{etam} on \ttype{thm},
    \ttype{etag} on \ttype{thg}, on \ttype{rm}
  \item
    \hsubx{set\_etamg\_nrmaxp1}: set \ttype{etam} and \ttype{etag} at
    \ttype{rm(nrmax+1)}
  \item
    \hsubx{set\_etamg\_rg}: set bounce point, \ttype{etam} on
    \ttype{thm}, \ttype{etag} on \ttype{thg}, on \ttype{rg}
  \item
    \hsubx{set\_rlambda\_tpb}: set \ttype{rlambda} on boundary layer
    on \ttype{rm}
  \item
    \hsubx{set\_rlambda\_tpb\_g}: set \ttype{rlambda} on boundary layer
    on \ttype{rg}
  \item
    \hsubx{recurrence\_alpha}: \ttype{alpha(i)=alpha(i-1)*(2*i-3)/(20*i)}
  \item
    \hsubx{recurrence\_beta}:  \ttype{beta(i)=(2*(2*i-2)*beta(i-1)
                                         -(2*i-3)*beta(i-2))/(2*i-1)}
  \item
    \hsubx{set\_rlambda\_ell}: \ttype{rlamda} on \ttype{rm}
  \item
    \hsubx{set\_rlambda\_ell\_g}: \ttype{rlamda} on \ttype{rg}
  \item
    \hsubx{recurrence\_j}:  \ttype{j2m(i)=((2*i-2)*(1+z**2)*j2m(i-1)
                                         -(2*i-3)*z**2*j2m(i-2))/(2*i-1)}
  \item
    \hsub{set\_rfsad}: calculate flux surface average of oint 1/psi ds
  \item
    \hsubx{set\_rlambda}: set \ttype{rlambda} on \ttype{rm}
  \item
    \hsubx{set\_rlambda\_nrmaxp1}: \ttype{rlamda} on \ttype{rm(nrmax+1)}
  \item
    \hsubx{set\_rlambda\_rg}: set \ttype{rlambda} for \ttype{etam\_rg}
  \item
    \hsubx{set\_rlambda\_rgmax}: set \ttype{rlambda} for \ttype{etamg\_rg}
  \item
    \hsubx{set\_etamg\_rgmax}: set \ttype{etama\_rg} for \ttype{eps\_rg2}
  \item
    \hsubx{set\_rlambda\_tpb\_gmax}: set \ttype{rlambda} on boundary layer
    on \ttype{rg2}
  \item
    \hsubx{set\_rlambda\_ell\_gmax}: set \ttype{rlambda} on \ttype{rg2}
  \item
    \hsubx{set\_rlambda\_tpb4}:
  \item
    \hsubx{set\_rlambda\_tpb3}:
  \item
    \hsubx{set\_rlambda\_tpb3\_g}:
  \item
    \hsubx{set\_rlambda\_tpb\_from\_dens}:
  \item
    \hsubx{set\_rlambda\_tpb\_from\_dens\_nrmaxp1}:
  \item
    \hsubx{set\_rlambda\_rg\_tpb\_from\_dens}:
  \end{itemize}
      
\item
  \hfile{fploop.f90}: Time loop for execution
  \begin{itemize}
  \item
    \hsub{fp\_loop}
  \item
    \hsubx{solve\_matrix\_update\_FNS0}
  \item
    \hsubx{implicit\_convergence\_update\_FNSP}
  \item
    \hsubx{bulk\_const1\_for\_non\_exp}
  \item
    \hsubx{bulk\_const1\_for\_exp}
  \item
    \hsubx{update\_radial\_f\_boundary}
  \end{itemize}
\item
  \hfile{fpexec.f90}: Execution of one time step
\end{itemize}

\subsection{\emA{Calculation of coefficients of equations}}
  \begin{itemize}
  \item \hfile{fpcoef.f90}: Calculation of various coefficients
  \item \hfile{fpcalc.f90}: Calculation of collisional term (linear operator)
  \item \hfile{fpcalcn.f90}: Calculation of collisional term (nonlinear operator)
  \item \hfile{fpcalcnr.f90}: Calculation of collisional term (relativistic nonlinear operator)
  \item \hfile{fpcale.f90}: Calculation of static electric field term
  \item \hfile{fpcalr.f90}: Calculation of radial diffusion term
  \item \hfile{fpcalw.f90}: Calculation of quasi-linear term (given wave field)
  \item \hfile{fpcalwm.f90}: Calculation of quasi-linear term (using wm results)
  \item \hfile{fpcalwr.f90}: Calculation of quasi-linear term (using wr results)
  \item \hfile{fpcdbm.f90}: Calculation of CDBM radial diffusion coefficients
  \item \hfile{fpnfrr.f90}: Calculation of fusion reaction term (isotropic distribution)
  \item \hfile{fpnflg.f90}:  Calculation of fusion reaction term (anisotropic distribution)
  \item \hfile{fpdisrupt.f90}: Calculation of disruption-related tems
  \end{itemize}

\subsection{\emA{Calculation of transport coefficients (by Ota)}}
  \begin{itemize}
  \item \hfile{fpcaldeff.f90}: Effective particle diffusion coefficients
  \item \hfile{fpcalchieff.f90}: Effectiv thermal diffusion coefficients
  \item \hfile{fpcaltp.f90}: Particle confinement time, tauP
  \item \hfile{fpcalte.f90}: Energy confinement time, tauE
  \item \hfile{fpchecknc.f90}: Radial diffusion coefficients (neoclassical diffusion)
  \end{itemize}

\subsection{\emA{File IO routines}}
  \begin{itemize}
  \item \hfile{fpfile.f90}: Save and load restart data
  \item \hfile{fpfout.f90}: File output of graphic data
  \item \hfile{fpoutdata.f90}: File output of intermediate data (by Ota)
  \item \hfile{fpread.f90}: Read FIT3D data from file
  \item \hfile{fpreadeg.f90}: Read Experimental data from file (by Nuga)
  \item \hfile{fpsave.f90}: File output of various data (by Nuga)
  \item \hfile{fpwmin.f90}: File input of full-wave analysis (wm) results
  \item \hfile{fpwrin.f90}: File input of ray/beam tracing analysis
    (wr) results
  \item \hfile{fpwrite.f90}: File output of trcoef data (by Ota)
  \end{itemize}

\subsection{\emA{Graphic routines}}
  \begin{itemize}
  \item \hfile{fpgout.f90}: Graphic output for gsaf
  \item \hfile{fpcont.f90}: Graphic subroutines
  \end{itemize}

\subsection{\emA{Library routines}}
  \begin{itemize}
  \item \hfile{fpmpi.f90}: MPI interface for fp
  \item \hfile{fpsub.f90}: Subroutine library (FPMXWL, FPNEWTON) 
  \end{itemize}

\subsection{\emA{Orbit-averaging routines (by Ota)}}
  \begin{itemize}
  \item \hfile{fowcomm.f90}: Definition of fow variables and
    allocation
    \begin{itemize}
    \item
      \hmod{fowcomm}: define quantities related to orbit-averaging
    \item
      \hsub{fow\_allocate}: allocate adjustable arrays
    \item
      \hsub{fow\_deallocate}: deallocate adjustable arrays
    \end{itemize}

  \item \hfile{fowprep.f90}: \hmod{fowprep}: Preparation of fow,
    initialization of variables
    \begin{itemize}
    \item
      \hsub{search\_pinch\_orbit}: calculate pinch orbit
    \item
      \hsub{calculate\_jacobian}: calculation of jacobians
    \end{itemize}
  \item \hfile{foworbit.f90}: \hmod{foworbit}: Interface to ob
    \begin{itemize}
    \item
      \hsub{fow\_set\_obparm}
      \begin{itemize}
      \item
        ob\_init, ob\_parm, ob\_prep, ob\_allocate, and spline psim,
        Fpsi, B, dradpsi
      \end{itemize}
    \item
      \hsub{fow\_orbit}
      \begin{itemize}
      \item
        calculated orbits
      \end{itemize}
    \item
      \hsub{fow\_cal\_local\_COMs}
      \begin{itemize}
      \item
        calculate thetaml, rhoml, tau\_loss
      \end{itemize}
    \item
      \hsubx{construct\_orbit}
      \begin{itemize}
      \item
        call ob\_calc to calculate ob structure
      \end{itemize}
    \item
      \hsubx{construct\_orbit\_zero}
      \begin{itemize}
      \item
        set zero into ob structure
      \end{itemize}
    \item
      \hsubx{save\_orbit}
      \begin{itemize}
      \item
        save orbit data into file
      \end{itemize}
    \item
      \hsubx{load\_orbit}
      \begin{itemize}
      \item
        load orbit data from file
      \end{itemize}
    \item
      \hsub{quantities\_at\_Bminimum}
      \begin{itemize}
      \item
        calculate quantities at B\_min
      \end{itemize}
    \item
      \hsub{mean\_ra\_quantities}
      \begin{itemize}
      \item
        calculate quantities at average radius
      \end{itemize}
    \end{itemize}

      
  \item \hfile{fowclassify.f90}: \hmod{fowcllasify}: Orbit
    classification and data output to file
    \begin{itemize}
    \item
      \hsub{output\_orbit\_cllasify}:
      \begin{itemize}
      \item
        prep\_orbit\_classify
      \item
        pinch\_orbit
      \item
        D\_orbit(beta\_D)
      \item
        stagnation\_orbit(beta\_stag)
      \item
        stagnation\_type(xi\_Xtype\_boundary\_ion)
      \item
        output to file dat/*\_obclass.txt
      \end{itemize}
    \item
      \hsubx{prep\_orbit\_classify}: calculate \ttype{theta\_m} and
      \ttype{xi}
    \item
      \hsubx{pinch\_orbit}: calculate all pinch points
    \item
      \hsubx{get\_pinch\_point}: calculate momentum of one pinch orbit
    \item
      \hsubx{D\_orbit}: calculate maximum momentum of trapped particle
    \item
      \hsubx{stagnation\_orbit}: calculate maximum momentum of
      not-forbidden orbit
    \item
      \hsubx{stagnation\_type}: 
    \end{itemize}
    
  \item \hfile{fowdistribution.f90}: Distribution conversion and
    integrated quantities
    \begin{itemize}
    \item \hsub{fI\_Maxwellian}: calculate Maxwellian distribution
    \item \hsub{convert\_fI\_to\_fu}: convert f(I) to f(local)
    \item \hsub{moment\_0th\_order\_COM}: calculate 0th order momentum
      from f(I)
    \item \hsub{moment\_2nd\_order\_COM}: calculate 2nd order momentum
      from f(I)
    \item \hsub{particle\_flux}: calculate total particle flux
    \item \hsub{particle\_flux\_element}: calculate particle flux from
      each component
    \item \hsub{total\_N}: calculate total density
    \item \hsub{effective\_diffusion\_cosfficient}: calculate particle
      diffusion coefficint from particle flux and density gradient
    \end{itemize}

  \item \hfile{fowloop.f90}: fmod{fowloop}: Time loop for execution
    \begin{itemize}
    \item
      \hsub{fow\_loop}:
      \begin{itemize}
      \item fI\_Maxwellian: calculate initial distribution function
      \item fow\_coef: calculate transport coefficients
      \item fow\_calculate\_source: calculate source
      \item loop: call fow\_exec, update coef
      \item calculate density and temperature
      \item output data
      \end{itemize}
    \item
      \hsubx{update\_bulk\_temperature}: calculate temperature
    \item
      \hsubx{output\_data}: output data to file dat/*.txt
    \end{itemize}
        
  \item \hfile{fowexec.f90}: \hmod{fowexec}:  Execution of one time step
    \begin{itemize}
    \item
      \hsub{fow\_exec}: Execution of one time step
      \begin{itemize}
      \item \ttype{mtx\_setup}: initialization of matrix solver
      \item \ttype{fowweight}: setup weight array
      \item \ttype{SET\_FM\_NMA}: setup index array
      \item \ttype{fowsetm}: calculate matrix coefficients in a row
      \item \ttype{IBC\_pinch, IBC\_X\_stagnation,
        IBC\_O\_stagnation}: set internal boundary
        conditions
      \item \ttype{mtx\_set\_matrix, mtx\_set\_vector,
        mtx\_set\_source}: set matrix coefficients, initial solution
        vector, and right-hand-side vector
      \item \ttype{mtx\_solve, mtx\_get\_vector}: solve matrix
        equation and obtain solution vector
      \item \ttype{shadow\_comm\_np, shadow\_comm\_nr}: gather
        solution vector
      \end{itemize}
    \item \hsubx{SET\_FM\_NMA}: setup index array
    \item \hsubx{fowweight}: setup weight array
    \item \hfuncx{fowwegh}: elementary weight function
    \item \hsubx{fowsetm}: calculate matrix coefficients in a row
    \item \hfuncx{Dfow}: matrix coefficients
    \item \hfuncx{w}: weighting 
    \item \hfuncx{check\_external\_boundary}: check within external boundary
    \item \hfuncx{get\_nma}: calculate matrix position
    \item \hsubx{IBC\_pinch}: set pinch boundary condition
    \item \hsubx{IBC\_X\_stagnation}: set X stagnation boundary condition
    \item \hsubx{IBC\_O\_stagnation}  set O stagnation boundary condition
    \item \hfuncx{f\_grid}: evaluate weight on boundary grid point
    \item \hfuncx{nma\_boundary}: check \ttype{nth}, \ttype{np},
      \ttype{nr} within range \emAs{(update the evaluation)}
  \end{itemize}

  \item \hfile{fowcoef.f90}: \hmod{hfowccoef}: Calculation of various
    coefficients
    \begin{itemize}
    \item \hsub{fow\_coef}:
      \begin{itemize}
      \item allocate arrays and initialize
      \item \ttype{convert\_fI\_to\_fu}: set local distribution function
      \item \ttype{fp\_calc}: calculate local coefficients
      \item \ttype{bounce\_average}: bounce average local coefficients
      \end{itemize}
    \item \hsubx{bounce\_average}: bounce average local coefficients
    \item \hsub{transformation\_matrix}: calculation of transformation matrix
    \item \hsub{make\_U\_Dxy}: calculation diffusion coefficints in U frame
    \item \hsub{interpolate\_D\_unlessZero}: calculate interporation
      coefficients
    \end{itemize}
        
  \item \hfile{fowsource.f90}: \hmod{fowsource}: Calculation of source
    terms
    \begin{itemize}
    \item \hsub{fow\_calculate\_source}:
      \begin{itemize}
      \item \ttype{beam\_source}: calculate beam source
      \item set \ttype{sppb}
      \end{itemize}
      \item \hsubx{beam\_source}: calculate and normalize beam source
      \item \hfuncx{construct\_beam}: evaluate \ttype{construct\_beam}
    \end{itemize}
    
  \item \hfile{fowlib.f90}: \hmod{fowlib}: Library for fow
    \begin{itemize}
    \item \hsub{solve\_quadratic\_equation}: quadratic equation solver
    \item \hsub{first\_order\_derivative}: evaluate first-order derivative
    \item \hsub{second\_order\_derivative}: evaluate second-order derivative
    \item \hsub{gauss\_jordan}: matrix solver
    \item \hsub{fow\_cal\_spl}: 1D spline for fow
    \item \hsub{fow\_cal\_spl2D}: 2D spline for fow
    \end{itemize}
  \end{itemize}

\end{document}

