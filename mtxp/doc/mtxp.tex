\documentclass[11pt]{article}
\usepackage{af}
\usepackage{sym}
\usepackage{doc}

\begin{document}
\begin{flushright}
2021-02-11
\end{flushright}

\begin{center}
{\Large \bf Parallel processing and matrix solver interface: mtxp}
\end{center}

\tableofcontents

\bigskip
\begin{center}
\textbf{Notice}
\end{center}
\begin{itemize}
\item
Version number of the software in the following will change owing to
updates of the software.
\end{itemize}
\bigskip

\section{Preparation}

Fortran and C compilers are required to compile 

\subsection{macOS}


\section{Parallel matrix solver library: PETSc}

\subsection{Download by git}

\begin{enumerate}
\item
Make a directory (if already /opt directory exists, skip first two lines)
\begin{quote}
\begin{verbatim}
sudo mkdir /opt
cd /opt
sudo mkdir PETSc
sudo chown username:username PETSc ("username" should be replaced by your user name)
cd PETSC
\end{verbatim}
\end{quote}
\item
  Download by git
\begin{quote}
\begin{verbatim}
git clone -b release https://gitlab.com/petsc/petsc.git petsc
cd petsc
\end{verbatim}
\end{quote}
\end{enumerate}

\subsection{Configure}
\begin{enumerate}
\item
  Create a configure script ``gfortran.py'' by your favarite editor
\item
  Compiler names, ``gcc-mp-10'', ``g++-mp-10'', ``gfortran-mp-10'',
  should be modified according your configuration.  The suffix shown
  as an example ``-mp-10'' is for gnu compiler version 10 installed by
  MacPorts
\item
  If you have already installed MPI libraries, such as MPICH or OpenMP, use
  ``  '--with-mpi-dir=/usr/local/bin','' for the installed directory instead
  of ``  '--download-mpich=1',''.

  
\begin{quote}
gfortran.py
\hrule
\begin{verbatim}
#!/usr/bin/env python

# Build PETSc with gfortran

configure_options = [
  '--with-cc=gcc-mp-10',
  '--with-cxx=g++-mp-10',
  '--with-fc=gfortran-mp-10',
  '--with-shared-libraries=0',
  '--with-cxx-dialect=C++11',
  '--download-mpich=1',
  '--download-hypre=0',
  '--download-fblaslapack=1',
  '--download-spooles=1',
  '--download-superlu=1',
  '--download-metis=1',
  '--download-parmetis=1',
  '--download-superlu_dist=1',
  '--download-blacs=1',
  '--download-scalapack=1',
  '--download-mumps=1'
  ]

if __name__ == '__main__':
  import sys,os
  sys.path.insert(0,os.path.abspath('config'))
  import configure
  configure.petsc_configure(configure_options)
\end{verbatim}
\hrule
\end{quote}
\item
Make ``gfortran.py'' executable
\begin{quote}
\begin{verbatim}
chmod 755 gfortran.py
\end{verbatim}
\end{quote}
\item
Setup environment variables for PETSc
\begin{quote}
\begin{verbatim}
export PETSC_DIR=/opt/PETSc/petsc
export PETSC_ARCH=gfortran
\end{verbatim}
\end{quote}
\item
Configure
\begin{quote}
\begin{verbatim}
./gfortran.py
\end{verbatim}
\end{quote}

\end{enumerate}

\subsection{Compile}
\begin{enumerate}
\item
Compile
\begin{quote}
\begin{verbatim}
make all
\end{verbatim}
\end{quote}
\end{enumerate}

\section{Compile of task/mtxp module}

\subsection{Setup make.mtxp file}
\begin{enumerate}
\item
Goto mtxp directory
\begin{quote}
\begin{verbatim}
cd task/mtxp
\end{verbatim}
\end{quote}
\item
Create a setup file 
\begin{quote}
\begin{verbatim}
cp make.mtxp.org make.mtxp
\end{verbatim}
\end{quote}
\item
Edit the setup file ``make.mtxp''
\begin{itemize}
\item
If MPI is not available, 
remove comment mark ``\#'' on lines 4--9
\item
If MPI is available but PETs not, 
remove comment mark ``\#'' on lines 12--17
\item
If MPI and PETSc are available, 
remove comment mark ``\#'' on lines 20--27
\end{itemize}
\end{enumerate}

\subsection{Modify make.header file}
\begin{enumerate}
\item
Go to task directory
\begin{quote}
\begin{verbatim}
cd ..
\end{verbatim}
\end{quote}
\item
Edit make.header to use lapack and blas libraries for fortran
\begin{itemize}
\item
Near the beginning of the file, remove comment and ajust the path
\begin{quote}
\begin{verbatim}
LAPACK = lapack.f
LIBLA=-L /opt/PETSc/petsc/gfortran/lib -lflapack -lfblas
\end{verbatim}
\end{quote}
\item
In the following, add comment marks
\begin{quote}
\begin{verbatim}
#LAPACK = nolapack.f
#LIBLA =
MODLA95 =
\end{verbatim}
Keep MODLA95 as it is, since lapack95 is not used.
\end{quote}
\item
Go back to mtxp directory
\begin{quote}
\begin{verbatim}
cd mtxp
\end{verbatim}
\end{quote}
\end{itemize}
\end{enumerate}

\subsection{Compile}
\begin{enumerate}
\item
Compile
\begin{quote}
\begin{verbatim}
make clean
make
\end{verbatim}
\end{quote}
\end{enumerate}

\subsection{Test mtxp}
\begin{enumerate}
\item
Test programs solving 1d, 2D and 3D Poisson equation are generated
\begin{itemize}
\item
\verb|testbnd|: Direct band matrix solver (non-parallel)
\item
\verb|testpcg|: Iterative band matrix solver (non-parallel)
\item
\verb|testdmumps|: Direct band matrix solver (parallel using MUMPS)
\item
\verb|testkdsp|: Iterative band matrix solver (parallel using PETSc)
\end{itemize}
\item
Input parameters
\begin{itemize}
\item
\verb|idimen| : number of dimension (i or 2 or 3),  0 for quit
\item
\verb|isiz| : number of mesh point in one dimension
\item
\verb|isource| : source position is all dimensions
\item
\verb|itype| : tyoe of initial guess for PETSc  0..5 (default=0)
\item
\verb|m1| : type of solver (methodKSP) of PETSc  0..13 (default=4)
\item
\verb|m2| : type of preconditioner (methodPC)  0..12 (default=5)
\item
\verb|tolerance| : tolerance in iterative method
\end{itemize}
\item
Example input For parallel processing
\begin{quote}
\begin{verbatim}
mpirun -np 4 ./testdmumps
# INPUT: idimen,isiz,isource,itype,m1,m2,tolerance,idebug= 
1,11,6,0,4,5,1.D-7/
3/
0/
\end{verbatim}
\end{quote}
\end{enumerate}

\section{Compile of task/fp and related modules}

\subsection{Update Makefile}
\begin{enumerate}
\item
Change directory
\begin{quote}
\begin{verbatim}
cd ../fp
\end{verbatim}
\end{quote}
\item
Edit Makefile
\begin{itemize}
\item
To use serial band matrix solver, remove comment mark ``\#'' on lines 4-5.
\item
To use serial iterative solver, remove comment mark ``\#'' on lines 6-7.
\item
To use parallel direct solver MUMPS, remove comment mark ``\#'' on lines 8-9.
\item
To use parallel direct solver library PETSc, remove comment mark ``\#''
on lines 10-11. .
\end{itemize}
\end{enumerate}

\subsection{Compile}
\begin{enumerate}
\item
Compile related modules and task/fp files
\begin{quote}
\begin{verbatim}
make
\end{verbatim}
\end{quote}
\end{enumerate}

\section{Install on Ubuntu}

\subsection{Install of required modules}

\begin{verbatim}
sudo apt-get install gfortran-8
sudo apt-get install gcc-8
sudo apt-get install g++-8

sudo apt-get install emacs
sudo apt-get install git
sudo apt-get install xorg-dev
sudo apt-get install valgrind
sudo apt-get install cmake
sudo apt-get install python
\end{verbatim}

\subsection{Install of MPICH}

\begin{enumerate}
\item
Download of mpich-3.3.1.tar.gz \qquad (\textit{See 1.1})
\item
Expand at ~/soft/mpich \qquad (\textit{See 1.1})
\item
Configure by executing ``./run''  \qquad (\textit{See 1.2})

\begin{quote}
run:
\hrule
\begin{verbatim}
CC=gcc-8 CFLAGS=''-m64'' CXX=g++-8 CXXFLAGS=''-m64'' FC=gfortran-8
FFLAGS=''-m64'' ./configure --prefix=/usr/local/mpich331-gfortran-gcc8
--enable-cxx --enable-fast --enable-romio --disable-shared
\end{verbatim}
\hrule
\end{quote}

\item
Compile and install \qquad (\textit{See 1.3})
\end{enumerate}

\subsection{Install of PETSc}

\begin{enumerate}
\item
Download of petsc-3.11.3.tar.gz \qquad (\textit{See 2.1})
\item
Expand at /opt/PETSc/ \qquad (\textit{See 2.1})
\item
Setup environment variables \qquad (\textit{See 2.2})
\begin{quote}
\begin{verbatim}
export PETSC_DIR=/opt/PETSc/petsc-3.11.3
export PETSC_ARCH=gfortran
\end{verbatim}
\end{quote}
\item
Configure by gfortran.py  \qquad (\textit{See 2.2})
\item
Execute fortran.py

\begin{quote}
gortran.py:
\hrule
\begin{verbatim}
#!/usr/bin/env python

# Build PETSc, with gfortran

configure_options = [
  '--with-mpi=1',
  '--with-mpi-dir=/usr/local/mpich331-gfortran-gcc8’,
  '--with-shared-libraries=0',
  '--with-cxx-dialect=C++11',
  '--download-mpich=0',
  '--download-hypre=0',
  '--download-fblaslapack=1',
  '--download-spooles=1',
  '--download-superlu=1',
  '--download-metis=1',
  '--download-parmetis=1',
  '--download-superlu_dist=1',
  '--download-blacs=1',
  '--download-scalapack=1',
  '--download-mumps=1'
  ]

if __name__ == '__main__':
  import sys,os
  sys.path.insert(0,os.path.abspath('config'))
  import configure
  configure.petsc_configure(configure_options)
\end{verbatim}
\hrule
\end{quote}
\end{enumerate}

\end{document}

