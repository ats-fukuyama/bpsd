%/home/bateman/Codes/Mixed_Model_2003/Mixed_Bohm_Gyro_Bohm/source/mixed_Bohm_gyro_Bohm.f
%
%  This is a LaTeX ASCII file.  To typeset document type:
% latex theory
%  To extract the Fortran source code, type:
% python s2tex.py mixed_model.tex
%
% The following lines control how the LaTeX document is typeset

\documentstyle{article}
\headheight 0pt \headsep 0pt  \topmargin 0pt  \oddsidemargin 0pt
\textheight 9.0in \textwidth 6.5in
\begin{document}

\begin{center}
\Large {\bf Mixed Bohm/gyro-Bohm Transport Model} \\
\vspace{1pc} \normalsize
Glenn Bateman, Alexei Pankin, and Arnold Kritz \\
Lehigh University Physics Department \\
16 Memorial Drive East, Bethlehem PA 18015 \\
bateman@fusion.physics.lehigh.edu \\
pankin@fusion.physics.lehigh.edu \\
kritz@fusion.physics.lehigh.edu
\end{center}

The \verb'mixed_Bohm_gyro_Bohm' module is used to compute
anomalous plasma transport coefficients using the Mixed-Bohm/gyro-Bohm
transport model.
The Mixed Bohm/gyro-Bohm anomalous transport model (also called the JET
or JETTO transport model) contains Bohm and gyro-Bohm contributions.
The Bohm contribution, which is linear in the gyro-radius, is a
non-local transport model, in which the transport throughout the plasma
depends on a finite difference approximation to the electron temperature
gradient at the edge of the plasma.  The gyro-Bohm contribution is a
local transport model, which is proportional to the square of the
gyro-radius.  The version of the Mixed Bohm/gyro-Bohm model in this
module is described in detail in Nuclear Fusion 38 (1998) 1013.
The magnetic and flow shear stabilization are described in Plasma
Physics and Controlled Fusion 44 (2002) A495.

There are two subroutines in this module:

The \verb'mixed_model' subroutine computes arrays of effective
diffusivities using arrays representing the profiles from the
center to the edge of the plasma.

The \verb'mixed' subroutine computes scalar effective diffusivities
given the plasma properties at a given point, and also given the
finite difference approximation to the electron temperature gradient
at the edge of the plasma.

The version of the Mixed-Bohm/gyro-Bohm
model that is used in this subroutine is described
in Nuclear Fusion {\bf 38} (1998) 1013 by
M.~Erba, T.~Aniel, V.~Basiuk, A.~Becoulet, and X.~Litaudon.
Both the electron and ion thermal diffusivities consist
of two terms.  One term has Bohm scaling
\begin{equation}
\chi^{\rm Bohm} \equiv \rho_s c_s  q^2
\frac{ a ( d p_e / d r ) }{ p_e } \Delta_{T_e},
\label{eq-Bohm}
\end{equation}
while the other term has gyro-Bohm scaling
\begin{equation}
\chi^{\rm gyro-Bohm} \equiv \frac{ \rho_s^2 c_s }{ a }
  \frac{ a ( d T_e / d r ) }{ T_e }.
\label{eq-gyro-Bohm}
\end{equation}
The notation is described in Table \ref{table-notation}.
In the Bohm diffusivity expression, $ \Delta_{T_e} $ is a finite
difference approximation to the normalized temperature electron
temperature difference at the plasma edge
\begin{equation}
\Delta_{T_e} \equiv \frac{T_e(r/a=0.8) - T_e(r/a=1)}{T_e(r/a=1)}.
\label{eq-Delta}
\end{equation}
The resulting anomalous
ion and electron thermal diffusivities are constructed from the
sum of these Bohm and gyro-Bohm terms, with empirically determined
coefficients \cite{erb98}
\begin{equation}
\chi_i^{\rm JET} = 1.6 \times 10^{-4} \chi^{\rm Bohm} +
 1.75 \times 10^{-2} \chi^{\rm gyro-Bohm}
\label{eq-chi-i}
\end{equation}
\begin{equation}
\chi_e^{\rm JET} = 8 \times 10^{-5} \chi^{\rm Bohm} +
 3.5 \times 10^{-2} \chi^{\rm gyro-Bohm}
\label{eq-chi-e}
\end{equation}
and the hydrogenic charged particle diffusivity is given by
\begin{equation}
D^{\rm JET} \propto  \frac{ \chi_i \chi_e }{ \chi_i + \chi_e }
\label{eq-D}
\end{equation}

It should be noted that the Mixed-Bohm/gyro-Bohm model is not a
local transport model --- it cannot be evaluated with just the
local plasma parameters.
The Bohm contribution to the core transport $ \chi^{\rm Bohm} $
is proportional to a finite difference approximation to the edge
electron temperature gradient through the term $ \Delta_{T_e} $.


\begin{table}
\caption{Notation}

% \renewcommand{\arraystretch}{0.7}

\label{table-notation}
\begin{center}
\begin{tabular}{lll}
\hline \hline
Variable \hspace{3pt} & Units \hspace{3pt} & Meaning \\
\hline % \\[-2mm]
$a$& m & minor radius (half-width) of plasma \\
$B_T$ & Tesla & vacuum toroidal magnetic field at \\ &
   & major radius $R$ along flux surface \\
$c_s$ & m/s & $ [k_b T_e / m_i]^{1/2} $ speed of sound \\
$D$ & m$^2$/s & effective charged particle diffusivity \\
 & & charged particle flux divided by density gradient \\
$e$ & C & electron charge \\
$I_p$ & MA & toroidal plasma current \\
$k_b$ &  & conversion from keV to Joules \\
$m_i$ & kg  & average ion mass \\
$n_e$ & m$^{-3}$ & electron density \\
$q$  & & magnetic q-value \\
$r$& m & minor radius (half-width) of each flux surface \\
$R$ & m & major radius to geometric \\
 & & center of flux surface \\
$T_e$ & keV & electron temperature \\
$T_i$ & keV & ion temperature \\
$Z_{\rm eff}$ & & $\sum_s n_s Z_s^2 / n_e $ summed over each species \\
$\chi$  & m$^2$/s & effective thermal diffusivity \\
 & & heat flux divided by density time temperature gradient \\
$\delta$ &  & plasma triangularity \\
$\kappa$ &  & plasma elongation \\
$\nu_{*}$ & & collision frequency divided by bounce frequency \\
$\rho_s$ & m &  gyroradius [$ c_s m_i / ( e B_T ) $] \\
$\rho_*$ & & normalized gyroradius ($ \rho_s / a $) \\
\hline \hline
\end{tabular}
\end{center}
\end{table}

\begin{verbatim}
c@mixed_model.tex
c--------1---------2---------3---------4---------5---------6---------7-c
c
      module mixed_Bohm_gyro_Bohm
      private
      public mixed_model
c
c..physical constants
c
      real, parameter  :: zckb = 1.60210e-16 ! energy conversion factor [Joule/keV]
      real, parameter  :: zcme = 9.1091e-31  ! electron mass [kg]
      real, parameter  :: zcmp = 1.67252e-27 ! proton mass [kg]
      real, parameter  :: zce  = 1.60210e-19 ! electron charge [Coulomb]
c
      contains
c
      subroutine mixed_model (
     &   rminor,  rmajor,  tekev,   tikev,   q
     & , btor,    aimass,  charge,  wexbs
     & , grdte,   grdne,   shear
     & , t_e_kev_edge, rminor_edge, npoints
     & , chi_i_mix,  themix,   thdmix
     & , thigb,   thegb,    thibohm, thebohm
     & , ierr
     & , lflowshear)
c
c
c    All the following 1-D arrays are assumed to be defined on flux
c    surfaces called zone boundaries where the transport fluxes are
c    to be computed.  The number of flux surfaces is given by npoints
c    (see below).  For example, if you want to compute the transport
c    on only one flux surface, set npoints = 1.
c
c  Input arrays:
c  -------------
c
c  rminor(jz)   = minor radius (half-width) of zone boundary [m]
c  rmajor(jz)   = major radius to geometric center of zone bndry [m]
c
c  tekev(jz)    = T_e (electron temperature) [keV]
c  tikev(jz)    = T_i (temperature of thermal ions) [keV]
c  q(jz)        = magnetic q-value
c  btor(jz)     = ( R B_tor ) / rmajor(jz)  [tesla]
c
c  aimass(jz)   = mean atomic mass of main thermal ions [AMU]
c               = ( sum_i n_i M_i ) / ( sum_i n_i ) where
c                 sum_i = sum over all ions, each with mass M_i
c
c  charge(jz)   = charge number of the main thermal ions
c                 = 1.0 for hydrogenic ions
c
c  wexbs(jz)    = ExB shearing rate in [rad/s]
c
c    All of the following normalized gradients are at zone boundaries.
c    r = half-width, R = major radius to center of flux surface
c
c  grdte(jz) = -R ( d T_e / d r ) / T_e
c  grdne(jz) = -R ( d n_e / d r ) / n_e
c  shear(jz) =  r ( d q   / d r ) / q    magnetic shear
c
c    The following variables are scalars:
c
c  t_e_kev_edge = electron temperature at the edge of the plasma [keV]
c  rminor_edge  = minor radius at the edge of the plasma [m]
c
c  npoints = number of values of jz in all of the above arrays [integer]
c
c  Control variable (input):
c  -------------------------
c
c  lflowshear = 0 for no magnetic and flow shear stabilization
c             = 1 to use magnetic and flow shear stabilzation by
c                 [T.J. Tala et al Plasma Phys. Controlled Fusion 44
c                 (2002) A495]
c
c  Output:
c  -------
c
c    The following effective diffusivities are given in MKS units m^2/sec
c
c  chi_i_mix(jz) = total ion thermal diffusivity from the MIXED model
c  themix(jz) = total electron thermal diffusivity from the MIXED model
c  thdmix(jz) = total hydrogenic ion diffusivity from the MIXED model
c
c    The following contributions to the effective diffusivities are
c  for diagnostic purposes:
c
c  thigb(jz) = gyro-Bohm contribution to the ion thermal diffusivity
c  thegb(jz) = gyro-Bohm contribution to the electron thermal diffusivity
c
c  thibohm(jz) = Bohm contribution to the ion thermal diffusivity
c  thebohm(jz) = Bohm contribution to the electron thermal diffusivity
c
c  ierr    = returning with value .ne. 0 indicates error
c
c***********************************************************************
c
c-----------------------------------------------------------------------
c
c  Compile this routine and routines that it calls with a compiler
c  option, such as -r8, to convert real to double precision when used on
c  workstations.
c
c-----------------------------------------------------------------------
c
c  External dependencies:
c
c  Call tree: mixed_model calls the following routines
c
c  mixed             - Computes diffusivity from the MIXED model
c
c-----------------------------------------------------------------------

      implicit none
c
c-----------------------------------------------------------------------
c..input variables
c
      integer, intent(in)  :: npoints ! number of radial points
c
      real, intent(in) ::
     &   rminor(*),  rmajor(*)
     & , tekev(*),   tikev(*),    q(*),       btor(*)
     & , wexbs(*),   aimass(*),   charge(*)
c
      real, intent(in) ::                    ! gradients
     &   grdne(*)
     & , grdte(*), shear(*)
c
      real, intent(in) ::  t_e_kev_edge, rminor_edge
c
c  optional input switch for flow-shear stabilization option
c
      integer, intent(in), optional :: lflowshear
c
c-----------------------------------------------------------------------
c..output variables
c
      real, intent(out) ::
     &   thigb(*),   thegb(*),    thibohm(*), thebohm(*)
c
      real, intent(out) ::
     &   chi_i_mix(*),   themix(*),  thdmix(*)
c
      integer, intent(out) :: ierr
c
c-----------------------------------------------------------------------
c..local variables
c
      integer  :: jz, j1, j2, jm
c
c  npoints1 = value of jz corresponding to 80 0f the normalized radius
c
      integer  :: npoints1, llflow

      integer  :: lswitch5
c
      real     :: zte_p8, zte_edge, zi
c
c..local variables connected to the mixed module
c

      real :: zq,    zsound,     zgyrfi,    zrhos, zra
     & , zchii, zchie, zdhyd
     & , zrmaj, zaimass, zcharge,   zbtor, zrmin
     & , zgte,  zti,  zte, zgne
     & , zrlpe, zshear, zgradte
     & , zchbe, zchbi, zchgbe, zchgbi
c
c.. variables for exb model
c
      real  zwexb
c
c  zwexb    = local copy of ExB shearing rate
c
c..initialize arrays
c
c
      thigb(1:npoints)  = 0.
      thegb(1:npoints)  = 0.
      thibohm(1:npoints)= 0.
      thebohm(1:npoints)= 0.
      chi_i_mix(1:npoints) = 0.
      themix(1:npoints) = 0.
      thdmix(1:npoints) = 0.
c
c..initialize switches
c
      if (present(lflowshear)) then ! flow shear correction
        llflow = lflowshear
      else
        llflow = 0
      endif
c
c-----------------------------------------------------------------------
c
c..physical constants
c
c     define the jz value at 0.8 0f normalized radius
c
      npoints1 = int(npoints*0.8)
c
c.. start the main do-loop over the radial index "jz"..........
c
c
      do 300 jz = 1, npoints
c
c  compute scalar quantities necessary for mixed module
c
        zshear   = shear(jz)
        zte      = tekev(jz)
        zti      = tikev(jz)
        zaimass  = aimass(jz)
        zcharge  = charge(jz)
        zrmin    = rminor(jz)
        zrmaj    = rmajor(jz)
        zgte     = grdte(jz)
        zgne     = grdne(jz)
        zgradte   = abs( ( zgte / zrmaj ) * zte)
        zbtor    = btor(jz)
        zq       = q(jz)
        zra      = rminor(jz) / rminor(npoints)
        zrlpe    = (zgte + zgne)
        zte_p8   = tekev(npoints1)
        zte_edge = tekev(npoints)
c
        zwexb = wexbs(jz)
c
c
      call mixed(
     &  zbtor,          zgradte,        zq,
     &  zra,            zrlpe,
     &  zrmaj,          zshear,         zte,            zte_p8,
     &  zte_edge,       zti,            zwexb,          zaimass,
     &  zcharge,        llflow,
     &  zchie,          zchii,          zdhyd,
     &  zchbe,          zchbi,          zchgbe,         zchgbi,
     &  ierr)
c
c
c  If ierr not equal to 0 an error has occured
c
        if (ierr .ne. 0) return
c
c  compute effective diffusivites for diagnostic purposes only
c
         thdmix(jz)  = zdhyd
         themix(jz)  = zchie
         chi_i_mix(jz)  = zchii
c
c  put value of gbohm and bohm terms into kb and rb terms for
c  diagnostic purposes
c
         thebohm(jz) = zchbe
         thibohm(jz) = zchbi
c
         thegb(jz)   = zchgbe
         thigb(jz)   = zchgbi
c
c..end of mixed model
c
c
 300  continue
c
c
c   end of the main do-loop over the radial index, "jz"----------
c
      return
      end subroutine mixed_model
c--------1---------2---------3---------4---------5---------6---------7-c
c
\end{verbatim}
\newpage
\begin{center}
{\LARGE Subroutine for Computing Particle and Energy Fluxes\\ \vskip8pt
Using the JET Mixed Bohm/gyro-Bohm\\ \vskip8pt
Transport Model
}\vskip1.0cm
Version 1.2: 22 August 2003 \\
Implemented by M. Erba, G. Bateman, A. H. Kritz, T. Onjun, and A. Pankin\\
 Lehigh University
\end{center}
For questions about this routine, please contact: \\
Arnold Kritz, Lehigh: {\tt kritz@plasma.physics.lehigh.edu}\\
Glenn Bateman, Lehigh: {\tt glenn@plasma.physics.lehigh.edu}\\
Thawatchai Onjun, Lehigh: {\tt onjun@fusion.physics.lehigh.edu}\\
Alexei Pankin, Lehigh: {\tt pankin@fusion.physics.lehigh.edu}\vskip8pt


\begin{verbatim}
c@mixed.tex
c--------1---------2---------3---------4---------5---------6---------7-c
c
      subroutine mixed(
     &  btor,           gradte,         q_safety,
     &  ra,             rlpe,
     &  rmaj,           shear,          tekev,          te_p8,
     &  te_edge,        tikev,          wexb,           aimass,
     &  zi,             lflow,
     &  chi_e,          chi_i,          d_hyd,
     &  chi_e_bohm,     chi_i_bohm,
     &  chi_e_gyro_bohm, chi_i_gyro_bohm,
     &  ierr)
c
c Revision History
c ----------------
c      date          description
c
c   25-Aug-2003      New version of flow shear sabilization correction
c                    Converted to Fortran-90 module
c   27-Jul-2000      Major rewrite by Bateman
c   12-Jun-2000      gyro-Bohm term restored
c   06-May-1999      Revised as module by Thawatchai Onjun
c   24-Feb-1999      First Version by Matteo Erba
c
c Mixed Transport Model (by M.Erba, V.V.Parail, A.Taroni)
c
c Inputs:
c    btor:      Toroidal magnetic field strength in Tesla
c                 at geometric center along magnetic flux surface
c    gradte:    Local electron temperature gradient [keV/m]
c    q_safety:  Local value of q, the safety factor
c    ra:        Normalized minor radius r/a
c    rlpe:      a/L_pe, where a is the minor radius of the plasma,
c                    and L_pe is the local electron pressure scale length
c    rmaj:      Major radius [m]
c    shear:     Local magnetic Shear
c    tekev:     Local electron temperature [keV]
c    tikev:     Ion temperature [keV]
c    te_p8:     Electron temperature at r/a = 0.8
c    te_edge:   Electron temperature at the edge [r/a = 1]
c    wexb:      EXB Rotation
c                    "Effects of {ExB} velocity shear and magnetic shear
c                    on turbulence and transport in magnetic confinement
c                    devices", Phys. of Plasmas, 4, 1499 (1997).
c    zi:        Ion charge
c
c Outputs:
c    chi_e:      Total electron thermal diffusivity [m^2/sec]
c    chi_i:      Total ion thermal diffusivity [m^2/sec]
c    d_hyd:      Hydrogenic ion particle diffusivity [m^2/sec]
c
c    chi_e_bohm:      Bohm contribution to electron thermal diffusivity
c                         [m^2/sec]
c    chi_i_bohm:      Bohm contribution to ion thermal diffusivity [m^2/sec]
c    chi_e_gyro_bohm: gyro-Bohm contribution to electron thermal diffusivity
c                 [m^2/sec]
c    chi_i_gyro_bohm:    gyro-Bohm contribution to ion thermal diffusivity
c                 [m^2/sec]
c    ierr:      Status code returned; 0 = OK, .ne.0 indicates error
c
c Coefficients set internally:
c
c    alfa_be:   Bohm contribution to electron thermal diffusivity
c    alfa_bi:   Bohm contribution to ion thermal diffusivity
c    alfa_gbe:  gyro-Bohm contribution to electron thermal diffusivity
c    alfa_gbi:  gyro-Bohm contribution to ion thermal diffusivity
c
c    coef1:     First coefficient for empirical hydrogen diffusivity
c    coef2:     Second coefficient for empirical hydrogen diffusivity
c
c Other internal variables:
c
c    gamma:     The characteristic growthrate for the ITG type of
c                    electrostatic turbulence
c    func:      Function for EXB and magnetic shear stabilization
c
      IMPLICIT NONE
c
c Declare variables
c
c..Input variables
c
      real, intent(in) ::
     &  btor,           ra,             rmaj,           rlpe,
     &  shear,          tekev,          te_p8,          te_edge,
     &  tikev,          gradte,         aimass,
     &  wexb,           zi,             q_safety
c
      integer, intent(in)  :: lflow
c
c..Output variables
c
      real, intent(out)    ::
     &  chi_e,               chi_i,               d_hyd,
     &  chi_e_gyro_bohm,     chi_i_gyro_bohm,
     &  chi_e_bohm,          chi_i_bohm
c
c
      integer, intent(out) :: ierr
c
c..Local variables
c
      REAL
     &  alfa_be,        alfa_bi,        alfa_gbe,       alfa_gbi,
     &  alfa_d,         chi0,           coef1,          coef2,
     &  delta_edge,
     &  em_i,           func,           gamma,
     &  omega_ce,       omega_ci,       rho,
     &  v_sound,        vte_sq,         vti,
     &  zepsilon
c
c
c..initialize diffusivities
c
      chi_e = 0.0
      chi_i = 0.0
      d_hyd = 0.0
c
      chi_e_bohm = 0.0
      chi_i_bohm = 0.0
      chi_e_gyro_bohm = 0.0
      chi_i_gyro_bohm = 0.0
c
c check input for validity
c
      zepsilon = 1.e-10
c
      ierr = 0
      if ( tekev .lt. zepsilon ) then
         ierr=1
         return
      elseif ( tikev .lt. zepsilon ) then
         ierr=2
         return
      elseif ( te_p8 .lt. zepsilon ) then
         ierr=3
         return
      elseif ( te_edge .lt. zepsilon ) then
         ierr=4
         return
      elseif ( rmaj  .lt. zepsilon ) then
         ierr=5
         return
      endif
\end{verbatim}

\section{The Mixed Bohm/gyro-Bohm model}

The Mixed Bohm/gyro-Bohm transport model derives from an
originally purely Bohm-like model for electron transport
developed for the JET Tokamak\cite{tar94}. This preliminary model has
subsequently been extended to describe ion transport\cite{erb95},
and a gyro-Bohm term has been added in order to simulate data from
different machines\cite{erb98}.

\subsection{Bohm term}

The mixed model is derived using the dimensional analysis approach,
whereby the diffusivity in a Tokamak plasma can be written as:\hfill
\[ \chi = \chi_0 F(x_1, x_2, x_3, ...)\]
where $\chi_0$ is some basic transport coefficient and F is a function
of the plasma dimensionless parameters $(x_1,\ x_2,\ x_3,\ ...)$. We choose
for $\chi_0$ the Bohm diffusivity:\hfill
\[ \chi_0 = \frac{cT_e}{eB}\]

The expression of the dimensionless function F is chosen according to
the following criteria:\
\begin{itemize}
\item{The diffusivity must be bowl-shaped, increasing towards the plasma
boundary}
\item{The functional dependencies of F must be in agreement with
scaling relationships of the global confinement time, reflecting
trends such as power degradation and linear dependence on plasma
current}
\item{The diffusivity must provide the right degree of resilience
of the temperature profile}
\end{itemize}

It easily shown that a very simple expression of F that satisfies
the above requirements is:\hfill
\[ F = q^2/|L_{pe}^*|\]

where q is the safety factor and $L_{pe}^*=p_e(dp_e/dr)^{-1}/a$, being a the
plasma minor radius. The resulting expression of the diffusivity
can be written as:\hfill
\[ \chi \propto |v_d| \Delta G\]
where $v_d$ is the plasma diamagnetic velocity, $\Delta=a$ and $G=q^2$,
so that it is clear that this model represents transport due to
long-wavelength turbulence.\\
The evidence coming up from the simulation of non-stationary
JET experiments \cite{erb97}(such as ELMs, cold pulses, sawteeth, {\sl etc}.)
suggested that the above Bohm term should depend non-locally
on the plasma edge conditions through the temperature
gradient averaged over a region near the edge:\hfill

\[ <L_{T_e}^*>_{\Delta V}^{-1} = \frac{T_e(x=0.8) - T_e(x=1)}{T_e(x=1)}\]
where x is the normalized toroidal flux coordinate. The final
expression of the Bohm-like model is:\hfill

\[ \chi_{e,i}^B = \alpha_{Be,i} \frac{cT_e}{eB} L_{pe}^{*-1} q^2 <L_{T_e}^*>_{\Delta V}^{-1}\]
where $\alpha_{e,i}^B$ is a parameter to be determined empirically,
both for ions and electrons.\hfill

\subsection{gyro-Bohm term}

The Bohm-like expression so derived proved to be very successful
in simulating JET discharges, but failed badly in smaller Tokamaks
such as START\cite{roa96}.\\
For this reason a simple gyro-Bohm-like term, also based on
dimensional analysis, was added:

\[ \chi_{e,i}^{gB} = \alpha_{e,i}^gB \frac{cT_e}{eB} L_{Te}^{*-1} \rho^*\]
where $\rho^*$ is the normalized larmor radius:

\[ \rho^* =  \frac {M^{1/2}cT_e^{1/2}}{aZ_ieB_t}\]

This expression is what can be expected from small scale
drift-wave turbulence. It is important to note that in large
Tokamaks such as JET and TFTR the gyro-Bohm term is negligible,
while in smaller machines, with larger values of $\rho^*$, the
gyro-Bohm term can play a role especially near the plasma centre.\hfill

\subsection{Final Model}
The resulting expressions of the diffusivities are:

\[ \chi_{e,i}=\chi_{Be,i}+\chi_{gBe,i}\]
where the Bohm and gyro-Bohm terms are defined above and the adopted values
of the empirical parameters are:\hfill

\[ \alpha_{Be} = 8\times10^{-5} ,\alpha_{Bi} = 2\times\alpha_{Be}\]
\[ \alpha_{gBe} = 3.5\times10^{-2} , \alpha_{gBi} = \alpha_{gBe}/2\]\vskip8pt


The following definitions are used:\vskip8pt

\begin{tabular}{lll}
el\_mass &electron mass    &$m_e$ \\
c       &velocity of light &$c$ \\
e       &electron charge  &$e$\\
vte\_sq &eletron thermal velocity squared &$v_{\rm te}^2$\\
omega\_ce &electron cyclotron frequency &$\omega_{\rm ce}$\\
chi0 &Bohm diffusitivity & $\chi_0$\\
em\_i & ion atomic mass [kg] & $M_{\rm i}$\\
%v_sound & ion
\end{tabular}\vskip8pt

\begin{verbatim}
c *
c * Definition of the mixed model
c *
c
c Coefficients
c
      alfa_be  =  8.00000000000000E-05
      alfa_bi  =  1.60000000000000E-04
      alfa_gbe =  3.50000000000000E-02
      alfa_gbi =  1.75000000000000E-02
c
      coef1    =  1.00000000000000E+00
      coef2    =  3.00000000000000E-01
c
c Calculate chi0
c
      vte_sq    = tekev * zckb / zcme
      omega_ce  = zce * btor / zcme
      chi0      = vte_sq / omega_ce
c
c Calculate chibohm
c
      delta_edge = abs((te_p8 - te_edge) / te_edge)
      chi_e_bohm
     &   = abs(alfa_be * chi0 * rlpe * (q_safety**2) * delta_edge)
      chi_i_bohm
     &   = abs(alfa_bi * chi0 * rlpe * (q_safety**2) * delta_edge)
c
c Calculate chi_gyrobohm
c
      em_i      = aimass * zcmp
      v_sound   = sqrt (tekev * zckb /em_i) !                   [m/sec]
      omega_ci  = zi * zce * btor / em_i
      rho       = v_sound / omega_ci
      chi_e_gyro_bohm = abs(alfa_gbe * chi0 * gradte * rho / tekev)
      chi_i_gyro_bohm = abs(alfa_gbi * chi0 * gradte * rho / tekev)
c
c Calculate function for EXB and magnetic shear stabilization
c   use lflow = 1 for the stabilization term described by
c   T.J. Tala et al Plasma Phys. Controlled Fusion 44 (2002) A495
c
c
      vti       = sqrt (2.0 * tikev * zckb / em_i)
      gamma     = vti / (q_safety * rmaj)
c
      func = 1.0
      if ( lflow .eq. 1 ) then
        func = -0.14 + shear - 1.47 * abs(wexb*rmaj/vti)
      elseif ( lflow .eq. 2 ) then
        func = 0.1 + shear - abs (wexb / gamma)
      endif
c
      if ( func .lt. 0.0 ) then
          chi_e_bohm = 0.0
          chi_i_bohm = 0.0
      endif
c
c Now determine the actual electron and ion thermal and particle
c diffusivities.
c
      chi_e      = chi_e_bohm + chi_e_gyro_bohm
      chi_i      = chi_i_bohm + chi_i_gyro_bohm
      alfa_d    = coef1 + (coef2 - coef1) * ra
      if (abs (chi_e + chi_i) .lt. 1e-10) then
        d_hyd = 0.0
      else
        d_hyd  = abs(alfa_d * chi_e * chi_i / (chi_e + chi_i))
      endif
c
c The impurity diffusivity is not included in the mixed
c model described in Ref. [1].
c
        return
        end subroutine mixed
        end module mixed_Bohm_gyro_Bohm
\end{verbatim}

%**********************************************************************c

\begin{thebibliography}{99}
\bibitem{tar94}
A. Taroni, M. Erba, E. Springmann and Tibone F.,
{\em Plasma Physics and Controlled Fusion,} {\bf 36} (1994) 1629.
\bibitem{erb95}
M. Erba, V. Parail, E. Springmann and A. Taroni,
{\em Plasma Physics and Controlled Fusion,} {\bf 37} (1995) 1249.
\bibitem{erb98}
M. Erba, et al.,
{\em Nuclear Fusion,} {\bf 38} (1998) 1013.
\bibitem{erb97}
M. Erba, et al.,
{\em Plasma Physics and Controlled Fusion,} {\bf 39} (1997) 261.
\bibitem{roa96}
C.M. Roach,
{\em Plasma Physics and Controlled Fusion,} {\bf 38} (1996) 2187.
\end{thebibliography}

%**********************************************************************c
\end{document}
